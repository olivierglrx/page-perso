\documentclass[a4paper, 11pt,reqno]{article}
\input{/Users/olivierglorieux/Desktop/BCPST/2020:2021/preambule.tex}
\geometry{hmargin=1.5cm,vmargin=2.5cm }
\usepackage{enumitem}
\newif\ifshow
\showfalse
\input{/Users/olivierglorieux/Desktop/BCPST/2021:2022/ifshow.tex}



\author{Olivier Glorieux}


\begin{document}

\title{DM4 \\
}
\begin{exercice}
Résoudre dans $\bC$ l'équation d'inconnue $z$: 
$$\left(\frac{z-2i}{z+2i}\right)^3+\left(\frac{z-2i}{z+2i}\right)^2+\left(\frac{z-2i}{z+2i}\right)+1=0$$
\end{exercice}


\begin{correction}
On pose, $Z =\left(\frac{z-2i}{z+2i}\right)$, l'équation devient alors: 
$$Z^3+Z^2+Z+1=0.$$
On  remarque que $-1$ est une racine du polynôme,   $Z^3+Z^2+Z+1$, qui se factorise alors en 
$(Z+1)(Z^2+1)$. $Z^2+1 =(Z-i)(Z+i)$ et on  a donc 
$$Z^3+Z^2+Z+1 =(Z+1)(Z-i)(Z+i).$$

\begin{enumerate}
\item Pour $Z=-1 \Longleftrightarrow \left(\frac{z-2i}{z+2i}\right)=-1$, on obtient 
$z-2i =-z-2i$ soit $$z=0.$$
\item Pour $Z=i \Longleftrightarrow \left(\frac{z-2i}{z+2i}\right)=i$, on obtient 
$z-2i =iz-2$. Soit $z(1-i) = -2+2i$, donc 
$$z=-2$$


\item Pour $Z=-i \Longleftrightarrow \left(\frac{z-2i}{z+2i}\right)=-i$, on obtient 
$z-2i =-iz+2$ soit $z(1+i) =2+2i$ donc 
$$z=2$$



Les solutions de l'équation sont donc 
\conclusion{$\cS=\{ -2,0,  2 \}$}

\end{enumerate}
\end{correction}


\begin{exercice}
 Soit  $n\in \N$. On définit la somme pour tout $x \in ]0, 2\pi[$ :
$$\ddp Z(x) =\sum_{k=0}^n e^{ikx}.$$
\begin{enumerate}
\item Montrer par récurrence que $Z(x) = \frac{1-e^{(n+1)ix}}{1-e^{ix}}$.

On suppose que $n\geq 2$, on pose: 
$$\ddp S_n = \sum_{k=1}^{n-1} \sin \left( \frac{k\pi}{n}\right).$$
\item Justifier que $\ddp S_n = \sum_{k=0}^{n} \sin \left( \frac{k\pi}{n}\right).$\\
\item Prouver que : $\ddp S_n = \frac{1}{ \tan \left( \frac{\pi}{2n}\right)}\, $.\\
\item En déduire la valeur de $\ddp \tan \left( \frac{\pi}{8}\right).$\\
\item Déterminer $\ddp \lim_{n\tv \infty} \frac{S_n}{n}\, .$
\end{enumerate}
\end{exercice}


\begin{correction}
\begin{enumerate}
\item C'est l'exercice 2 du TD 1 - Récurrence, où $q=e^{ix}$. 
\item $\ddp \sum_{k=0}^{n} \sin \left( \frac{k\pi}{n}\right) =  \ddp \sum_{k=1}^{n-1} \sin \left( \frac{k\pi}{n}\right)  + \sin\left( \frac{0\pi}{n}\right) + \sin\left( \frac{n\pi}{n}\right) $
Or $ \sin\left( \frac{0\pi}{n}\right)=0$ et $ \sin\left( \frac{n\pi}{n}\right)=\sin(\pi)=0 $. Donc 
$$\ddp S_n = \sum_{k=0}^{n} \sin \left( \frac{k\pi}{n}\right).$$
\item  On a $\ddp Z(\frac{\pi}{n})= \sum_{k=0}^n e^{ik\frac{\pi}{n}}.$ D'après la question $1$ : 
\begin{align*}
\sum_{k=0}^n e^{ik\frac{\pi}{n}}&= \frac{1-e^{i(n+1)\frac{\pi}{n}}}{1-e^{i\frac{\pi}{n}}}\\
												&=\frac{1-e^{i\pi +i\frac{\pi}{n}}}{1-e^{i\frac{\pi}{n}}}\\
												&=\frac{1+e^{i\frac{\pi}{n}}}{1-e^{i\frac{\pi}{n}}}\\
												&=\frac{e^{i\frac{\pi}{2n}}  \left(e^{-i\frac{\pi}{2n}}+e^{i\frac{\pi}{2n}}\right)  }{   e^{i\frac{\pi}{2n}}  \left(e^{-i\frac{\pi}{2n}}-e^{i\frac{\pi}{2n}}\right)  }\\
											&=\frac{  \left(e^{-i\frac{\pi}{2n}}+e^{i\frac{\pi}{2n}}\right)  }{     \left(e^{-i\frac{\pi}{2n}}-e^{i\frac{\pi}{2n}}\right)  }\\
										&=\frac{ 2\cos(\frac{\pi}{2n} ) }{  2i \sin(\frac{\pi}{2n})} \\
										&=\frac{ 1  }{  i \tan(\frac{\pi}{2n})} \\										
\end{align*}
De plus 
\begin{align*}
\Im(Z(x))&= \Im\left(\sum_{k=0}^n e^{ikx}\right)\\
							&=\sum_{k=0}^n \Im(e^{ikx})\\
							&=\sum_{k=0}^n \sin(kx))
\end{align*}
Donc $S_n= \Im( Z(\frac{\pi}{n})) = \Im(\frac{ 1  }{  i \tan(\frac{\pi}{2n})}) =\frac{ 1  }{  \tan(\frac{\pi}{2n})}$

\item On a d'après la question précédente $\frac{1}{\tan(\frac{\pi}{8}}= S_4$
Donc $\tan(\frac{\pi}{8})= \frac{1}{S_4}$.

Par ailleurs $S_4=\ddp \sum_{k=1}^{3} \sin \left( \frac{k\pi}{4}\right) =  \sin \left( \frac{1\pi}{4}\right)+ \sin \left( \frac{2\pi}{4}\right)+ \sin \left( \frac{3\pi}{4}\right) = \frac{\sqrt{2}}{2}+1+\frac{\sqrt{2}}{2}=1+\sqrt{2}$.

Donc 
\begin{center}
\fbox{$ \tan(\frac{\pi}{8}) = \frac{1}{1+\sqrt{2}}$}
\end{center}

\item Montrons que $\ddp\lim_{x\tv 0} \frac{\tan(x)}{x}=1$. 
On a en effet pour tout $x\in ]-\pi/2, \pi/2[$  :
\begin{align*}
\tan'(x) &=\frac{\sin'(x)\cos(x)-\cos'(x)\sin(x)}{\cos^2(x)}\\
			&=\frac{\sin^2(x)+\cos^2(x)}{\cos^2(x)}\\
			&=\frac{1}{\cos^2(x)}			
\end{align*}
En particulier $\tan'(0)=1$ et par définition  de la dérivée en $0$: 
$$\lim_{x\tv 0} \frac{\tan(x)}{x}=\lim_{x\tv 0} \frac{\tan(x)-\tan(0)}{x-0}=\tan'(0)=1$$

On a  $\frac{S_n}{n}= \frac{1}{n \tan(\frac{\pi}{2n})}$, et 
\begin{align*}
n \tan(\frac{\pi}{2n}) &= \frac{ \tan(\frac{\pi}{2n}) }{\frac{1}{n}}\\
								&= \frac{ \frac{\pi}{2}\tan(\frac{\pi}{2n}) }{\frac{\pi}{2n}}
\end{align*}
On vient de voir que $\ddp\lim_{x\tv 0} \frac{\tan(x)}{x}=1$, comme $\ddp \lim_{n\tv \infty} \frac{\pi}{2n}=0$ on a par composé de limites : 
$$\lim_{n\tv \infty} n \tan(\frac{\pi}{2n})  =\frac{\pi}{2}.$$
En conclusion : 
\begin{center}
\fbox{$\ddp \lim_{n\tv \infty} \frac{S_n}{n} = \frac{2}{\pi}\, .$}
\end{center}
\end{enumerate}
\end{correction}




\end{document}
