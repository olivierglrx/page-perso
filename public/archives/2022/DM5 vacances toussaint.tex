\documentclass[a4paper, 11pt,reqno]{article}
\input{/Users/olivierglorieux/Desktop/BCPST/2020:2021/preambule.tex}
\geometry{hmargin=1.5cm,vmargin=2.5cm }
\usepackage{enumitem}
\newif\ifshow
\showfalse
\input{/Users/olivierglorieux/Desktop/BCPST/2021:2022/ifshow.tex}



\author{Olivier Glorieux}


\begin{document}

\title{DM3 \\
}
\begin{exercice}(Fin de l'exercice du DS) 
Dans cet exercice, on considère une suite quelconque de nombres réels $\suite{a}$, et on pose pour tout $n\in \N$:
$$b_n =\sum_{k=0}^n \binom{n}{k} a_k.$$
\begin{center}
\textbf{Partie I : Quelques exemples}
\end{center}
\begin{enumerate}
\item Calculer $b_n$ pour tout $n \in \N$ lorsque la suite $\suite{a}$ est la suite constante égale à $1$.
\item Calculer $b_n$ pour tout $n \in \N$ lorsque la suite $\suite{a}$ est définie par $a_n=\exp(n)$. 

\item 
\begin{enumerate}
\item Démontrer que, pour tout $(n\geq 1,n\geq k\geq 1)$, $$k\binom{n}{k}=n \binom{n-1}{k-1}.$$
\item En déduire que : $\forall n \in \N, \ddp \sum_{k=0}^n \binom{n}{k} k = n2^{n-1}$.
\item Calculer la valeur de $b_n$, pour tout $n \in \N$ lorsque la suite $\suite{a}$ est définie par $a_n=\frac{1}{n+1}$. 
\end{enumerate}
\end{enumerate}
\begin{center}
\textbf{Partie II : Formule d'inversion }
\end{center}
Le  but de cette partie est de montrer que la suite $\suite{a}$ s'exprime en fonction de la suite $\suite{b}$. 
\begin{enumerate}
\item Montrer que pour tout $(k, n , p ) \in \N^3,$ tel que $k\leq p \leq n$ on  a:
$$\binom{n+1}{p}\binom{p}{k}=\binom{n+1}{k}\binom{n+1-k}{p-k}.$$
\item Montrer que, pour tout $(k,n) \in \N^2$, tel que $k\leq n$ on  a :
$$\sum_{i=0}^{n-k} (-1)^i  \binom{n+1-k}{i}=(-1)^{n-k}.$$ 
\item  Montrer que pour tout $n\in \N$ on a $$\sum_{p=0}^{n}\sum_{k=0}^p  \binom{n+1}{k}\binom{n+1-k}{p-k} (-1)^{p-k}  b_k	 = \sum_{k=0}^{n} (-1)^{n-k}  \binom{n+1}{k} b_k $$
\item Donner, pour tout $n\in \N$, l'expression de $a_{n+1}$ en fonction de $b_{n+1}$ et de $a_0, ..., a_n$. 
\item Prouver, par récurrence forte sur $n$ que :
$$\forall n \in \N, a_n =\sum_{k=0}^n (-1)^{n-k} \binom{n}{k} b_k.$$
\item En utilisant le résultat précédent montrer que pour tout $n\in \N$:
$$\sum_{k=0}^{n}   \binom{n}{k}k2^k(-1)^{n-k}=2n.$$ 
\end{enumerate}




\end{exercice}

\begin{correction}
\begin{center}
\textbf{Partie I : Quelques exemples}
\end{center}
\begin{enumerate}
\item Pour $a_n=1$, $b_n=\sum_{k=0}^n \binom{n}{k}  =2^n$.
\item  Pour $a_n=exp(n)$, $b_n=\sum_{k=0}^n \binom{n}{k}e^k  =(1+e^1)^n$.
\item
\begin{enumerate}
\item  $$k\binom{n}{k}= k \frac{n! }{k! (n-k)!} = \frac{n! }{(k-1)! (n-k)!}  = n \frac{(n-1)! }{(k-1)! ((n-1)-(k-1))!}=  n\binom{n-1}{k-1}$$
\item
Comme le premier terme est nul  $\sum_{k=0}^n \binom{n}{k} k = \sum_{k=1}^n n\binom{n-1}{k-1}$
Et d'après la question précédente on a donc $ \sum_{k=0}^n \binom{n}{k} k= n  \sum_{k=1}^n \binom{n-1}{k-1}$
Or en faisant un changement de variable on obtient $\sum_{k=1}^n \binom{n-1}{k-1}= \sum_{k=0}^{n-1} \binom{n-1}{k}$. 
Donc $$ \sum_{k=0}^n \binom{n}{k} k = n 2^{n-1}$$

\item 
D'après la question 3a) on a $ (k+1)\binom{n+1}{k+1} = (n+1)\binom{n}{k}$. Donc 
$$\frac{1}{n+1}\binom{n+1}{k+1} =\frac{1}{k+1}\binom{n}{k}$$

Ainsi $$\sum_{k=0}^n \binom{n}{k} \frac{1}{k+1} = \sum_{k=0}^{n} \frac{1}{n+1}\binom{n+1}{k+1}$$
On fait un changement de variable $k+1=j$ on obtient 
\begin{align*}
\sum_{k=0}^n \binom{n}{k} \frac{1}{k+1} &=  \sum_{j=1}^{n+1} \frac{1}{n+1}\binom{n+1}{j}\\
															&=  \frac{1}{n+1} \left( \sum_{j=0}^{n+1} \binom{n+1}{j} -1\right)\\
															&= \frac{1}{n+1} \left( 2^{n+1}-1\right)
\end{align*}
\end{enumerate}
\end{enumerate}



\begin{center}
\textbf{Partie II : Formule d'inversion }
\end{center}
\begin{enumerate}
\item C'est l'exercice 2 du DM 4.
\item 
$$\sum_{i=0}^{n-k} (-1)^i  \binom{n+1-k}{i}= \sum_{i=0}^{n-k+1} (-1)^i  \binom{n+1-k}{i} -(-1)^{n-k+1} $$ 
Et d'après le BdN : 
$$ \sum_{i=0}^{n-k+1} (-1)^i  \binom{n+1-k}{i} =(1-1)^{n-k} =0$$ 
et 
$$-(-1)^{n-k+1} =(-1)^{n-k}$$
Ce qui donne le résultat. 
\item $b_{n+1} = \ddp \sum_{k=0}^{n+1} \binom{n+1}{k} a_k =\ddp  \sum_{k=0}^{n} \binom{n+1}{k} a_k   +a_{n+1}$. Donc
$$a_{n+1} = b_{n+1}-\sum_{k=0}^{n} \binom{n+1}{k} a_k$$
\item Soit $P(n)$ la propriété : " $\forall p \leq n \, a_p =\ddp \sum_{k=0}^p (-1)^{p-k} \binom{p}{k} b_p.$"\\

Montrons $P(0) : " \forall j \leq 0 \, a_j =\ddp \sum_{k=0}^j (-1)^{j-k} \binom{j}{k} b_k.$" Il suffit  de vérifier $a_0 = \ddp \sum_{k=0}^0 (-1)^{0-k} \binom{0}{k} b_k.$

Et on a  $\ddp \sum_{k=0}^0 (-1)^{0-k} \binom{0}{k} b_k. = b_0$ 
Par ailleurs, par définition $b_0 =  \ddp  \sum_{k=0}^{0} \binom{0}{k} a_k =a_0$. Ainsi $P(0)$ est vraie.  

\underline{Hérédité}


On suppose que $P$ est vraie pour un certain entier naturel $n$ fixé. Montrons $P(n+1)$. 
Pour cela il suffit de vérifier que 
$$a_{n+1} =\sum_{k=0}^{n+1} (-1)^{n+1-k} \binom{n+1}{k} b_k$$
Or on a vu que  
$$a_{n+1}=b_{n+1}-\sum_{p=0}^{n} \binom{n+1}{p}  a_p$$ 
et en utilisant l'hypothése de récurrence on obtient 
\begin{align*}
\sum_{p=0}^{n} \binom{n+1}{p} a_p &=\sum_{p=0}^{n} \binom{n+1}{p} \sum_{k=0}^p (-1)^{p-k} \binom{p}{k} b_k\\
												&=\sum_{p=0}^{n}\sum_{k=0}^p  \binom{n+1}{p} \binom{p}{k} (-1)^{p-k}  b_k\\
												&=\sum_{p=0}^{n}\sum_{k=0}^p  \binom{n+1}{k}\binom{n+1-k}{p-k} (-1)^{p-k}  b_k												
\end{align*}
D'après la question  II . 1. 


On échange les deux symboles sommes on obtient : 
\begin{align*}
\sum_{p=0}^{n}\sum_{k=0}^p  \binom{n+1}{k}\binom{n+1-k}{p-k} (-1)^{p-k}  b_k		&=
\sum_{k=0}^{n}\sum_{p=k}^n  \binom{n+1}{k}\binom{n+1-k}{p-k} (-1)^{p-k}  b_k	\\
&=\sum_{k=0}^{n}   \binom{n+1}{k} b_k \sum_{p=k}^n \binom{n+1-k}{p-k} (-1)^{p-k}  	\\
&= \sum_{k=0}^{n}   \binom{n+1}{k} b_k \sum_{i=0}^{n-k} \binom{n+1-k}{i} (-1)^{i}  	
\end{align*}
En faisant le cahngement d'indice $p-k=i$. 

On obtient finalement  en utilisant la question  II. 2.
\begin{align*}
\sum_{p=0}^{n} \binom{n+1}{p} a_p &=\sum_{k=0}^{n}   \binom{n+1}{k} b_k (-1)^{n-k}
\end{align*}
On conclut en remarquant que $b_{n+1} 	= 	 \binom{n+1}{n+1} b_{n+1} (-1)^{n+1-(n+1)}$
et ainsi 
$$a_{n+1}= 	 (-1)^{n+1-(n+1)}  \binom{n+1}{n+1} b_{n+1} + \sum_{k=0}^{n}  (-1)^{n+1-k} \binom{n+1}{k} b_k = \sum_{k=0}^{n+1}  (-1)^{n+1-k} \binom{n+1}{k} b_k $$
									
											 

\item On a vu dans la partie I que pour $a_n=n$ on a
$b_n=n2^{n-1}$. Donc en appliquant le résultat précédent on a 
$$ \sum_{k=0}^{n}   \binom{n}{k}k2^{k-1}(-1)^{n-k} = n$$ 
Ce qui donne finalement 
$$ \sum_{k=0}^{n}   \binom{n}{k}k2^k(-1)^{n-k} =  2 \sum_{k=0}^{n}   \binom{n}{k}k2^{k-1}(-1)^{n-k}= 2 n$$
\end{enumerate}


\end{correction}




\begin{exercice}
Soit $\suite{u}$ la suite définie par 
$$\left\{ 
\begin{array}{ccl}
u_0&=&1\\
u_{n+1} &=& \sin(u_n)
\end{array}
\right.$$

\begin{enumerate}
\item Montrer que pour tout $n\in \N$, $0<u_n<\frac{\pi}{2}$.
\item On note $f(x) = \sin(x)-x$. Montrer que pour tout $x\in \R_+^*$, $f(x)<0.$
\item En déduire le sens de variation de $\suite{u}$.
%\item Montrer que $\suite{u}$ est bornée.
\item En déduire que $\suite{u}$ converge vers $\ell \in \R$
\item  Montrer que $f(x)=0 \equivaut x=0$.
\item Déterminer la valeur de $\ell$. 
\end{enumerate}

Info 
\begin{enumerate}
\item Ecrire une fonction qui prend en paramètre $n\in \N$ et qui retourne la valeur de $u_n$. (Pour ceux qui n'ont pas encore vu les fonctions, vous pouvez écrire un script qui retourne la valeur de $u_n$ sans les fonctions, mais bon c'est pas si différent... ) 
\item 
Ecrire une fonction qui prend en paramètre $e\in \R^+$ et qui retourne la valeur du premier terme $n_0\in \N$ telle que $|u_{n_0}| \leq e$ et la valeur de $u_{n_0}$. (même remarque) 
\end{enumerate}

\end{exercice}

\begin{correction}
\begin{enumerate}
\item On fait une récurrence. 
Pour tout $n\in \N$ on note  $P(n)$ la propriété définie par:  $" 0<u_n<\frac{\pi}{2}"$
Par définition $u_0= 1$, et on a bien $0<1<\frac{\pi}{2}$ (car $\pi>3$) 
Donc la propriété $P$ est vraie au rang $0$.  

 On suppose qu'il existe $n_0\in \N$ tel que $P_{n_0}$ soit vraie et on  va montrer que ceci implique $P_{n_0+1}$ 

En effet, pour tout $x\in ]0,\frac{\pi}{2}[,$ $\sin(x) \in ]0,1[\subset ]0,\pi/2[$ \footnote{en d'autres termes, $]0,\pi/2[ $ est stable par la fonction sinus}. Donc 
si $P_{n_0}$ est vraie, c'est à dire  $u_{n_0} \in  ]0,\frac{\pi}{2}[ $, on a  alors $u_{n_0+1}=\sin(u_{n_0}) \in ]0,1[$.
De nouveau comme $1< \frac{\pi}{2}$ ceci implique $P_{n_0+1}$. 

Par récurrence, la propriété $P(n)$ est vraie pour tout $n\in \N$. 


\item  La fonction $f$ est dérivable sur $\R$ et $f'(x) =\cos(x)-1\leq 0$. 
Donc $f$ est décroissante et $f(0)=0$. Donc pour tout $x\in \R_+^*,$ $f(x)< 0$.
\item $u_{n+1}-u_n =\sin(u_n)-u_n=f(u_n)$
Comme pour tout $n\in \N$, $u_n>0$ d'après la question 1, on a donc 
$f(u_n) <0$ d'après la question 2. Ainsi pour tout $n\in \N$
$$u_{n+1} \leq u_n$$ ce qui assure que la suite $\suite{u}$ est décroissante. 

\item La suite $\suite{u}$ est minorée (par 0) d'après la question $1$ et décroissante d'après la question précédente. Par théorème de la limite monotone, la suite converge vers $\ell \geq 0$ 

\item L'étude de $f$ a montré que $f(x)<0$ sur $\R^*_+$  et $f(x)>0$ sur $\R^*_-$. Ainsi $f(x)=0 \implique x=0$. Réciproquement, si $x=0$ , $f(0) =\sin(0)-0=0$. L'équivalence est bien montrée. 

\item Comme $\suite{u}$ converge vers $\ell\in \R$ on a aussi 
$\lim u_{n+1} =\ell$. De plus, comme la fonction sinus est continue sur $\R$ on a $\lim \sin(u_n) = \sin(\lim u_n) $. Ainsi la limite $\ell$ satisfait  $\ell =\sin(\ell)$. Ce qui d'après la question précédente implique $\ell=0$. 

Finalement $$\lim u_n= 0$$

\end{enumerate}

INFO


\begin{lstlisting}[language =Python]
from math import sin
def u(n):
  x=1	#valeur de u0
  for i in range(n):
     x=sin(x) 	#relation de recurrence que l'on applique n fois avec range(n)
  return(x)

from math import abs
def limite(e):
   L=0 #valeur de la limite
   n=0  #on met en place un compteur
   val=u(n)  #valeur de u0 
 
   while abs(val-L)>e: #tant que la valeur de |u(n)-L| est plus grande que e
      n+=1 #on incremente la valeur du compteur de 1 
      val =u(n) #on actualise la valeur de u(n)
      
   return(n, u(n))
\end{lstlisting}



\end{correction}





\end{document}
