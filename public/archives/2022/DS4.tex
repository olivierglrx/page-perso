\documentclass[a4paper, 11pt,reqno]{article}
\input{/Users/olivierglorieux/Desktop/BCPST/2020:2021/preambule.tex}
\usepackage{enumitem}
\geometry{hmargin=2.0cm, vmargin=2.5cm}
\lstset{basicstyle=\ttfamily, keywordstyle=\rmfamily\bfseries}
\newif\ifshow
\showtrue
\input{/Users/olivierglorieux/Desktop/BCPST/2021:2022/ifshow.tex}

\author{Olivier Glorieux}


\begin{document}

\title{Correction - DS 4}




\begin{exercice}
\begin{enumerate}
\item Résoudre l'inéquation :  $$ (E_1)   \quad : \quad 2x-1\leq \frac{1}{2x+1}$$
\item En déduire les solutions de $(E_2)$ sur $[0, 2\pi[$
$$(E_2)   \quad : \quad   2\cos(X)-1\leq \frac{1}{2\cos(X)+1}$$

\end{enumerate}

\end{exercice}
\begin{correction}
\begin{enumerate}
\item \begin{align*}
(E_1)\equivaut 2x-1 -\frac{1}{2x+1}\leq 0\\
		\equivaut  \frac{4x^2-1-1}{2x+1}\leq 0\\
		\equivaut  \frac{4x^2-2}{2x+1}\leq 0\\
		\equivaut  \frac{4(x+\frac{1}{\sqrt{2}})(x-\frac{1}{\sqrt{2}}) }{2x+1}\leq 0\\
\end{align*}
On en déduit (après avoir fait un tableau de signe si besoin) les solutions de $(E_1) $
\conclusion{ $S_1 = \left] -\infty, \frac{-1}{\sqrt{2}}\right]\cup \left]\frac{-1}{2} , \frac{1}{\sqrt{2}}\right]$}
\item $X$ est solution de $(E_2)$ si et seulement si $\cos(X) $ est solution de $(E_1)$ c'est à dire si et seulement si 
$$\cos(X) \in  
\left] -\infty, \frac{-1}{\sqrt{2}}\right]\cup \left]\frac{-1}{2} , \frac{1}{\sqrt{2}}\right]$$

Comme pour tout $X\in \R$, $\cos(X)\geq-1$ on obtient 
$$\cos(X) \in  
\left[ -1 ,\frac{-1}{\sqrt{2}}\right]\cup \left]\frac{-1}{2} , \frac{1}{\sqrt{2}}\right]$$

A l'aide du cercle trigonométrique, on trouve les solutions sur $[0,2\pi[:$
$$X\in \left[\frac{\pi}{4}, \frac{2\pi}{3}  \right[ \cup   \left[\frac{3\pi}{4}, \frac{5\pi}{4}\right]  \cup   \left] \frac{4\pi}{3}, \frac{7\pi}{4}\right]$$


\end{enumerate}

\end{correction}
\vspace{0.6cm}
\begin{exercice}
Une urne contient 3 boules jaunes, 2 boules vertes et 5 boules rouges. Les boules sont toutes distingables, numérotées par exemple.  On tire successivement et avec remise 4 boules.
\begin{enumerate}
\item Combien y-a-t-il de tirages possibles ? 
\item Combien de tirages amènent aucune boule rouge ?
\item Combien de tirages amènent que des boules vertes ?
\item Combien de tirages amènent exactement 2 boules jaunes ?
\item Combien de tirages amènent des boules d'une seule couleur ?
\item (INFO) On modélise une boule par une lettre 'J' pour jaune 'V' pour verte et 'R' pour rouge. On modélise l'urne par une liste \texttt{U} dont chaque élément element est une boule. 
\begin{enumerate}
\item Créer la liste \texttt{U}
\item A l'aide de la fonction \texttt{randint(a,b)} qui choisit un nombre entier aléatoirement entre a et b (inclus), écrire une fonction \texttt{tirage} qui retourne une liste correspondant au 4 tirages successifs dans l'urne. 
\end{enumerate}
\end{enumerate}

\end{exercice}

\begin{correction}


\begin{enumerate}
\item On fait $4 (=p)$ tirages successifs (ordre) avec remise dans un ensemble à 10 éléments ($n=10$)
\conclusion{ Il  y  a $10^4$ tirages possibles}
\item Pour obtenir aucune boule rouge il faut tirer des boules vertes ou jaunes, il y en  a 5. On  a donc 
\conclusion{ Il  y  a $5^4$ tirages possibles sans boule rouge}
\item Il y a 2 boules vertes donc 
\conclusion{ Il  y  a $2^4$ tirages possibles avec que des boules vertes}

\item Il faut tirer 2 boules jaunes (3 possibilités) et 2 boules parmi les vertes ou rouges (7 possiblités). Ensuite il faut positionner les boules jaunes parmi les 4 tirages, cela fait $\binom{4}{2}$ positions possibles. 
\conclusion{ Il  y  a $\binom{4}{2}3^2 7^2$ tirages possibles exactement 2 boules jaunes}
\item Pour obtenir qu'une seule couleur on a 3 façons de faire : que des vertes $V$ , que des jaunes $J$ ou que des rouges $R$. 
On a déjà calcule le cardinal de $V$ à la question 3. On fait de même avec $J$ on obtient $\Card(J) =3^4$ et $\Card(R)=5^4$. Finalement 
\conclusion{ Il y a $2^4+3^4+5^4$ tirages qui amènent qu'une seule couleur}
\begin{lstlisting}
U=['J']*3+['V']*2+['R']*5
import random as rd
def tirage(n,U):
  ''' n correspond au nombre de tirages effectue
  U est une liste qui modelise l'urne. 
  pour modeliser l experience decrite on appliquera la fonction tirage(4,U)
    '''
   L=[]
   for i in range(n):
     r=rd.randint(0,len(U)-1) 
       #attention len(U)-1 est compris dans le tirage aleatoire effectue par randint
     L=L.append(U[r])
   return(L)
\end{lstlisting}


\end{enumerate}
\end{correction}

\vspace{1cm}
\begin{exercice}
On considère les mains  de $5$ cartes (tirages simultanés de 5 cartes) que l'on  peut obtenir d'un jeu de $52$ cartes. 
\begin{enumerate}
\item Combien y-a-t-il de mains différentes ? 
\item  Combien y-a-t-il de mains  comprenant exactement deux as ?
\item  Combien y-a-t-il de mains  comprenant au moins un coeur ?
\item  Combien y-a-t-il de mains  comprenant exactement un roi et un coeur  ?
\end{enumerate}
\end{exercice}





\begin{correction}
\begin{enumerate}
\item C'est un tirage sans ordre et sans répétition. 
\conclusion{ Il y a $\binom{52}{5}$ mains possibles.}
\item Pour dénombrer les mains avec exactement 2 as, il faut dénombrer le choix de deux as parmi les 4 possibles : $\binom{4}{2}$ puis les 3 cartes parmi les 48 autres possibles : $\binom{48}{3}$. Au final il y  a 
\conclusion{ $\binom{4}{2}\binom{48}{3}$ avec exactement 2 as.}
\item On cherche l'événement contraire : les mains comprenant 0 coeur. Il y en a 
$\binom{52-13}{5}$ Ainsi il y  a
\conclusion{ $\binom{52}{5}- \binom{39}{5}$ mains contenant au moins un coeur.}
\item Soit $E=\{ \text{mains contenant  exactement un roi et un coeur }\}$. On chercher à déterminer le cardinal de $E$. 

Il faut regarder à part les mains contenant le roi de coeur. 
Soit $$A_1= \{ \text{mains contenant le roi de coeur et pas d'autre coeur ni de roi}\}$$
$$A_2= \{ \text{mains contenant le roi -qui n'est pas de coeur - et un coeur -qui n'est pas le roi.}\}$$
On a d'une part 
$$\Card(A_1) =  \binom{1}{1} \binom{52-16}{5} $$
et d'autre part
$$\Card(A_2) =  \binom{3}{1} \binom{12}{1} \binom{52-16}{3}$$
Par ailleurs, on a $E=A_1\cup A_2$ et $A_1\cap A_2=\emptyset$
Donc $\Card(E) = \Card(A_1) +\Card(A_2)$ et 
\conclusion{ $\Card(E)=\binom{36}{5}+ 3\times 12\binom{36}{3}$}




\end{enumerate}
\end{correction}





%%%%

\vspace{0.6cm}

\begin{exercice}
Soit $\suite{u}$ la suite définie par 
$$\left\{ 
\begin{array}{ccl}
u_0&=&1\\
u_{n+1} &=& \sin(u_n)
\end{array}
\right.$$

\begin{enumerate}
\item Montrer que pour tout $n\in \N$, $0<u_n<\frac{\pi}{2}$.
\item  A l'aide d'une étude de fonction, montrer que pour tout $x\in \R_+^*$, $$\sin(x)<x.$$
\item En déduire le sens de variation de $\suite{u}$.
%\item Montrer que $\suite{u}$ est bornée.
\item En déduire que $\suite{u}$ converge vers un réel $\ell\in [0,\frac{\pi}{2}]$
\item  Montrer que $f(x)=0 \equivaut x=0$.
\item Déterminer la valeur de $\ell$. 
\end{enumerate}

Info 
\begin{enumerate}
\item Ecrire une fonction qui prend en paramètre $n\in \N$ et qui retourne la valeur de $u_n$. 
\item 
Ecrire une fonction qui prend en paramètre $e\in \R^+$ et qui retourne la valeur du premier terme $n_0\in \N$ telle que $|u_{n_0}-\ell| \leq e$ et la valeur de $u_{n_0}$.
\end{enumerate}

\end{exercice}

\begin{correction}
\begin{enumerate}
\item On fait une récurrence. 
Pour tout $n\in \N$ on note  $P(n)$ la propriété définie par:  $" 0<u_n<\frac{\pi}{2}"$
Par définition $u_0= 1$, et on a bien $0<1<\frac{\pi}{2}$ (car $\pi>3$) 
Donc la propriété $P$ est vraie au rang $0$.  

 On suppose qu'il existe $n_0\in \N$ tel que $P_{n_0}$ soit vraie et on  va montrer que ceci implique $P_{n_0+1}$ 

En effet, pour tout $x\in ]0,\frac{\pi}{2}[,$ $\sin(x) \in ]0,1[\subset ]0,\pi/2[$ \footnote{en d'autres termes, $]0,\pi/2[ $ est stable par la fonction sinus}. Donc 
si $P_{n_0}$ est vraie, c'est à dire  $u_{n_0} \in  ]0,\frac{\pi}{2}[ $, on a  alors $u_{n_0+1}=\sin(u_{n_0}) \in ]0,1[$.
De nouveau comme $1< \frac{\pi}{2}$ ceci implique $P_{n_0+1}$. 

Par récurrence, la propriété $P(n)$ est vraie pour tout $n\in \N$. 


\item  La fonction $f$ est dérivable sur $\R$ et $f'(x) =\cos(x)-1\leq 0$. 
Donc $f$ est décroissante et $f(0)=0$. Donc pour tout $x\in \R_+^*,$ $f(x)< 0$.
\item $u_{n+1}-u_n =\sin(u_n)-u_n=f(u_n)$
Comme pour tout $n\in \N$, $u_n>0$ d'après la question 1, on a donc 
$f(u_n) <0$ d'après la question 2. Ainsi pour tout $n\in \N$
$$u_{n+1} \leq u_n$$ ce qui assure que la suite $\suite{u}$ est décroissante. 

\item La suite $\suite{u}$ est minorée (par 0) d'après la question $1$ et décroissante d'après la question précédente. Par théorème de la limite monotone, la suite converge vers $\ell \geq 0$ 

\item L'étude de $f$ a montré que $f(x)<0$ sur $\R^*_+$  et $f(x)>0$ sur $\R^*_-$. Ainsi $f(x)=0 \implique x=0$. Réciproquement, si $x=0$ , $f(0) =\sin(0)-0=0$. L'équivalence est bien montrée. 

\item Comme $\suite{u}$ converge vers $\ell\in \R$ on a aussi 
$\lim u_{n+1} =\ell$. De plus, comme la fonction sinus est continue sur $\R$ on a $\lim \sin(u_n) = \sin(\lim u_n) $. Ainsi la limite $\ell$ satisfait  $\ell =\sin(\ell)$. Ce qui d'après la question précédente implique $\ell=0$. 

Finalement $$\lim u_n= 0$$

\end{enumerate}

INFO


\begin{lstlisting}[language =Python]
from math import sin
def u(n):
  x=1	#valeur de u0
  for i in range(n):
     x=sin(x) 	#relation de recurrence que l'on applique n fois avec range(n)
  return(x)

from math import abs
def limite(e):
   L=0 #valeur de la limite
   n=0  #on met en place un compteur
   val=u(n)  #valeur de u0 
 
   while abs(val-L)>e: #tant que la valeur de |u(n)-L| est plus grande que e
      n+=1 #on incremente la valeur du compteur de 1 
      val =u(n) #on actualise la valeur de u(n)
      
   return(n, u(n))
\end{lstlisting}



\end{correction}











\newpage
\begin{exercice}


Le but de cet exercice est de calculer la valeur de 
$$\lim_{n\tv \infty} \sum_{k=0}^n \frac{1}{k!}$$

\paragraph{Convergence}
On note $S_n=\ddp   \sum_{k=0}^n \frac{1}{k!}$ et $R_n=\ddp \sum_{k=0}^n \frac{1}{k!} +\frac{1}{n(n!)}$
\begin{enumerate}
\item Donner la monotonie de $\suite{S}$ et de $\suiteun{R}$
\item En déduire que les suites $\suite{S}$ et $\suiteun{R}$ convergent et ont même limite. 
\end{enumerate}

\paragraph{Informatique}
\begin{enumerate}
\item Ecrire une fonction \texttt{factorielle} qui prend en argument un entier $n$ et retourne la valeur de $n!$
\item Ecrire deux fonctions \texttt{S} et \texttt{R} qui prennent en argument un entier $n$ et retourne respectivement la valeur de $S_n$ et $R_n$. 
\item Ecrire une fonction \texttt{limite} qui prend en argument un réel positif $\epsilon$ et  retourne la valeur de $S_n$ pour laquelle $|S_n -R_n|\leq \epsilon$ (la premiere valeur pour laquelle cette condition est satisfaite).
\end{enumerate}


\paragraph{Calcul de la limite}
Pour tout $n\in \N$ on définit la fonction $f_n$ par 
$$f_n(x)  = \ddp \sum_{k=0}^n \frac{x^k}{k!} \quadet g_n(x) =f_n(x)e^{-x}$$
On rappelle que par convention $\forall x \in \R, \, x^0 =1$, et $0!=1$
\begin{enumerate}
\item Exprimer $g_1(x)$ sans le signe somme. 
\item Calculer $g_n(0)$ et exprimer $g_n(1)$ à l'aide de $S_n$. 
\item Montrer que pour tout $x\in \R$ et tout $n\in \N$: $$f'_n(x) = \sum_{k=0}^{n-1} \frac{x^k}{k!}$$
\item En déduire que  pour tout $x\in \R$ et tout $n\in \N$ $g'_n(x)= \frac{-x^ne^{-x}}{n!}$
\item  \begin{enumerate}
\item Exprimer en fonction de $n\in \N$ la valeur de  $\int_0^1 \frac{-e^{-x}}{n!}dx$
\item A l'aide d'un encadrement de $g_n'(x)$, montrer que pour tout $x\in \R$ et tout $n\in \N$ $$ \frac{e^{-1}-1}{n!}\leq \int_0^1 g'_n(x) dx\leq 0$$
\end{enumerate}


\item En déduire que pour tout $n\in \N$:  $$ \frac{e^{-1}-1}{n!} \leq S_ne^{-1}-1\leq 0$$
\item En déduire la limite de $\suite{S}$. 
\end{enumerate}

\end{exercice}
\begin{correction}
\paragraph{Convergence}
\begin{enumerate}
\item $\forall n\in \N$ 
$$S_{n+1}-S_n = \sum_{k=0}^{n+1} \frac{1}{k!}- \sum_{k=0}^n \frac{1}{k!}=\frac{1}{(n+1)!}\geq 0$$
Donc \conclusion{ $\suite{S}$ est croissante}

 $\forall n\in \N^*$ 
 \begin{align*}
 R_{n+1}-R_n &= S_{n+1}-S_n +\frac{1}{(n+1) ((n+1)!)}-\frac{1}{n (n!)}\\
 					&=\frac{1}{(n+1)!} +\frac{n-(n+1)^2}{n(n+1)((n+1)!)}\\
 					&=\frac{n(n+1)+n-(n+1)^2}{n(n+1)((n+1)!)}\\
 					&=\frac{n^2+n+n-(n^2+2n+1)}{n(n+1)((n+1)!)}\\
 					&=\frac{-1}{n(n+1)((n+1)!)}
 \end{align*}
donc $R_{n+1}-R_n<0$, ainsi

\conclusion{ $\suiteun{R}$ est décroissante}
\item $\forall n \in \N^*,  R_n-S_n=\frac{1}{n (n!)}$ donc 
$$\lim_{n\tv +\infty} R_n-S_n= 0$$
De plus $\suite{S}$ est croissante, $\suiteun{R}$ est décroissante. Donc les deux suites sont adjacentes. 

\conclusion{Le théorème sur les suites adjacentes assure que les suites convergent et ont même limite. }

\end{enumerate}
\paragraph{Informatique}
\begin{enumerate}
\item \begin{lstlisting}
def factorielle(n):
  p=1
  for i in range(1,n+1):
    p=p*i
  return(p)
\end{lstlisting}
\item \begin{lstlisting}
def S(n):
  s=0
  for k in range(n+1):
    s=s+1/factorielle(k)
  return(s)
  
def R(n):
  return(S(n)+1/(n*factorielle(n))

\end{lstlisting}
\item \begin{lstlisting}
def limite(epsilon):
  n=1
  while n*factorielle(n) >1/epsilon:
    n=n+1
  return(S(n)) 
\end{lstlisting}
\end{enumerate}

\paragraph{Calcul de la limite}
\begin{enumerate}
\item On a pour tout $x\in \R$ : $g_1(x) =f_1(x)e^{-x} = \ddp \sum_{k=0}^1 \frac{x^k}{k!} e^{-x}= (1+x)e^{-x}$
\conclusion{ $g_1(x) =(1+x)e^{-x}$}
\item $g_n(0)= f_n(0)e^{-0}=\sum_{k=0}^n \frac{0^k}{k!}=1 $
\conclusion{ $g_n(0)=1$}

$g_n(1) = f_n(1)e^{-1}= \sum_{k=0}^n \frac{1}{k!}e^{-1}=S_ne^{-1}$
\conclusion{ $g_n(1)=S_n e^{-1}$}

\item La dérivée d'une somme de fonction est égale à la somme des dérivées des fonctions et $u_k:x\mapsto \frac{x^k}{k!}$ se dérive en 
$$u'_k(x) = \frac{kx^{k-1}}{k!} = \frac{x^{k-1}}{(k-1)!}$$
si $k\geq 1$ et $u'_0(x)=0$

Ainsi $f_n'(x) = \sum_{k=0}^n u'_k(x)=  u'_0(x)+\sum_{k=1}^n \frac{x^{k-1}}{(k-1)!}$
On fait ensuite le changement de variable $i=k-1$ et  on obtient 
\conclusion{$f_n'(x)= \sum_{i=0}^{n-1} \frac{x^{i}}{(i)!}$}

\item Pour tout $x\in \R$ et tout $n\in \N$ on a 
\begin{align*}
g'_n(x) &= f'_n(x)e^{-x} -f_n(x)e^{-x}\\
			&= (f'_n(x)-f_n(x))e^{-x}
\end{align*}
D'après la question précédente : $f'_n(x)-f_n(x)= -\frac{x^{n}}{n!}$. On obtient donc 

\conclusion{ $g'_n(x)= -\frac{x^{n}e^{-x}}{n!}$}

\item 
\begin{enumerate}
\item  $\int_0^1 \frac{-e^{-x}}{n!}dx= \frac{-1}{n!} \int_0^1 e^{-x}= \frac{-1}{n!} [-e^{-x}]_0^1 = \frac{-1}{n!} (-e^{-1}+1)= \frac{e^{-1}-1}{n!}$
\item 
D'après la question précédente on a pour tout $x\in [0,1]$ et tout $n\in \N $ : $x^n\geq 0$ et $e^{-x}\geq 0$ donc 
$$g'_n(x) \leq 0$$
Par positivité de l'intégrale, on a alors 
\conclusion{$\ddp \int_0^1 g'_n(x) dx \leq 0.$}

De même, $x^n\leq 1 $ donc 
$$g'_n(x) \geq \frac{-e^{-x}}{n!}
$$

Par positivité de l'intégrale, on a alors 
$$\int_0^1 g'_n(x) dx \geq \int_0^1 \frac{-e^{-x}}{n!}dx.$$
Donc 
\conclusion{  $\ddp \frac{e^{-1}-1}{n!}\leq  \int_0^1g'_n(x)dx$}
\end{enumerate}
\item On a pour tout $n\in \N$
$$\int_0^1g_n'(x) dx = [g_n(x)]_0^1 =g_n(1)-g_n(0) = S_ne^{-1} -1$$
Donc d'après la question précédente 
\conclusion{$\frac{e^{-1}-1}{n!}\leq S_ne^{-1} -1\leq 0$}

\item On  a d'après la question précédente
$$\frac{e^{-1}-1}{n!}\leq S_ne^{-1} -1\leq 0$$
donc en isolant $S_n$ au milieu des deux inégalités on obtient :  
$$e(\frac{e^{-1}-1}{n!}+1)\leq S_n\leq e$$



Or 
$\lim_{n\tv +\infty} \frac{e^{-1}-1}{n!}=0$. Donc
$$\lim_{n\tv +\infty} e(\frac{e^{-1}-1}{n!}+1)=e$$
Ainsi d'après le théorème d'encadrement la suite $\suite{S}$ converge et 
\conclusion{ $\lim_{n\tv+\infty } S_n=e$ }



\end{enumerate}

\end{correction}




\end{document}