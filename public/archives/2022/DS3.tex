\documentclass[a4paper, 11pt,reqno]{article}
\input{/Users/olivierglorieux/Desktop/BCPST/2020:2021/preambule.tex}
\usepackage{enumitem}
\geometry{hmargin=2.0cm, vmargin=2.5cm}
\lstset{basicstyle=\ttfamily, keywordstyle=\rmfamily\bfseries}
\newif\ifshow
\showfalse
\input{/Users/olivierglorieux/Desktop/BCPST/2021:2022/ifshow.tex}

\author{Olivier Glorieux}


\begin{document}

\title{ DS 3
}



\begin{exercice}
On s'intéresse dans cet exercice aux sommes 
$$C_n(x)= \sum_{k=0}^n \cos(kx)\quadet S_n(x)= \sum_{k=0}^n \sin(kx)$$
\begin{enumerate}
\item Soit $S$ l'ensemble des $x\in \R$ tel que $e^{ix}=1$. Déterminer $S$.
\item Déterminer $C_n(x)$ en fonction de $n\in \N$ pour $x\in S$. 

\item Rappeler la valeur de $\ddp \sum_{k=0}^n q^k $ en fonction  de  $n\in \N$ pour $q\neq 1$ et  en déduire la valeur de  $\ddp \sum_{k=0}^n e^{ikx} $  en fonction de $n$ pour $x\notin S$. 

\item On considère $Z_n(x)=C_n(x) +iS_n(x)$. Montrer pour $x\notin S$ :
$$Z_n(x) = e^{i\frac{nx}{2}} \frac{\sin\left(\frac{(n+1)x}{2}\right)}{\sin\left(\frac{x}{2}\right)}$$
\item En déduire la valeur de $C_n(x)$ en fonction de $x$, pour $x\notin S$. 
\end{enumerate}
\end{exercice}
\begin{correction}
\begin{enumerate}
\item $e^{ix}=1\equivaut \cos(x)+i\sin(x)=1 \equivaut 
\left\{ 
\begin{array}{cc}
\cos(x)&=1\\
\sin(x)&=0
\end{array}
\right.
\equivaut x\in \{ 2k \pi\, |\, k\in \Z\}$
\conclusion{ $S=\{ 2k \pi\, |\, k\in \Z\}$}

\item Si $x\in S$ alors $x=2m\pi $ avec $m\in \Z$ donc 
pour tout $k\in \N$, $\cos(kx)= \cos(2\pi k  m) =1$. Ainsi 
$$C_n(x) = \sum_{k=0}^n 1= (n+1)$$

\conclusion{ $\forall n\in \N, \, \forall x\in S$, $C_n(x)=n+1$}
\item 
Pour tout $q\neq 1$:
\conclusion{$\sum_{k=0}^n q^k= \frac{1-q^{n+1}}{1-q}$}
Or
$$\sum_{k=0}^n e^{ikx} =\sum_{k=0}^n (e^{ix}) ^k$$
Donc si $x\notin S$, $e^{ix}\neq 1$ et 

\conclusion{ $\sum_{k=0}^n e^{ikx} = \frac{1-e^{i(n+1)x}}{1-e^{ix}}$}

\item 
En utilisant la linéarité de la somme et la question précédente on obtient pour $x\notin S$ : 
\begin{align*}
Z_n(x)&= \sum_{k=0}^n \cos(kx)+i\sum_{k=0}^n \sin(kx)\\
	 &=  \sum_{k=0}^n \cos(kx)+i \sin(kx)\\
	 &=  \sum_{k=0}^n e^{ikx}\\
	 &=  \sum_{k=0}^n  \frac{1-e^{i(n+1)x}}{1-e^{ix}}
\end{align*}

On va transformer l'expression pour faire apparaitre les sinus : 

\begin{align*}
 \frac{1-e^{i(n+1)x}}{1-e^{ix}}  &=  \frac{ e^{i\frac{n+1}{2}x} \left(    e^{-i\frac{n+1}{2}x}-e^{i\frac{n+1}{2}x}\right)}{e^{i\frac{1}{2}x}\left(e^{-i\frac{1}{2}x}-e^{i\frac{1}{2}x}\right) }  
\end{align*}

et par ailleurs on a $  e^{-i\frac{n+1}{2}x}-e^{i\frac{n+1}{2}x}=-2i \sin \left( \frac{n+1}{2}x\right) $ et $  e^{-i\frac{1}{2}x}-e^{i\frac{1}{2}x}=-2i \sin \left( \frac{1}{2}x\right) $.

On obtient finalement 
\begin{align*}
 \frac{1-e^{i(n+1)x}}{1-e^{ix}} &=e^{i\frac{n+1}{2}x- i \frac{1}{2}x}  \frac{-2i \sin \left( \frac{n+1}{2}x\right)}{-2i \sin \left( \frac{1}{2}x\right)}\\
 &= e^{i\frac{n}{2}x}  \frac{\sin \left( \frac{n+1}{2}x\right)}{\sin \left( \frac{1}{2}x\right)}\\
\end{align*}

On trouve bien : 
\conclusion{ $\forall n\in \N, \forall x\notin S , Z_n(x)=  e^{i\frac{n}{2}x}  \frac{\sin \left( \frac{n+1}{2}x\right)}{\sin \left( \frac{1}{2}x\right)}$}

\item Par définition de $Z_n$, on a $C_n(x) =\Re(Z_n(x))$
Donc 
$$C_n(x)= \Re\left(
e^{i\frac{n}{2}x}  \frac{\sin \left( \frac{n+1}{2}x\right)}{\sin \left( \frac{1}{2}x\right)}\right)$$

\conclusion{$C_n(x)= 
\cos\left(\frac{nx}{2}\right) \frac{\sin \left( \frac{n+1}{2}x\right)}{\sin \left( \frac{1}{2}x\right)}$}

\end{enumerate}
\end{correction}



\begin{exercice}
On s'intéresse dans cet exercice à la  suite $\suite{u}$ qui vérifie les relations suivantes : 
$$(R)\, :\, 
u_{n+1} =2u_n+n^2\quadet (CI) u_0=1$$
 On pose : $ \forall n\in \N, v_n =an^2+bn+c$ où $(a,b,c)$ sont trois réels.  
\begin{enumerate}
\item Déterminer les triplets $(a,b,c)\in \R^3$ tel que :
$$\forall n\in \N,\,v_{n+1} =2v_n+n^2$$
\item Soit $\suite{v}$ une des suites précédentes et $\suite{x}$ la suite définie par  $x_n=u_n-v_n$. Montrer que $\suite{x}$ est géométrique. 
\item En déduire l'expression de $x_n$ en fonction de $n$ puis de $u_n$. 
\end{enumerate}
\end{exercice}

\begin{correction}
\begin{enumerate}
\item Si $\suite{v}$ vérifie la relation $(R)$ on a alors 
$$a(n+1)^2 +b(n+1) +c = 2(an^2+bn+c) +n^2$$
Et donc 
$$a(n^2+2n+1) +bn+b+c = 2an^2 +2bn+2c +n^2$$
En regroupant les différents termes on obtient 
$$a n^2 +(2a+b)n +a+b+c = (2a+1)n^2 +2bn +2c$$
et en identifiant on a : 

$$\left\{ \begin{array}{ll}
a&=2a+1\\
2a+b&=2b\\
a+b+c&=2c
\end{array}\right.\equivaut 
\left\{ \begin{array}{ll}
a&=-1\\
b&=-2\\
c&=-3
\end{array}\right.
$$
\conclusion{ $v_n= -n^2-2n-3$}
\item  Soit $x_n= u_n-v_n$, comme $\suite{u} $ et $\suite{v}$ vérifient la relation $(R)$ on obtient pour $\suite{x}$:

\begin{align*}
x_{n+1}  &= u_{n+1}-v_{n+1}\\
			  &= 2u_n+n^2-2v_n+n^2\\
			  			  &= 2u_n-2v_n\\
			  			  &= 2x_n
\end{align*}

\conclusion{Ainsi $\suite{x}$ est géométrique de raison $2$. }

\item On a donc pour tout $n\in \N$  : $x_n =x_02^n$ 
et $x_0 = u_0-v_0 =1-(-3)= 4$ donc 
\conclusion{ $x_n = 4\times 2^n$}
or $u_n=x_n+v_n$ 
et donc
\conclusion{ $u_n = 4\times 2^n -n^2-2n-3$}


\end{enumerate}

\end{correction}
\vspace{1cm}

\begin{exercice}
\begin{enumerate}
\item Donner en fonction du paramétre $\lambda\in \R$ le rang du système:
$$(S_\lambda) \,:\,\left\{\begin{array}{rl}
8x +5y&=\lambda x\\
-10x-7y&=\lambda y
\end{array}\right.$$
\item On appelle $\Sigma$ l'ensemble des valeurs telles que le système $(S_\lambda)$ \textbf{n'est pas} de Cramer. Déterminer $\Sigma$.
\item Résoudre $(S_\lambda)$ pour $\lambda \in\Sigma$.
\item Résoudre $(S_\lambda)$ pour $\lambda \notin\Sigma$.
\end{enumerate}

\end{exercice}

\begin{correction}
\begin{enumerate}
\item \begin{align*}
(S_\lambda)\equivaut &\left\{\begin{array}{rcrl}
(8-\lambda)x &+&5y&=0\\
-10x&+&(-7-\lambda)y&=0
\end{array}\right.
\equivaut \left\{\begin{array}{rcrl}
-10x&+&(-7-\lambda)y&=0\\
(8-\lambda)x &+&5y&=0
\end{array}\right.\\
\overset{L_2\leftarrow 10L_2+(8-\lambda)L_1}{\equivaut} &\left\{\begin{array}{rcrl}
-10x&+&(-7-\lambda)y&=0\\
& &(5\times 10+(8-\lambda)(-7-\lambda))y&=0
\end{array}\right.
\\
\equivaut &\left\{\begin{array}{rcrl}
-10x&+&(-7-\lambda)y&=0\\
& &(50+(\lambda^2-\lambda -56))y&=0
\end{array}\right.
\\
\equivaut &\left\{\begin{array}{rcrl}
-10x&+&(-7-\lambda)y&=0\\
& &(\lambda^2-\lambda -6)y&=0
\end{array}\right.
\end{align*}

Regardons maintenant les racines de $(\lambda^2-\lambda -6)$. Le discriminant vaut $\Delta = 1+24=25>0$. Il y a donc deux racines réelles : 
$$\lambda_1 =-2\quadet \lambda_2=3$$

\underline{Si $\lambda_1 \notin \{-2,3\}$ }
alors $(\lambda^2-\lambda -6)\neq 0$ et 
\conclusion{ Le système  est de rang 2}

\underline{Si $\lambda=-2$ ou  $\lambda=3$ }
alors $(\lambda^2-\lambda -6)=0$ et 
\conclusion{ Le système  est de rang 1}

\item $\Sigma =\{ -2,3\}$ d'après la question précédente. 

\item \underline{Si $\lambda =-2$} alors 
$$(S_\lambda)\equivaut \left\{\begin{array}{rcrl}
-10x&+&(-7+2)y&=0\\
& &0&=0
\end{array}\right.\equivaut \left\{\begin{array}{rcrl}
-10x&+&-5y&=0
\end{array}\right.
$$
$$(S_\lambda)\equivaut\left\{\begin{array}{rl}
x&=\frac{-1}{2}y
\end{array}\right.
$$
Pour $\lambda=-2$ les solutions sont :
\conclusion{ $S = \{ (\frac{-1}{2}y,y) \, |\, y \in \R\}$}


\underline{Si $\lambda =3$} alors 
$$(S_\lambda)\equivaut \left\{\begin{array}{rcrl}
-10x&+&(-7-3)y&=0\\
& &0&=0
\end{array}\right.\equivaut \left\{\begin{array}{rcrl}
-10x&+&-10y&=0
\end{array}\right.
$$
$$(S_\lambda)\equivaut\left\{\begin{array}{rl}
x&=-1y
\end{array}\right.
$$
Pour $\lambda=3$ les solutions sont :
\conclusion{ $S = \{ (-y,y) \, |\, y \in \R\}$}

\item \underline{Si $\lambda \notin \Sigma$} alors 
Le système est de Cramer, comme il est homogéne $(0,0)$ est solution. 
\conclusion{ $S = \{ (0,0) \}$}
\end{enumerate}

\end{correction}

\vspace{1cm}
\begin{exercice}
On s'intéresse dans cet exercice aux suites $\suite{u}$, $\suite{v}$ qui vérifient les relations suivantes \footnote{Bien que les coéfficients soient les mêmes que dans l'exercice précédent les deux exercices sont indépendants. } : 
$$(R)\, :\, \left\{\begin{array}{rl}
u_{n+1} &=8u_n +5v_n\\
v_{n+1}&= -10u_n-7v_n
\end{array}\right. \quadet u_0=1, v_0=1.$$

On propose deux solutions distinctes. 

\paragraph{Méthode 1}
\begin{enumerate}

\item On considère $X_n = 2u_n+v_n$ et $Y_n=u_n+v_n$.
Montrer que $\suite{X}$ et $\suite{Y}$ sont géométriques. 
\item En déduire la valeur de $X_n$ et $Y_n$ en fonction de $n$. 

\item Résoudre le système d'inconnue $(U,V)\in \R^2$ et de paramètres $(X,Y)\in \R^2$ 
$$(P)\,:\, \left\{\begin{array}{rl}
2U +V&=X\\
U+V&=Y
\end{array}\right.$$

\item En déduire l'expression de $u_n$ et $v_n$ en fonction de $X_n$ et $Y_n$. 
\item Conclure en donnant l'expression de $X_n$ en fonction de $n$.
\end{enumerate}
\paragraph{Méthode 2 }
\begin{enumerate}
\item A l'aide\footnote{Au cours des calculs il est judicieux de garder des formules factorisées $(5\times 7= 7\times 5)$... } de la relation $(R)$, montrer que pour tout $n\in \N$ 
$$u_{n+2}=u_{n+1}+6u_n$$
\item En déduire l'expression de $\suite{u}$ en fonction de $n\in \N$. 
\end{enumerate}
\end{exercice}

\begin{correction}
Méthode 1
\begin{enumerate}
\item \begin{align*}
X_{n+1}&= 2u_{n+1} +v_{n+1}\\
			&= 2(8u_n+5v_n) -10u_n-7v_n\\
			&= 6u_n+3v_n\\
			&= 3(2u_n+v_n)\\
			&= 3X_n
\end{align*}

\conclusion{Donc $\suite{X}$ est géométrique de raison $3$}

\begin{align*}
Y_{n+1}&= u_{n+1} +v_{n+1}\\
			&= 8u_n+5v_n -10u_n-7v_n\\
			&= -2u_n-2v_n\\
			&= -2(u_n+v_n)\\
			&= -2Y_n
\end{align*}

\conclusion{Donc $\suite{Y}$ est géométrique de raison $-2$}

\item $X_0=2u_0+v_0=3$ et $Y_0=u_0+v_0=2$
Comme $\suite{X}$ et $\suite{Y}$ sont géométriques on  a 
\conclusion{ $\forall n \in \N, \, X_n = 3\times 3^n \quadet Y_n=2(-2)^n$}

\item 
$$(P)\,:\, \left\{\begin{array}{rl}
2U +V&=X\\
U+V&=Y
\end{array}\right.\overset{L_2\leftarrow2L_2-L_1}{\equivaut } \left\{\begin{array}{rcrl}
2U &+&V&=X\\
&&V&=2Y-X
\end{array}\right.$$

$$(P) \equivaut \left\{\begin{array}{rcrl}
2U &+&2Y-X&=X\\
&&V&=2Y-X
\end{array}\right.\equivaut \left\{ \begin{array}{rl}
U&=-Y+X\\
V&=2Y-X
\end{array}\right.$$

Les solutions de $(P)$ sont : 
\conclusion{ $\cS=\{ (-Y+X, 2Y-X)\}$}
\item 
La question précédente montre que 
\conclusion{$u_n=-Y_n+X_n \quadet v_n=2Y_n-X_n$}

\item On obtient alors en remplacant les valeurs de $\suite{X}$ et $\suite{Y}$:
\conclusion{$u_n = -2(-2)^n+3\times 3^n= (-2)^{n+1} +3^{n+1}$}

\end{enumerate}
Methode 2
\begin{enumerate}
\item D'aprés la premiere condition de la relation $(R)$ on a  
$$u_{n+2} =8u_{n+1} +5v_{n+1}$$
Or $v_{n+1} = (-10u_n-7v_n)$ donc 
\begin{align*}
u_{n+2} &=8u_{n+1} +5 (-10u_n-7v_n)\\
			&=8u_{n+1}-50u_n -7\times 5 v_n
\end{align*}
Or $5v_n= u_{n+1}-8u_n$ donc 
\begin{align*}
u_{n+2} &=8u_{n+1}-50u_n -7\times  (u_{n+1}-8u_n )\\
			&=u_{n+1}+6u_n
\end{align*}

\item La suite $\suite{u}$ est donc une suite récurrente linéaire d'ordre 2 à coefficients constants. Son équation caractéristique est $X^2=X+6$
On cherche donc les racines de $X^2-X-6$.  Le discriminant vaut $\Delta = 1+24=25>0$. Il y a donc deux racines réelles : 
$$x_1 =-2\quadet x_2=3$$

La suite $\suite{u}$ s'écrit alors 
$$u_n = A(-2)^n+B3^n$$
où $A,B$ sont deux réels à déterminer. 

Comme $u_0= 1= A+B$ et $u_1=8+5=13=-2A+3B$ on tombe sur 
$$A=(-2) \quadet B=3$$

On obtient de nouveau : 
\conclusion{$u_n =  (-2)^{n+1} +3^{n+1}$}
\end{enumerate}



\end{correction}

\begin{exercice}
Pour chaque script, dire ce qu'affiche la console : 


\begin{minipage}{0.45\textwidth}   %left column
\begin{enumerate}
\item  \texttt{Script1.py}

\begin{lstlisting}[language=Python]
a=0
n=10
for i in range(n):
  a=a+i^3
print(a/25)
\end{lstlisting}

\vspace{1cm}


\item \texttt{Script2.py}
\begin{lstlisting}[language=Python]
a=0
x=3.1415926
while a<x:
  a=a+1
print(a)
\end{lstlisting}



\vspace{1cm}


\end{enumerate}
\end{minipage}
\hfill\vline\hfill
\begin{minipage}{0.44\textwidth} %right column

\begin{enumerate} \setcounter{enumi}{2}
\vspace{0.4cm}
\item \texttt{Script3.py} On rappelle que \texttt{floor} calcule la partie entière d'un nombre
\begin{lstlisting}[language=Python]
from math import floor
x=12
a=0
b=100
c=50
for i in range(4):
  if c>x:
    b=c
    c=floor((a+b)/2)
  else:
    a=c
    c=floor((a+b)/2)
  print(a,b,c)
\end{lstlisting}
\vspace{0.4cm}
\item \texttt{Script4.py}
\begin{lstlisting}[language=Python]
a=78
for i in range(1,79):
  if a%i==0:
    print(i)
\end{lstlisting}


\end{enumerate}



\end{minipage}
\begin{enumerate}
 \setcounter{enumi}{4}
 \item Ecrire un script Python qui permet d'afficher les termes de $0$ à $100$ de la suite $\suite{u}$, définie par 
$$u_0= 1 \quadet \forall n \in\N,\, u_{n+1} =u_n^2-u_n+1.$$
 
 
 \item Ecrire un script Python qui permet d'afficher le  terme $u_{100}$ de la suite $\suite{u}$, définie par 
$$u_0= 1, u_1=1 \quadet \forall n \in\N,\, u_{n+2} =u_{n+1}-u_n^2.$$ 
On pourra ici considérer deux variables \texttt{u}, \texttt{v} qui correspondent respectivement à $u_n$ et $u_{n+1}$
\end{enumerate}


\end{exercice}



\end{document}