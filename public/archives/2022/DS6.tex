\documentclass[a4paper, 11pt,reqno]{article}
\input{/Users/olivierglorieux/Desktop/BCPST/2020:2021/preambule.tex}
\usepackage{enumitem}
\geometry{hmargin=3.5cm, vmargin=3.5cm}
\lstset{basicstyle=\ttfamily, keywordstyle=\rmfamily\bfseries}
\newif\ifshow
\showfalse
\input{/Users/olivierglorieux/Desktop/BCPST/2021:2022/ifshow.tex}

\author{Olivier Glorieux}


\begin{document}

\title{DS 6\\
\Large{Durée 3h00}
}




\vspace{1cm}
\begin{center}

\begin{description}
\item$\bullet$ Les calculatrices sont \underline{interdites} durant les cours, TD et \emph{a fortiori} durant les DS de mathématiques. \\

\item $\bullet $ Si vous pensez avoir découvert une erreur, indiquez-le clairement sur la copie et justifiez les initiatives que vous êtes amenés à prendre. \\

\item $\bullet$ Une grande attention sera apportée à la clarté de la rédaction et à la présentations des solutions. (Inscrivez clairement en titre le numéro de l'exercice, vous pouvez aussi encadrer les réponses finales.)  \\

\item $\bullet$ Vérifiez vos résultats. \\

\item $\bullet$ Le résultat d'une question peut être admis et utilisé pour traiter les questions suivantes en le signalant explicitement sur la copie. 
\end{description}

\end{center} 

\newpage
\begin{exercice}
Soit $P_1$ le plan de l'espace d'équation $x+y+z+1=0$ et $P_2$ le plan de l'espace  d'équation $2x-y-z+2=0$
\begin{enumerate}
\item Justifier que ces deux plans s'intersctent le long d'une droite que l'on note $D$
\item Donner un vecteur directeur de $D$. 
\item Soit $A$ le point de coordonnées $(2,1,0)$. Déterminer les coordonnées de $H$,  projeté orthogonal de $A$ sur $P_1$
\end{enumerate}
\end{exercice}
\vspace{2cm}
\begin{exercice}  \;
D\'eterminer les coordonnées de $M$, intersection de $\mathcal{D} : 2x+5y-10=0$ et de la droite $\mathcal{D}'$ passant par $A(-1,2)$ et dirig\'ee par $\vec{u}(3,2)$.
\end{exercice}
\vspace{2cm}
\begin{exercice}
Soit $P_0=1$ et pour $n\geq 1$,  $P_n =(X^2-1)^n$. 
\begin{enumerate}
\item Donner le degré et le coefficient dominant de $P_n$. 
\item Donner les racines de $P_n$ ainsi que leur multiplicités. \\

Soit $Q_n= P_n^{(n)}$ où $(n)$ désigne la dérivée $n$-éme. 
\item Calculer $Q_0, Q_1$ et $Q_2$. 
\item Que vaut $P_n^{(n-1)} (1)$ et $P_n^{(n-1)} (-1)$?
\item Soit $v(x)= P_m^{(m-1)}(x)$. Justifier que $v$ est dérivable et que $v'(x)=Q_m(x)$
\item Montrer à l'aide d'une intégration par parties que pour tout $(n,m)\in \N^2$ 
$$\int_{-1}^{1} Q_n(x) Q_m(x)dx = -\int_{-1}^{1} P_n^{(n+1)}(x) P_m^{(m-1)} (x)dx$$
\item Montrer à l'aide d'une récurrence sur $k$ que pour tout $(n,m)\in \N^2$ et tout $k\in\intent{0,m}$ que 
$$\int_{-1}^{1} Q_n(x) Q_m(x)dx = (-1)^k\int_{-1}^{1} P_n^{(n+k)}(x) P_m^{(m-k)} (x)dx$$
\item On suppose que $n<m$ que vaut $P_n^{(n+m)}$ ?
\item On suppose que $n<m$ déduire des questions précédentes que 
$$\int_{-1}^{1} Q_n(x) Q_m(x)dx = 0$$
\end{enumerate}
\end{exercice}

\vspace{2cm}
\begin{exercice}
Soit $\suite{T}$ la suite de polynômes définie par 
$T_0 =1, \, T_1=X$ et 
$$\forall n\in \N, \, T_{n+2}=2XT_{n+1} -T_n$$
\begin{enumerate}
\item Expliciter $T_2$ et $T_3$
\item Montrer, à l'aide d'une récurrence double, que $\deg(T_n)=n$ et  son coefficient dominant  vaut $2^n$. 
\item Montrer que pour tout $\theta\in \R$ et tout $n\in \N$ que 
$$2\cos(\theta)\cos((n+1)\theta) -\cos(n\theta) = \cos((n+2)\theta)$$
\item En déduire, à l'aide d'une récurrence double, que pour tout $\theta \in \R$ et tout $n\in \N$ on a 
$$T_n(\cos(\theta))=\cos(n\theta)$$
%\item En déduire les valeur de $T_n(1)$.
\item Résoudre $\cos(n\theta) =0$ pour $\theta \in [0,\pi]$
\item Déterminer les racines de $T_n$ appartenant à  l'intervalle $[-1,1]$.
\item Justifier que l'on obtient ainsi toutes les racines de $T_n$. 
\item En déduire la factorisation de $T_n$ dans $\R[X]$
\end{enumerate}
\end{exercice}
\vspace{3cm}

\begin{exercice}
On souhaite représenter informatiquement un polynôme. Pour cela, à un polynôme $P=\ddp \sum_{k=0}^n a_k X^k$ on associe la liste 
$[a_0, a_1, \dots, a_n]$. Par exemple le polynôme $P=1+X+X^3$ serait représenté par la liste $L=[1,1,0,1]$
\begin{enumerate}
\item Soit $Q=1+X^2- 2X^3 +X^5$. Donner la liste représentant $Q$. 
\item Expliciter le polynôme représenté par la liste \texttt{[0,0,1,0,0]}. Donner une autre liste qui représente aussi ce polynôme. 
\item Ecrire une fonction \texttt{evaluation} qui prend en argument un flotant $x$ et  une liste $L$  qui représente un polynôme $P$ et retourne la valeur de $P(x)$.
Par exemple \texttt{evaluation([1,1,0,1],2)} doit retourner la valeur 11.
\item 
Ecrire une fonction \texttt{simplification} qui prend en argument une liste \texttt{L} et retourne une liste qui ne comporte pas de $0$ à droite. 
Par exemple \texttt{simplication([1,1,0,1])} retourne \texttt{[1,1,0,1]} et 
 \texttt{simplication([1,1,0,1,0,0,0,0])} retourne \texttt{[1,1,0,1]}. On rappelle que \texttt{L.pop()} enlève le dernier élément d'une liste \texttt{L}.
 
\item Ecrire une fonction \texttt{degre} qui prend en argument une liste \texttt{L} représentant un polynôme $P$ et retourne le degré de $P$. Par exemple \texttt{degre([1,1,0,1])} retourne 3. On fera attention qu'un polynôme puisse être représenté par plusieurs listes comme on l'a vu dans la question 2

\item  La fonction suivante ne renvoit rien, elle modifie les listes \texttt{L1} et \texttt{L2}
\begin{lstlisting}[language =Python]
def mystere(L1,L2):
  n1,n2 =len(L1), len(L2)
  L1=L1+[0]*len(L2)
  L2=L2+[0]*len(L1)
\end{lstlisting}

Que vaut la longueur de \texttt{L1} et \texttt{L2} après avoir appliqué la fonction \texttt{mystere(L1,L2)} (on répondra en fonction de  \texttt{n1=L1} et \texttt{n2=L2} les longueurs des listes \underline{avant} l'éxécution de \texttt{mystere}) 


\item Ecrire une fonction \texttt{somme} qui prend en agument deux listes \texttt{L1} et \texttt{L2} représentant des polynômes $P_1$ et $P_2$ et retourne une liste qui correspond au polynôme $P_1+P_2$.
Par exemple \texttt{somme([1,2,3],[0,-2])} retourne \texttt{[1,0,3]}
%\item Ecrire une fonction \texttt{derivation} qui prend en agument  une liste qui correspond à un polynôme $P$ et retourne une liste qui correspond au  polynôme dérivée $P'$. Par exemple 
%\texttt{derivation([1,1,0,1])} retourne \texttt{[1,0,3]}
\item Que fait la fonction suivante où $L$ est une liste qui représente un polynôme $P$. 
\begin{lstlisting}[language=Python]
def mystere2(L):
  D=[]
  for k in range(1,len(L)):
     D.append(k*L[k])
  return(D)
\end{lstlisting}

\item Ecrire une fonction \texttt{multipl} qui prend en argument
une liste qui correspond à un polynôme $P$ et retourne une liste qui correspond au  polynôme $2XP$. Par exemple 
\texttt{multipl([1,1,0,1])} retourne \texttt{[0,2,2,0,2]}
\item Ecrire une fonction \texttt{Tchebychev} qui prend en argument un entier $n$ et retourne la liste correspondant  au polynôme $V_n$ défini par $V_0 =1, \, V_1=X$ et 
$$\forall n\in \N, \, V_{n+2}=2XV_{n+1} +V_n$$
Par exemple \texttt{Tchebychev(1)} retourne \texttt{[0,1]}


\end{enumerate}

\end{exercice}



\end{document}