\documentclass[a4paper, 11pt,reqno]{article}
\input{/Users/olivierglorieux/Desktop/BCPST/2020:2021/preambule.tex}
\newif\ifshow
\showtrue
\input{/Users/olivierglorieux/Desktop/BCPST/2021:2022/ifshow.tex}

%\geometry{hmargin=2.0cm, vmargin=2cm}

\author{Olivier Glorieux}
\begin{document}
\title{DM 11 - Probabilité }






\begin{exercice}   \;
Une urne A contient 1 boule rouge et 2 noires. Une urne B contient 3 rouges et 1 noire. Au d\'epart, on choisit une urne, la probabilit\'e de choisir l'urne A est $p\in \; \rbrack 0,1\lbrack$. Puis on choisit une boule dans cette urne. Si, \`a un tirage quelconque, on a tir\'e une boule rouge, le tirage suivant se fait dans A, sinon, on choisit une boule de B. Les tirages se font avec remise. On note $p_n$ la probabilit\'e de choisir une boule rouge au tirage de num\'ero $n$.
\begin{enumerate}
\item  Calculer $p_1$.
\item Exprimer $p_{n+1}$ en fonction de $p_n$.
\item  En d\'eduire $p_n$ en fonction de $n$.
\item  Calculer la limite de la suite $(p_n)_{n\in\N^{\star}}$.
\end{enumerate} 
\end{exercice}
\begin{correction}
On note $R_n$ l'événement " tirer une Rouge au tirage $n$" et $N_n$: " tirer une Noire au tirage $n$". On a évidemment $\overline{N_n} =R_n$. 

\begin{align*}
p_{n+1} &= \bP(R_{n+1} ) \quad \text{Par définition} \\
			&= \bP(R_{n+1}  | R_n ) \bP(R_n) +   \bP(R_{n+1}  | N_n ) \bP(N_n)  \quad \text{Par la formule des probabilités totales} 
\end{align*}

$\bP(R_{n+1}  | R_n ) =\frac{1}{3}$  car si on a tiré une boule rouge au tirage $n$ le tirage se fait dans l'urne $A$ qui contient 3 boules dont seuleemnt une noire. De même $\bP(R_{n+1}  | N_n ) =\frac{3}{4}$.
Ainsi
\begin{align*}
p_{n+1} &= \frac{1}{3}p_n + \frac{3}{4}(1-p_n)\\
			&= \frac{-5}{12} p_n + \frac{3}{4}
\end{align*}

C'est une suite arithmético géométrique. On cherche $\ell\in \R$ tel que 
$$\ell =  \frac{-5}{12} \ell+ \frac{3}{4}$$
on trouve $\ell = \frac{9}{17}$
On sait d'après le cours (ou  on refait le calcul) que la suite $u_n=p_n - \ell  $ est géométrique de raison $\frac{-5}{12}$. 
Ainsi pour tout $n\in \N^*$ on a 
$$u_n = u_1 \left(\frac{-5}{12} \right)^{n-1}$$
et $u_1 = p_1- \frac{9}{17} $
Il faut encore calculer $p_1$ 
On a $p_1= \bP(R_1) = \bP(R_1|A) \bP(A) +  \bP(R_1|B) \bP(B)$ d'après la formule des probabilités totales (ici $A$ et $B$ sont les événements 'choix de l'urne .... ') 
On a donc $p_1 = \frac{1}{3}p +\frac{3}{4}(1-p)= \frac{-5}{12} p + \frac{3}{4}$

Finalement $u_1= \frac{-5}{12} p + \frac{3}{4} - \frac{9}{17}=  \frac{-5}{12} p-\frac{15}{68}$

Et $$p_n = \frac{9}{17}  + \left(\frac{-5}{12} p-\frac{15}{68}\right) \left(\frac{-5}{12} \right)^{n-1}$$
La limite de $p_n$ est $\frac{9}{17}$.

\end{correction}










\begin{exercice}

On considère trois points distincts du plan nommés $A, B$ et $C$. Nous allons étudier le déplacement aléatoire d'un pion se déplaçant sur ces trois points. A l'étape $n=0$, on suppose que le pion se trouve sur le point $A$. Ensuite, le mouvement aléatoire du pion respecte les deux règles suivantes :
\begin{itemize}
\item  le mouvement du pion de l'étape $n$ à l'étape $n+1$ ne dépend que de la position du pion à l'étape $n$;
\item pour passer de l'étape $n$ à l'étape $n+1$, on suppose que le pion a une chance sur deux de rester sur place, sinon il se déplace de manière équiprobable vers l'un des deux autres points.

\end{itemize}

Pour tout $n \in \mathbb{N}$, on note $A_{n}$ l'évènement "le pion se trouve en $A$ à l'étape $n$ ", $B_{n}$ l'évènement "le pion se trouve en $B$ à l'étape $n$ " et $C_{n}$ l'évènement "le pion se trouve en $C$ à l'étape $n$ ". On note également, pour tout $n \in \mathbb{N}$,
$$
a_{n}=P\left(A_{n}\right), b_{n}=P\left(B_{n}\right), c_{n}=P\left(C_{n}\right) \text { et } V_{n}=\left(\begin{array}{l}
a_{n} \\
b_{n} \\
c_{n}
\end{array}\right)
$$
\begin{enumerate}
\item  Calculer les nombres $a_{n}, b_{n}$ et $c_{n}$ pour $n=0,1$.
\item  Pour $n \in \mathbb{N}$, exprimer $a_{n+1}$ en fonction de $a_{n}, b_{n}$ et $c_{n} .$ Faire de même pour $b_{n+1}$ et $c_{n+1}$.
\item  Donner une matrice $M$ telle que, pour tout $n \in \mathbb{N}$, on a $V_{n+1}=M V_{n}$.
\item  On admet que, pour tout $n \in \mathbb{N}$, on a
$$
M^{n}=\frac{1}{3 \cdot 4^{n}}\left(\begin{array}{ccc}
4^{n}+2 & 4^{n}-1 & 4^{n}-1 \\
4^{n}-1 & 4^{n}+2 & 4^{n}-1 \\
4^{n}-1 & 4^{n}-1 & 4^{n}+2
\end{array}\right)
$$
En déduire une expression de $a_{n}, b_{n}$ et $c_{n}$ pour tout $n \in \mathbb{N}$.
\item  Déterminer les limites respectives des suites $\left(a_{n}\right),\left(b_{n}\right)$ et $\left(c_{n}\right)$. Interpréter le résultat.
\end{enumerate}


\end{exercice}



\begin{correction}
\begin{enumerate}


\item  Puisqu'en $n=0$ le pion est en $A$, on a $a_{0}=1, b_{0}=0$ et $c_{0}=0 .$ A l'étape $n=1$, d'après les informations de l'énoncé, $a_{1}=1 / 2, b_{1}=c_{1}$. Puisque $a_{1}+b_{1}+c_{1}=1$, on a $b_{1}=c_{1}=1 / 4$.
\item  Les événements $A_{n}, B_{n}$ et $C_{n}$ forment un système complet d'événements. D'après la formule des probabilités totales,
$$
P\left(A_{n+1}\right)=P_{A_{n}}\left(A_{n+1}\right) P\left(A_{n}\right)+P_{B_{n}}\left(A_{n+1}\right) P\left(B_{n}\right)+P_{C_{n}}\left(A_{n+1}\right) P\left(C_{n}\right) .
$$
Comme à la question précédente, on a $P_{A_{n}}\left(A_{n+1}\right)=1 / 2, P_{B_{n}}\left(A_{n+1}\right)=1 / 4$ et $P_{C_{n}}\left(A_{n+1}\right)=1 / 4$. On en déduit que
$$
a_{n+1}=\frac{1}{2} a_{n}+\frac{1}{4} b_{n}+\frac{1}{4} c_{n}
$$
En raisonnant de la même façon, ou par symétrie,
$$
\begin{gathered}
b_{n+1}=\frac{1}{4} a_{n}+\frac{1}{2} b_{n}+\frac{1}{4} c_{n} \\
c_{n+1}=\frac{1}{4} a_{n}+\frac{1}{4} b_{n}+\frac{1}{2} c_{n}
\end{gathered}
$$
\item  D'après la question précédente, la matrice
$$
M=\frac{1}{4}\left(\begin{array}{lll}
2 & 1 & 1 \\
1 & 2 & 1 \\
1 & 1 & 2
\end{array}\right)
$$
convient.
\item  On a $V^{n}=M^{n} V_{0}$, Il suffit donc de réaliser le produit $M^n $ par$V_0 = \begin{pmatrix}
1\\
0\\
0
\end{pmatrix} $ d'après la question 1.  ce qui donne
$$
V_n=\frac{1}{3 \cdot 4^{n}}
\begin{pmatrix}
\ddp 4^{n}+2 \\
\ddp 4^{n}-1\\
\ddp 4^{n}-1
\end{pmatrix}
$$
(C'est la première colonne de $M^n$) 
On obtient \emph{in fine}

\conclusion{ $a_n = \ddp \frac{4^{n}+2}{3 \cdot 4^{n}}, b_n = \ddp \frac{4^{n}-1}{3 \cdot 4^{n}} $ et $ c_n =\ddp  \frac{4^{n}-1}{3 \cdot 4^{n}}$}

On remarque qu'on a bien $a_{n}+b_{n}+c_{n}=1$.
\end{enumerate}
\end{correction}


\end{document}