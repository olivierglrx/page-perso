\documentclass[a4paper, 11pt,reqno]{article}
\input{/Users/olivierglorieux/Desktop/BCPST/2020:2021/preambule.tex}
\geometry{hmargin=1.5cm,vmargin=2.5cm }
\usepackage{enumitem}
\newif\ifshow
\showtrue
\input{/Users/olivierglorieux/Desktop/BCPST/2021:2022/ifshow.tex}



\author{Olivier Glorieux}


\begin{document}

\title{Correction - DM 5 
}

\begin{exercice}
Soit $z,z'$ deux nombres complexes.

\begin{enumerate}
\item Rappeler les valeurs de $A=z\bar{z}$, $B=|z\bar{z}|$, $C=|\bar{z}z'|^2$ en fonction de $|z|$ et $|z'|$. 
\item On suppose dans cette question et  la suivante que $|z|<1 $ et $|z' |<1$. Montrer que $$\bar{z}z'\neq 1$$

\item  Montrer que 
$$1- \left| \frac{z-z'}{1-\bar{z} z' } \right|^2 = \frac{(1-|z'|^2)(1-|z|^2)}{|1-\bar{z}z'|^2}$$
\item Soit $\suite{z}$ une suite de nombres complexes vérifiant : $|z_0|<1, |z_1|<1$  et pour tout $n\in \N$ :
$$z_{n+2} =\frac{z_n-z_{n+1}}{1-\bar{z_{n}} z_{n+1}}$$

Montrer que pour tout $n\in \N$, $|z_n|<1$ et que $\bar{z_n}z_{n+1}\neq 1$, et donc que $\suite{z}$ est bien définie pour tout $n\in \N$. \\
\footnotesize{On pourra utiliser les deux questions précédentes dans une récurrence double}

\end{enumerate}
\end{exercice}

\begin{correction}
\begin{enumerate}
\item $A=|z|^2$, $B=|z||z'|$ $C=|z|^2|z'|^2$
\item Comme $|z|<1$ et $|z'|<1$ on a $|\bar{z}z'|= |\bar{z}| |z'| =|z||z'| <1$. Or si deux nombres complexes sont égaux ils ont même module, donc $\bar{z}z' $ ne peut pas être égal à $1$, sinon ils auraient le même module.  \\

C'est un raisonement par l'absurde, voici la bonne façon de le rédiger : \\

Soit deux complexes $z,z'$ tel que $|z|<1, |z'|<1$.  On suppose par l'absurde que $\overline{z}z'=1$. 
On a alors $|\overline{z}z'|=|1|$ et donc $|z||z'|=1$ 
Comme $|z|<1, |z'|<1$, on obtient $|z||z'|<1$ et finalement $1<1$ ce qui est absurde. Ainsi $\overline{z}z'\neq1$. )


\item Après avoir mis au même dénominateur le membre de gauche, on va utiliser le fait que pour tout complexe $u$, on a $|u|^2 = u\bar{u}$ :
\begin{align*}
1- \left| \frac{z-z'}{1-\bar{z} z' } \right|^2  &= \frac{| 1-\bar{z} z' |^2 -|z-z' |^2}{|1-\bar{z} z'  |^2}\\
&= \frac{( 1-\bar{z} z' )(\bar{ 1-\bar{z} z'} )   -(z-z' )(\bar{z-z'})}{|1-\bar{z} z'  |^2}\\
&= \frac{( 1-\bar{z} z' )(1-z\bar{z'}  )   -(z-z' )(\bar{z}-\bar{z'})}{|1-\bar{z} z'  |^2}\\
&= \frac{( 1-\bar{z} z' -z\bar{z'} + |\bar{z}z'|^2  )   -(|z|^2-\bar{z'}z - \bar{z}z' +|z'|^2)}{|1-\bar{z} z'  |^2}\\
&= \frac{( 1+ |\bar{z}z'|^2-|z|^2-|z'|^2)}{|1-\bar{z} z'  |^2}
\end{align*}
Remarquons enfin que $(1-|z|^2) (1-|z'|^2)  =1 +|zz'|^2 -|z|^2 -|z'|^2$. Or 
$ |\bar{z}z'|^2 = |\bar{z}|^2|z'|^2 =|z|^2 |z'|^2  = |zz'|^2$.
On a bien 
\conclusion{$\ddp 1- \ddp \left| \ddp \frac{z-z'}{1-\bar{z} z' } \right|^2    = \frac{(1-|z|^2) (1-|z'|^2) }{|1-\bar{z} z'  |^2}$}


\item Soit $P(n)$ la propriété : \og $|z_n|<1$ et $|z_{n+1}|<1$\fg \,. Remarquons que d'après la question 2, $P(n)$ implique que $\bar{z_n}z_{n+1}\neq 1 $ et donc que $z_{n+2}$ est bien définie. 

Prouvons $P(n)$ par récurrence. 

\underline{Initialisation} : 
$P(0)$ est vraie d'après l'énoncé : $|z_0|<1 $ et $|z_1|<1$.\\

\underline{Hérédité} : On suppose qu'il existe $n\in \N$   tel que $P(n)$ soit vraie. Montrons alors $P(n+1)$ :  \og $|z_{n+1}|<1$ et $|z_{n+2}|<1$\fg. Par hypothèse de récurrence on sait déjà que $|z_{n+1}|<1$ il reste donc à prouver que $|z_{n+2} <1$. 

On a $$|z_{n+2}| = \left|\frac{z_n-z_{n+1}}{1-\bar{z_{n}} z_{n+1}}\right|$$
 Or d'après la question 3, 
 $$ \left|\frac{z_n-z_{n+1}}{1-\bar{z_{n}} z_{n+1}}\right|^2 = 1- \frac{(1-|z_n|^2) (1-|z_{n+1}|^2) }{|1-\bar{z_n} z_{n+1}  |^2}$$
 
Par hypothèse de récurrence,  $(1-|z_n|^2) (1-|z_{n+1}|^2)>0$. Le dénominateur est aussi positif, donc $\frac{(1-|z_n|^2) (1-|z_{n+1}|^2) }{|1-\bar{z_n} z_{n+1}  |^2}>0$ et ainsi :
 $$ \left|\frac{z_n-z_{n+1}}{1-\bar{z_{n}} z_{n+1}}\right|^2 < 1$$
Donc $|z_{n+2}|<1$. On a donc prouvé que la propriété $P$ était hériditaire. 

\underline{Conclusion} : Par principe de récurrence, $P(n)$ est vraie pour tout $n\in \N$ et comme remarqué au début de récurrence, ceci implique que $\suite{z}$ est bien définie pour tout $n\in \N$.  

 
\end{enumerate}
\end{correction}



\begin{exercice}
On considère l'équation du second degré suivante : 
$$z^2+(3i-4)z+1-7i=0 \quad (E) $$

\begin{enumerate}
\item A la manière d'une équation réelle, calculer le discriminant $\Delta$ du polynôme complexe, et montrer que $\Delta=3+4i$
\item On se propose de résoudre $ (E_2) \, : \, u^2=\Delta \, $  d'inconnue complexe $u$. 
\begin{enumerate}
\item On écrit $u=x+iy$ avec $(x,y)\in \R^2$. Montrer que $(E_2)$ est équivalent à 
$$ x^4-3x^2-4=0 \quadet y =\frac{2}{x}.$$
\item En déduire que les solutions de $(E_2)$ sont 
$$u_1=2+i\quadet u_2=-2-i$$
\end{enumerate}
\item Soit $u_1$ une solution de l'équation précédente. 
On considère $r_1 = \frac{-3i+4 +u_1}{2}$. Montrer que $r_1$ est solutions de l'équation  $(E)$.
\item Quelle est à l'autre solution  de  $(E)$ ? 
\end{enumerate}

\end{exercice}

\begin{correction}
On suit les étapes indiquées dans l'énoncé. 
\begin{enumerate}
\item Le discriminant vaut 
$$\Delta = (3i-4)^2 -u^4 (1-7i) = -9-24i +16 -4+28i = 3+4i$$
\item Résolvons $u^2=3+4i$. \begin{enumerate}
\item On pose donc $u=x+iy$ avec $x,y\in \R$ 
On a  donc $(x+iy)^2 = 3+4i $, soit $x^2-y^2 +2xyi =3+4i$ En identifiant partie réelle et partie imaginaire on obtient : 
$$x^2 -y^2 =3 \quad 2xy=4$$

Comme $x\neq 0 $ (sinon $\Delta\in \R_-$ ), la deuxième équation devient 
\conclusion{$y=\frac{2}{x}.$} On remplace alors $y$ avec cette valeur dans la première équation, ce qui donne : 
$$x^2 -\frac{4}{x^2}=3$$ et  en multipliant par $x^2$ 
\conclusion{ $x^4 -3x-4=0$}

\item On fait un changement de variable $X=x^2$ dans l'équation $x^4-3x^2-4=0$. On obtient 
$$X^2 -3X-4=0$$
De discriminant $\Delta_2 = 9+4*4=25=5^2$. Cette équation admet ainsi deux solutions réelles : 
$$X_1= \frac{3-5}{2}= -1\quadet X_2 =\frac{3+5}{2}=4$$
Remarquons maintenant que $X$ doit être positif car $x^2=X$ ainsi, les solutions pour la variable $x$ sont 
$$x_1 =\sqrt{4}=2 \quadet x_2 =-\sqrt{4}=-2$$
Ce qui correspond respectivement à $y_1= 1$ et $y_2= -1$
On obtient finalement deux solutions pour $u^2=\Delta $ 
à savoir 
\conclusion{$u_1= 2+i \quadet u_2 =-2-i$}



\end{enumerate}
\item  On considère donc $r_1 = \frac{-3i+4+2+i}{2}= 3-i$. Montrons que $r_1$ est solution de $(E)$ 

$$r_1^2 = (3-i)^2 = 9-6i-1=8-6i$$
$$(3i-4)r_1 =(3i-4) (3-i) = 9i+3-12+4i = -9+13i$$
Donc 
$r_1^2 +(3i-4)r_1 = 8-6i -9+13i  =-1 +7i$
Soit 
$$r_1^2 +(3i-4)r_1 +1-7i=0$$
\conclusion{Donc $r_1$ est bien solution de $(E)$. }

\item L'autre solution est sans aucun doute 
\conclusion{ $r_2 = \frac{-3i+4+u_2}{2} = 1-2i$}

\end{enumerate}
\end{correction}




\end{document}
