\documentclass[a4paper, 11pt,reqno]{article}
\input{/Users/olivierglorieux/Desktop/BCPST/2020:2021/TD/preambule_TD.tex}



\newcommand{\implique}{\Longrightarrow}
\newcommand{\equivaut}{\Longleftrightarrow}
\renewcommand{\ddp}{\displaystyle}
\newcommand\vv[1]{\overrightarrow{#1}}
%\fancyhead[R]{TD 1 : Nombres}
%\usepackage{fancybox}
\geometry{hmargin=2cm, vmargin=2cm}
\author{Olivier Glorieux}
\newtheorem{exercice}{Exercice}

\begin{document}
\title{Exercices vacances février}
\vspace{0,5cm}

\begin{center}
\textbf{Lundi 20 Février }
\end{center}
\begin{enumerate}
\item Résoudre $\frac{1}{x+1}\leq \frac{x}{x+2}$.
\item Calculer $A^3$ où $A=\begin{pmatrix}
1&-1\\
2&1\\
\end{pmatrix}$
\item Calculer l'ensemble de définition et donner la dérivée de $$f(x) =\exp(\cos(x)-1)$$
\item Déterminer $u_n$ en fonction de $n$ où $u_0=1$ et 
$$\forall n \geq 0\, u_{n+1} =2u_n+1$$
\end{enumerate}

\begin{center}
\textbf{Mardi 21 Février }
\end{center}

\begin{enumerate}
\item Résoudre $x^3+2x\leq 0$.
\item Calculer $N^2$ où $N=\begin{pmatrix}
0&1\\
0&0\\
\end{pmatrix}$. En déduire (à l'aide du binome de Newton) la valeur de $A^n$ où $A=\begin{pmatrix}
2&1\\
0&2\\
\end{pmatrix}$. 
\item Calculer l'ensemble de définition et donner la dérivée de $$f(x) =\sqrt{x^2+x+1}$$
\item Déterminer $u_n$ en fonction de $n$ où $u_0=1$ et 
$$\forall n \geq 0\, u_{n+1} =\frac{-1}{2}u_n+1$$

\item Ecrire un script Python qui permet de calculer le terme $u_n$ de la suite définie par $u_0=1$ et 
$$\forall n \in \N, \, u_{n+1} = \sin(u_n)$$
\end{enumerate}



\begin{center}
\textbf{Mercredi 22 Février }
\end{center}

\begin{enumerate}
\item Résoudre $x\leq \sqrt{x+1}$.
\item Soit $A=\begin{pmatrix}
1 & 2\\
2 & 3
\end{pmatrix}$ Résoudre l'équation
$$
AX=\begin{pmatrix}
1\\
2
\end{pmatrix}$$
 d'inconnue $X=\begin{pmatrix}
x\\y
\end{pmatrix}$.
\item Calculer l'ensemble de définition et donner la dérivée de $$f(x) =\sqrt{ \ln(x) +1}$$
\item Déterminer $u_n$ en fonction de $n$ où $u_0=1, u_1=2$ et 
$$\forall n \geq 0\, u_{n+2} =5u_{n+1}-6u_n$$
\item Ecrire un script Python qui permet de calculer le terme $S_n$ de la suite définie par 
$$\forall n \in \N, \, S_{n} = \sum_{k=1}^n\sin(k)$$
\end{enumerate}



\begin{center}
\textbf{Jeudi 23 Février }
\end{center}
\begin{enumerate}
\item Résoudre le systéme de d'inconnue $(x,y)$ et de paramétre $\lambda \in \R$
$$\left\{ \begin{array}{ccc}
x+y&=& \lambda x\\
x-y&=& \lambda y\\
\end{array}\right.$$

\item Résoudre l'équation
$$
AX=\begin{pmatrix}
1\\
2
\end{pmatrix}$$
 d'inconnue $X=\begin{pmatrix}
x\\y
\end{pmatrix}$  et où $A=\begin{pmatrix}
1 & 2\\
2 & 4
\end{pmatrix}$
\item Calculer l'ensemble de définition et donner la dérivée de $$f(x) =\frac{x}{\ln(x)-1}$$
\item Calculer $\int_1^2 xe^xdx$
\item Ecrire un fonction Python qui prend en argument une liste d'entier et retourne le maximum de cette liste. 
\end{enumerate}



\begin{center}
\textbf{Vendredi 24 Février }
\end{center}

\begin{enumerate}
\item Résoudre $\frac{1}{\sqrt{x+1}}\leq \sqrt{x}$.
\item Déterminer une équation cartésienne de la droite du plan passant par $A=(1,2)$ et $B=(3,4)$
\item Calculer l'ensemble de définition et donner la dérivée de $$f(x) =\frac{x^2}{x-1}$$
\item Calculer $\int_0^\pi x\cos(x)dx$
\item Ecrire un fonction Python qui prend en argument une liste d'entier et retourne le minimum de cette liste. 
\end{enumerate}

\begin{center}
\textbf{Samedi 25 Février }
\end{center}
\begin{enumerate}
\item Montrer que $f $ définie par $f(x) = xe^x$ réalise une bijection entre deux intervalles de $\R$ à déterminer. 
\item Déterminer une équation cartésienne de la droite du plan passant par $A=(1,2)$ et dirigée par $\vv{u}=\begin{pmatrix}
1\\-1
\end{pmatrix}$
\item Calculer l'ensemble de définition et donner la dérivée de $$f(x) =\frac{x^2}{x-\sqrt{x}}$$
\item Calculer $\int_1^2 \frac{1}{x\sqrt{x}} 
dx$
\item Ecrire un fonction Python qui prend en argument une liste d'entier la moyenne. 
\end{enumerate}



\begin{center}
\textbf{Dimanche 26 Février }
\end{center}

\vspace{0.3cm}
\begin{center}
DODO ! 
\end{center}
\vspace{0.3cm}
\begin{center}
\textbf{Lundi 27 Février }
\end{center}
\begin{enumerate}
\item Résoudre $\cos(2x+\frac{\pi}{2}) =\sin(2x)$.
\item Déterminer  l'intersection des droites  $D$ et $D'$ définie par :
\begin{itemize}
\item $D$ passe par $A=(1,2) $ et $B=(3,-2)$
\item $D'$ passe par $B$ et est normale à $\vv{n}= \begin{pmatrix}
-2\\
1
\end{pmatrix}$ 
\end{itemize}
\item Déterminer la limite de $\ln(\frac{2n+1}{n^2+1})+\ln(n+3)$
\item Calculer $\ddp \int_{-\pi/4}^0 \tan(x)
dx$

\item Ecrire un fonction Python qui prend en argument une liste d'entier la médiane (La moitié des nombres est plus petit que la médiane, l'autre moitié est plus grande. On ne se préoccupera  de savoir si c'est exactement la moitié, ou 'la moitié +1') 
\end{enumerate}



\begin{center}
\textbf{Mardi 28 Février }
\end{center}

\begin{enumerate}
\item Résoudre $e^{2x} +e^x-2\leq 0$.
\item Déterminer une équation cartésienne du plan de l'espace passant par $A=(1,2,3)$ $B=(0,1,2)$ et $C=(1,1,1)$
\item Calculer l'ensemble de définition et donner la dérivée de $$f(x) =x^x$$
\item Calculer $\ddp \int_{2}^3 \frac{x}{x^2-1}dx$
\item Ecrire un fonction Python qui prend en argument  trois entiers $a$, $b$ et $n$  et qui retourne une liste de $n$ nombres choisis aléatoirement entre $a$ et $b$ de tel sorte que les nombres soit croissant. (Il faut donc  que $n\geq b-a$ - question probablement assez difficile pour le faire bien) 
\end{enumerate}




\begin{center}
\textbf{Mercredi 1  Mars}
\end{center}
\begin{enumerate}
\item Résoudre $\frac{\ln(x)}{\ln(x)+1}\leq \ln{(x^2)}$.
\item Déterminer  une équation cartésienne du plan de l'espace passant par $A=(1,2,3)$ et dirigé par $\vv{u} =\begin{pmatrix}
1\\2\\-1
\end{pmatrix}$ et $\vv{v} =\begin{pmatrix}
1\\2\\2
\end{pmatrix}$ 
\item Simplifier $\ddp \sum_{k=1}^n 2^n$
\item Résoudre $y'+xy=2x$ avec la condition initiale $y(0)=1$
\item Ecrire une fonction qui prend deux listes correpondans aux  coordonnées de deux points du plan : $A_0=[x_0, y_0]$ et $A_1=[x_1, y_1]$ et qui retourne trois réels $(a,b,c)$ tel que $ax+by+c=0$ est une équation cartésienne de la droite $(A_0A_1)$
\end{enumerate}




\begin{center}
\textbf{Jeudi 2  Mars}
\end{center}
\begin{enumerate}
\item Résoudre l'équation d'inconnue $z\in \bC$ 
$$z^2+z+1=0$$
\item Déterminer le projeté orthogonale du point $A=(1,2,-1)$ sur le plan d'équation $x+y+z+1=0$
\item Simplifier $\ddp \sum_{k=1}^n \ln\left( \frac{n+1}{n}\right)$
\item Résoudre $y''+y=2x$ avec la condition initiale $y(0)=1$ et $y'(0)=0$
\end{enumerate}





\begin{center}
\textbf{Vendredi 4  Mars}
\end{center}

\begin{enumerate}
\item Soit $A=\begin{pmatrix}
2 &1& 0\\
 0&1 &0  \\
 -1&-1&1\\
\end{pmatrix}$. Résoudre l'équation d'inconnue $\begin{pmatrix}
x\\y\\z
\end{pmatrix}$ et de paramétre $\lambda \in \R$ suivante : 
$$(A -\lambda I_3) X = 0_{3,1}$$
\item Montrer que $A$ est inversible et donner son inverse.
\end{enumerate}




\begin{center}
\textbf{Samedi  5  Mars}
\end{center}

\begin{enumerate}
\item Montrer que les plans d'équation $x+y+z+1=0$ et $x-y+2z-3=0$ s'intersectent le long d'une droite.  Déterminer un vecteur directeur de cette droite. 
\item Determiner une équation paramétrique du plan d'équation $x+y+z-1=0$
\item Calculer la limite de $u_n=\frac{(n)! n^2}{n\ln(n)+e^n}$
\end{enumerate}

\begin{center}
\textbf{Dimanche  6  Mars}
\end{center}
\begin{center}
DODO ! 
\end{center}


\end{document}