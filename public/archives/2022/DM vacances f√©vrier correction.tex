\documentclass[a4paper, 11pt,reqno]{article}
\input{/Users/olivierglorieux/Desktop/BCPST/2020:2021/preambule.tex}
\newif\ifshow
\showtrue
\input{/Users/olivierglorieux/Desktop/BCPST/2021:2022/ifshow.tex}

\usepackage{cancel}



\newcommand\vv[1]{\overrightarrow{#1}}
%\fancyhead[R]{TD 1 : Nombres}
%\usepackage{fancybox}
\geometry{hmargin=2cm, vmargin=2cm}
\author{Olivier Glorieux}


\begin{document}
\title{Exercices vacances février}
\vspace{0,5cm}

\begin{center}
\textbf{Lundi 20 Février }
\end{center}
\begin{enumerate}
\item Résoudre $\frac{1}{x+1}\leq \frac{x}{x+2}$.
\item Calculer $A^3$ où $A=\begin{pmatrix}
1&-1\\
2&1\\
\end{pmatrix}$
\item Calculer l'ensemble de définition et donner la dérivée de $$f(x) =\exp(\cos(x)-1)$$
\item Déterminer $u_n$ en fonction de $n$ où $u_0=1$ et 
$$\forall n \geq 0\, u_{n+1} =2u_n+1$$
\end{enumerate}

\begin{correction}
\begin{enumerate}

\item $\cancel{S=]-\infty, -2[ \cup ]-1,\frac{1-\sqrt{5}}{2}] \cup [\frac{1+\sqrt{5}}{2},+\infty[}$

$$S=]-\infty, -2[ \cup [-\sqrt{2},-1[ \cup [\sqrt{2},+\infty[$$
\item $A^3= \begin{pmatrix}
-5& -1\\
2&-5 
\end{pmatrix}$
\item $D_f=\R$
$f'(x) = -\sin(x)\exp(\cos(x)-1)$
\item $u_n =2^{n+1}-1$ 
\end{enumerate}
\end{correction}
 
\begin{center}
\textbf{Mardi 21 Février }
\end{center}

\begin{enumerate}
\item Résoudre $x^3+2x\leq 0$.
\item Calculer $N^2$ où $N=\begin{pmatrix}
0&1\\
0&0\\
\end{pmatrix}$. En déduire (à l'aide du binome de Newton) la valeur de $A^n$ où $A=\begin{pmatrix}
2&1\\
0&2\\
\end{pmatrix}$. 
\item Calculer l'ensemble de définition et donner la dérivée de $$f(x) =\sqrt{x^2+x+1}$$
\item Déterminer $u_n$ en fonction de $n$ où $u_0=1$ et 
$$\forall n \geq 0\, u_{n+1} =\frac{-1}{2}u_n+1$$

\item Ecrire un script Python qui permet de calculer le terme $u_n$ de la suite définie par $u_0=1$ et 
$$\forall n \in \N, \, u_{n+1} = \sin(u_n)$$
\end{enumerate}
\begin{correction}
\begin{enumerate}
\item $S=]-\infty, 0[$
\item $A^n =\begin{pmatrix}
2^n & n2^{n-1}\\
0& 2^n
\end{pmatrix}
$
\item $D_f= \R$ $f'(x) =\frac{2x+1}{2\sqrt{x^2+x+1} }$
\item $u_n=\frac{1}{3}\left(\frac{-1}{2}\right)^n  +\frac{2}{3}$
\item 
\begin{lstlisting}
import math as m
def suiteu(n):
  u=1
  for i in range(n):
    u=m.sin(u)
  return(u)
\end{lstlisting}
\end{enumerate}
\end{correction}
 

\begin{center}
\textbf{Mercredi 22 Février }
\end{center}

\begin{enumerate}
\item Résoudre $x\leq \sqrt{x+1}$.
\item Soit $A=\begin{pmatrix}
1 & 2\\
2 & 3
\end{pmatrix}$ Résoudre l'équation
$$
AX=\begin{pmatrix}
1\\
2
\end{pmatrix}$$
 d'inconnue $X=\begin{pmatrix}
x\\y
\end{pmatrix}$.
\item Calculer l'ensemble de définition et donner la dérivée de $$f(x) =\sqrt{ \ln(x) +1}$$
\item Déterminer $u_n$ en fonction de $n$ où $u_0=1, u_1=2$ et 
$$\forall n \geq 0\, u_{n+2} =5u_{n+1}-6u_n$$
\item Ecrire un script Python qui permet de calculer le terme $S_n$ de la suite définie par 
$$\forall n \in \N, \, S_{n} = \sum_{k=1}^n\sin(k)$$
\end{enumerate}

\begin{correction}
\begin{enumerate}
\item $S= ]-\infty , \frac{1+\sqrt{5}}{2}$
\item $S=\{ \begin{pmatrix}
1\\0
\end{pmatrix}\} $
\item $D_f =[e^{-1}, +\infty[$ et $\warning$ $f$ est dérivable sur $]e^{-1}, +\infty [$ $\warning$

$f'(x)= \frac{1}{2x\sqrt{\ln(x)+1} }$
\item $u_n=2^n$
\item \begin{lstlisting}
import math as m
def suiteS(n):
  s=0
  for k in range(n+1):
    s=s+m.sin(k)
  return(s)
\end{lstlisting}
\end{enumerate}
\end{correction}
 








\begin{center}
\textbf{Jeudi 23 Février }
\end{center}
\begin{enumerate}
\item Résoudre le systéme de d'inconnue $(x,y)$ et de paramétre $\lambda \in \R$
$$\left\{ \begin{array}{ccc}
x+y&=& \lambda x\\
x-y&=& \lambda y\\
\end{array}\right.$$

\item Résoudre l'équation
$$
AX=\begin{pmatrix}
1\\
2
\end{pmatrix}$$
 d'inconnue $X=\begin{pmatrix}
x\\y
\end{pmatrix}$  et où $A=\begin{pmatrix}
1 & 2\\
2 & 4
\end{pmatrix}$
\item Calculer l'ensemble de définition et donner la dérivée de $$f(x) =\frac{x}{\ln(x)-1}$$
\item Calculer $\int_1^2 xe^xdx$
\item Ecrire un fonction Python qui prend en argument une liste d'entier et retourne le maximum de cette liste. 
\end{enumerate}

\begin{correction}
\begin{enumerate}
\item  Si $\lambda \notin \{ \sqrt{2}, -\sqrt{2}\}$
$$S=\{ (0,0)\}$$

Si $\lambda =\sqrt{2}$

$$S=\{ (x,(\sqrt{2}-1)x\, | x\in \R\}$$


Si $\lambda =-\sqrt{2}$

$$S=\{ (x,(-\sqrt{2}-1)x\, | x\in \R\}$$
\item $S=\{ (1-2y, y) \, | y \in \R\}$
\item $D_f =]0,e[\cup ]e, +\infty[$

$f'(x) = \frac{\ln(x)-2 }{(\ln(x)-1)^2}$ 
\item cf TD intégration
\item 
\begin{lstlisting}
def maximum(L):
  M=L[0]
  for el in L:
    if M<el:
      M=el
  return(M)
\end{lstlisting}
\end{enumerate}
\end{correction}
 










\begin{center}
\textbf{Vendredi 24 Février }
\end{center}

\begin{enumerate}
\item Résoudre $\frac{1}{\sqrt{x+1}}\leq \sqrt{x}$.
\item Déterminer une équation cartésienne de la droite du plan passant par $A=(1,2)$ et $B=(3,4)$
\item Calculer l'ensemble de définition et donner la dérivée de $$f(x) =\frac{x^2}{x-1}$$
\item Calculer $\int_0^\pi x\cos(x)dx$
\item Ecrire un fonction Python qui prend en argument une liste d'entier et retourne le minimum de cette liste. 
\end{enumerate}

\begin{correction}
\begin{enumerate}
\item  $S=[\frac{1+\sqrt{5} }{2}, +\infty[$
\item $-x+y-1=0$
\item $D_f =\R\setminus{1}$

$f'(x) = \frac{x^2-2x }{(x-1)^2}$ 
\item cf TD intégration
\item 
\begin{lstlisting}
def minimum(L):
  m=L[0]
  for el in L:
    if m>el:
      m=el
  return(m)
\end{lstlisting}
\end{enumerate}
\end{correction}
 













\begin{center}
\textbf{Samedi 25 Février }
\end{center}
\begin{enumerate}
\item Montrer que $f $ définie par $f(x) = xe^x$ réalise une bijection entre deux intervalles de $\R$ à déterminer. 
\item Déterminer une équation cartésienne de la droite du plan passant par $A=(1,2)$ et dirigée par $\vv{u}=\begin{pmatrix}
1\\-1
\end{pmatrix}$
\item Calculer l'ensemble de définition et donner la dérivée de $$f(x) =\frac{x^2}{x-\sqrt{x}}$$
\item Calculer $\int_1^2 \frac{1}{x\sqrt{x}} 
dx$
\item Ecrire un fonction Python qui prend en argument une liste d'entier la moyenne. 
\end{enumerate}


\begin{correction}
\warning : Il y avait une erreur dans la correction du lundi 20 février sur  le premier exercice. \footnote{Merci Valentine pour la remarque} 
Résoudre $\frac{1}{x+1}\leq \frac{x}{x+2}$.

$\cancel{S=]-\infty, -2[ \cup ]-1,\frac{1-\sqrt{5}}{2}] \cup [\frac{1+\sqrt{5}}{2},+\infty[}$

$$S=]-\infty, -2[ \cup [-\sqrt{2},-1[ \cup [\sqrt{2},+\infty[$$






\begin{enumerate}
\item  $f'(x) =(1+x)e^{x}$, $f$ strictement croissnate sur $[-1, +\infty[$ $f([-1,+\infty[) =[-e^{-1} , +\infty[$

$f$ est continue, strictement croissante, $f$ réalise une bijection de $ [-1, +\infty[$ sur $[-e^{-1} , +\infty[$. 



\item $x+y-3=0$
\item $\cancel{D_f =\R_+^*}\quad$.\footnote{Merci Victor pour la remarque}


$$D_f= ]0,1[\cup ]1,+\infty[$$


$f'(x) = \frac{x}{(\sqrt{x}-1)^2}$ 
\item $I= 2-\sqrt{2}$
\item 
\begin{lstlisting}
def moyenne(L):
  moy=0
  for el in L:
      moy+=el
  return(moy/len(L))
\end{lstlisting}
\end{enumerate}
\end{correction}
 




\begin{center}
\textbf{Dimanche 26 Février }
\end{center}

\newpage
\vspace{0.3cm}
\begin{center}
DODO ! 
\end{center}

\newpage
\vspace{0.3cm}
\begin{center}
\textbf{Lundi 27 Février }
\end{center}
\begin{enumerate}
\item Résoudre $\cos(2x+\frac{\pi}{2}) =\sin(2x)$.
\item Déterminer  l'intersection des droites  $D$ et $D'$ définie par :
\begin{itemize}
\item $D$ passe par $A=(1,2) $ et $B=(3,-2)$
\item $D'$ passe par $B$ et est normale à $\vv{n}= \begin{pmatrix}
-2\\
1
\end{pmatrix}$ 
\end{itemize}
\item Déterminer la limite de $\ln(\frac{2n+1}{n^2+1})+\ln(n+3)$
\item Calculer $\ddp \int_{-\pi/4}^0 \tan(x)
dx$

\item Ecrire un fonction Python qui prend en argument une chaine de caractéres et retourne le nombre de fois où la lettre $A$ apparait. 
\end{enumerate}

\begin{correction}
\begin{enumerate}
\item $S=\{ \frac{k \pi}{2}\, |\, k\in \Z\}$
\item $(3,-2)$
\item $\ln(2)$
\item $\frac{-1}{2}\ln(2)$
\item 
\begin{lstlisting}
def  nombre_de_A(chaine):
  c=0
  for lettre in chaine:
    if lettre=='A':
      c+=1
  return c
\end{lstlisting}
\end{enumerate}

\end{correction}
\newpage


\begin{center}
\textbf{Mardi 28 Février }
\end{center}

\begin{enumerate}
\item Résoudre $e^{2x} +e^x-2\leq 0$.
\item Déterminer une équation cartésienne du plan de l'espace passant par $A=(1,2,3)$ $B=(0,1,2)$ et $C=(1,1,1)$
\item Calculer l'ensemble de définition et donner la dérivée de $$f(x) =x^x$$
\item Calculer $\ddp \int_{2}^3 \frac{x}{x^2-1}dx$
\item Ecrire un fonction Python qui prend en argument  trois entiers $a$, $b$ et $n$  et qui retourne une liste de $n$ nombres choisis aléatoirement entre $a$ et $b$ de tel sorte que les nombres soit croissant. (Il faut donc  que $n\geq b-a$ - question probablement assez difficile pour le faire bien) 
\end{enumerate}

\begin{correction}
\begin{enumerate}
\item $S=]-\infty,\ln(2)]$
\item $-x+2y-z=0$
\item $D_f=R_+^*$, $f'(x)= (\ln(x)+1)x^x$
\item $\frac{1}{2} \ln(\frac{8}{3})$
\item 

\begin{lstlisting}
import ramdom as rd
def nombre(a,b,n):
  L=[]
  while len(L)<n:
    p=rd.randint(a,b+1)
    L.append(p)
  return(L)
\end{lstlisting}
Cette fonction va probablement faire une boucle infinie si n est trop grand... 

\end{enumerate}
\end{correction}

\newpage
\begin{center}
\textbf{Mercredi 1  Mars}
\end{center}
\begin{enumerate}
\item Résoudre $\frac{\ln(x)}{\ln(x)+1}\leq \ln{(x^2)}$.
\item Déterminer  une équation cartésienne du plan de l'espace passant par $A=(1,2,3)$ et dirigé par $\vv{u} =\begin{pmatrix}
1\\2\\-1
\end{pmatrix}$ et $\vv{v} =\begin{pmatrix}
1\\2\\2
\end{pmatrix}$ 
\item Simplifier $\ddp \sum_{k=1}^n 2^n$ et $\ddp \sum_{k=1}^n 2^k$
\item Résoudre $y'+xy=2x$ avec la condition initiale $y(0)=1$
\item Ecrire une fonction qui prend deux listes correpondans aux  coordonnées de deux points du plan : $A_0=[x_0, y_0]$ et $A_1=[x_1, y_1]$ et qui retourne trois réels $(a,b,c)$ tel que $ax+by+c=0$ est une équation cartésienne de la droite $(A_0A_1)$
\end{enumerate}

\begin{correction}
\begin{enumerate}
\item $]e^{-1}, e^{-1/2}] \cup [1,+\infty[$
\item $2x-y=0$ (oui, oui, c'est une equation de plan)
\item $n2^n$ et $2^{n+1}-2$
\item $S=\{ x\mapsto -e^{-x^2} +2\}$
\item 
\begin{lstlisting}
def equation_de_droite(A0,A1):
  x0,y0=A0[0], A0[1]
  x1,y1=A1[0], A1[1]
  return( y0-y1, x1-x0, (y0-y1)x0+(x1-x0)y0)
\end{lstlisting}
(J'ai fait les calculs sur un papier et j'ai donné les réels ensuite) 
\end{enumerate}
\end{correction}
\newpage

\begin{center}
\textbf{Jeudi 2  Mars}
\end{center}
\begin{enumerate}
\item Résoudre l'équation d'inconnue $z\in \bC$ 
$$z^2+z+1=0$$
\item Déterminer le projeté orthogonale du point $A=(1,2,-1)$ sur le plan d'équation $x+y+z+1=0$
\item Simplifier $\ddp \sum_{k=1}^n \ln\left( \frac{k+1}{k}\right)$
\item Résoudre $y''+y=2x$ avec la condition initiale $y(0)=1$ et $y'(0)=0$
\end{enumerate}


\begin{correction}
\begin{enumerate}
\item $S=\{ \frac{-1+i\sqrt{3}}{2}, \frac{-1-i\sqrt{3}}{2}\}$
\item $(0,1,-2)$
\item $\ln(n+1)$
\item $S=\{ x\mapsto \cos(x)-2\sin(x)+2x\}$
\end{enumerate}
\end{correction}

\newpage
\begin{center}
\textbf{Vendredi 4  Mars}
\end{center}

\begin{enumerate}
\item Soit $A=\begin{pmatrix}
2 &1& 0\\
 0&1 &0  \\
 -1&-1&1\\
\end{pmatrix}$. Résoudre l'équation d'inconnue $\begin{pmatrix}
x\\y\\z
\end{pmatrix}$ et de paramétre $\lambda \in \R$ suivante : 
$$(A -\lambda I_3) X = 0_{3,1}$$
\item Montrer que $A$ est inversible et donner son inverse.
\end{enumerate}


\begin{correction}
\begin{enumerate}
\item Si $\lambda =1 $
$$S =\{ \begin{pmatrix}
x\\
-x\\
z
\end{pmatrix}\, (x,z)\in \R^2\}$$

Si $\lambda =2$ 
$$S =\{ \begin{pmatrix}
x\\
0\\
-x
\end{pmatrix}\, x\in \R\}$$

Si $\lambda \notin \{ 1,2\}$
$$S =\{ \begin{pmatrix}
0\\
0\\
0
\end{pmatrix}\}$$
\item (pivot de gauss)  $A^{-1}= \begin{pmatrix}
1/2 & -1/2 & 0\\
0 & 1 & 0\\
1/2 & 1/2 & 1
\end{pmatrix}$

\end{enumerate}
\end{correction}
\newpage
\begin{center}
\textbf{Samedi  5  Mars}
\end{center}

\begin{enumerate}
\item Montrer que les plans d'équation $x+y+z+1=0$ et $x-y+2z-3=0$ s'intersectent le long d'une droite.  Déterminer un vecteur directeur de cette droite. 
\item Determiner une équation paramétrique du plan d'équation $x+y+z-1=0$
\item Calculer la limite de $u_n=\frac{(n)! n^2}{n\ln(n)+e^n}$
\end{enumerate}

\begin{correction}
\begin{enumerate}
\item $\vv{u} =(-3,1,2)$ (où un vecteur proportionnel à celui là) 
\item 
$$\left\{ 
\begin{array}{ccc}
x &=& 1-\lambda -\mu\\
y &=& \lambda \\
z &=& \mu
\end{array}
\right. \quad (\lambda, \mu)\in \R^2$$

\item $+\infty$
\end{enumerate}
\end{correction}
\newpage
\begin{center}
\textbf{Dimanche  6  Mars}
\end{center}
\begin{center}
DODO ! 
\end{center}


\end{document}