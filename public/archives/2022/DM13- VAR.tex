\documentclass[a4paper, 11pt,reqno]{article}
\input{/Users/olivierglorieux/Desktop/BCPST/2020:2021/preambule.tex}


\newif\ifshow
\showtrue
\input{/Users/olivierglorieux/Desktop/BCPST/2021:2022/ifshow.tex}


\geometry{hmargin=2.0cm, vmargin=2cm}

\author{Olivier Glorieux}
\begin{document}
\title{DM 13 }


\begin{exercice}




Une urne contient une boule blanche et une boule noire, les boules \'etant
indiscernables au toucher.

On y pr\'el\`eve une boule, chaque boule ayant la m\^{e}me probabilit\'e d'%
\^{e}tre tir\'ee, on note sa couleur, et on la remet dans l'urne avec $c$
boules de la couleur de la boule tir\'ee. On r\'ep\`ete cette \'epreuve, on
r\'ealise ainsi une succession de $n$ tirages ($n\geqslant 2$).

\paragraph{\small{A - \'Etude du cas $c=0$.}\\}

On effectue donc ici $n$ tirages avec remise de la boule dans l'urne.

On note $X$ la variable al\'{e}atoire r\'{e}elle \'{e}gale au nombre de
boules blanches obtenues au cours des $n$ tirages et $Y$ la variable al\'{e}%
atoire r\'{e}elle d\'{e}finie par~: 
\begin{equation*}
\begin{cases}
Y=k & \text{si l'on obtient une boule blanche pour la premi\`{e}re fois au }%
k^{i\grave{e}me}\text{ tirage.} \\ 
Y=0 & \text{si les $n$ boules tir\'{e}es sont noires.}%
\end{cases}%
\end{equation*}

\begin{enumerate}
\item D\'{e}terminer la loi de $X$. Donner la valeur de $E(X)$ et de $V(X)$.

\item Pour $k\in \{1,\ldots ,n\}$, d\'{e}terminer la probabilit\'{e} $P(Y=k) 
$ de l'\'{e}v\'{e}nement $(Y=k)$, puis d\'{e}terminer $P(Y=0)$.

\item V\'{e}rifier que~: 
\begin{equation*}
\sum_{k=0}^{n}P(Y=k)=1.
\end{equation*}

\item Pour $x\neq 1$ et $n$ entier naturel non nul, montrer que~: 
\begin{equation*}
\sum_{k=1}^{n}kx^{k}=\frac{nx^{n+2}-(n+1)x^{n+1}+x}{(1-x)^{2}}.
\end{equation*}

\item En d\'{e}duire $E(Y)$.
\end{enumerate}

\paragraph{\small{B - \'Etude du cas $c\neq 0$.}\\}

On consid\`{e}re les variables al\'{e}atoires $\left( X_{i}\right)
_{1\leqslant i\leqslant n}$ d\'{e}finies par~: 
\begin{equation*}
\begin{cases}
X_{i}=1 & \text{si on obtient une boule blanche au }i^{\grave{e}me}\text{
tirage.} \\ 
X_{i}=0 & \text{sinon.}%
\end{cases}%
\end{equation*}%
On d\'{e}finit alors, pour $2\leqslant p\leqslant n$, la variable al\'{e}%
atoire $Z_{p}$, par~: 
\begin{equation*}
Z_{p}=\sum_{i=1}^{p}X_{i}.
\end{equation*}

\begin{enumerate}
\item Que repr\'{e}sente la variable $Z_{p}$~?

\item Donner la loi de $X_{1}$ et l'esp\'{e}rance $E(X_{1})$ de $X_{1}$.

\item D\'{e}terminer la loi du couple $(X_{1},X_{2})$. En d\'{e}duire la loi
de $X_{2}$ puis l'esp\'{e}rance $E(X_{2})$.

\item D\'{e}terminer la loi de probabilit\'{e} de $Z_{2}$.

\item D\'{e}terminer l'univers image $Z_{p}\left( \Omega \right) $ de $Z_{p} 
$.

\item Soit $p\leqslant n-1$.

\begin{enumerate}
\item D\'{e}terminer $P_{Z_{p}=k}(X_{p+1}=1)$ pour $k\in Z_{p}\left( \Omega
\right) $.

\item En utilisant la formule des probabilit\'{e}s totales, montrer que~: 
\begin{equation*}
P(X_{p+1}=1)=\frac{1+cE(Z_{p})}{2+pc}.
\end{equation*}

\item En d\'{e}duire que $X_{p}$ est une variable al\'{e}atoire de Bernoulli
de param\`{e}tre $\displaystyle\frac{1}{2}$.

(On raisonnera par r\'{e}currence sur $p$~: les variables $X_{1}$, $X_{2}$,
...., $X_{p}$ \'{e}tant suppos\'{e}es suivre une loi de de Bernoulli de param%
\`{e}tre $\displaystyle\frac{1}{2}$, et on calculera $E(Z_{p})$).
\end{enumerate}
\end{enumerate}

\end{exercice}


\begin{correction}


Une urne contient une boule blanche et une boule noire, les boules \'etant
indiscernables au toucher.

On y pr\'el\`eve une boule, chaque boule ayant la m\^{e}me probabilit\'e d'%
\^{e}tre tir\'ee, on note sa couleur, et on la remet dans l'urne avec $c$
boules de la couleur de la boule tir\'ee. On r\'ep\`ete cette \'epreuve, on
r\'ealise ainsi une succession de $n$ tirages ($n\geqslant 2$).

\paragraph{\'Etude du cas $c=0$.}

On effectue donc ici $n$ tirages avec remise de la boule dans l'urne.

On note $X$ la variable al\'{e}atoire r\'{e}elle \'{e}gale au nombre de
boules blanches obtenues au cours des $n$ tirages et $Y$ la variable al\'{e}%
atoire r\'{e}elle d\'{e}finie par~:

\begin{itemize}
\item $Y=k$ si l'on obtient une boule blanche pour la premi\`{e}re fois au $%
k^{i\grave{e}me}$ tirage.

\item $Y=0$ si les $n$ boules tir\'{e}es sont noires.
\end{itemize}

\begin{enumerate}
\item On effectue $n$ tirages ind\'{e}pendants (le contenu de l'urne ne
change pas) pour lesquels la probabilit\'{e} d'obtenir $blanc$ est toujours
1/2 (boules \'{e}quiprobables). Donc $X\hookrightarrow \mathcal{B}\left(
n,1/2\right) $ et $E\left( X\right) =n/2$ et $V\left( x\right) =n/4$

\item Pour $k\in \{1,\ldots ,n\}$, $\left( Y=k\right) $ signifie qu'on
obtient $B$ pour la premi\`{e}re fois au $k^{i\grave{e}me}$ tirage. Donc que
l'on a eu $N$ pour les tirages pr\'{e}c\'{e}dents

\begin{equation*}
\left( Y=k\right) =\bigcap_{i=1}^{k-1}N_{i}\cap B_{k}
\end{equation*}
et les tirages \'{e}tants ind\'{e}pendants, . 
\begin{equation*}
p\left( Y=k\right) =\prod_{i=1}^{k-1}p\left( N_{i}\right) \cdot p\left(
B_{k}\right) =\left( \frac{1}{2}\right) ^{k}
\end{equation*}

$\left( Y=0\right) $ signifie qu'il n'y a eu que des $N$ lors des $n$
tirages. Et donc $\displaystyle P(Y=0)=\left( \frac{1}{2}\right) ^{n}$

\item Pour calculer cette somme, il faut traiter \`{a} part la valeur $k=0$
: 
\begin{eqnarray*}
\sum_{k=0}^{n}p\left( Y=k\right) &=&\sum_{k=1}^{n}P(Y=k)+p\left( Y=0\right)
\\
&=&\sum_{k=1}^{n}\left( \frac{1}{2}\right) ^{k}+\left( \frac{1}{2}\right)
^{n}=\sum_{k=0}^{n}\left( \frac{1}{2}\right) ^{k}-\left( \frac{1}{2}\right)
^{0}+\left( \frac{1}{2}\right) ^{n} \\
&=&\frac{\left( \frac{1}{2}\right) ^{n+1}-1}{\frac{1}{2}-1}-1+\left( \frac{1%
}{2}\right) ^{n}=\frac{\left( \frac{1}{2}\right) ^{n}-1+\frac{1}{2}-\frac{1}{%
2}\left( \frac{1}{2}\right) ^{n}}{-\frac{1}{2}} \\
&=&1
\end{eqnarray*}

\item On le d\'{e}montre par r\'{e}currence : Pour $x\neq 1$

\begin{itemize}
\item Pour $n=1$ on a : 
\begin{eqnarray*}
\sum_{k=1}^{1}kx^{k} &=&x\mbox{ et } \\
\frac{1x^{1+2}-(1+1)x^{1+1}+x}{(1-x)^{2}} &=&x\frac{x^{2}-2x+1}{(1-x)^{2}}=x
\end{eqnarray*}
d'o\`{u} l'\'{e}galit\'{e}.

\item Soit $n\in \mathbb{N}^{*}$ tel que 
\begin{equation*}
\sum_{k=1}^{n}kx^{k}=\frac{nx^{n+2}-(n+1)x^{n+1}+x}{(1-x)^{2}}.
\end{equation*}

alors 
\begin{eqnarray*}
\sum_{k=1}^{n+1}kx^{k} &=&\sum_{k=1}^{n}kx^{k}+\left( n+1\right) x^{n+1} \\
&=&\left( n+1\right) x^{n+1}+\frac{nx^{n+2}-(n+1)x^{n+1}+x}{(1-x)^{2}} \\
&=&\frac{\left( n+1\right) x^{n+1}(1-x)^{2}+nx^{n+2}-(n+1)x^{n+1}+x}{%
(1-x)^{2}} \\
&=&\frac{\left( n+1\right) x^{n+1}-2\left( n+1\right) x^{n+2}+\left(
n+1\right) x^{n+3}+nx^{n+2}-(n+1)x^{n+1}+x}{(1-x)^{2}} \\
&=&\frac{\left( n+1\right) x^{n+3}+-\left( n+2\right) x^{n+2}+x}{(1-x)^{2}}
\end{eqnarray*}

Ce qu'il fallait d\'{e}montrer

\item Donc la propri\'{e}t\'{e} est vraie pour tout entier $n\ge 1$
\end{itemize}

\item On a alors 
\begin{eqnarray*}
E\left( Y\right) &=&\sum_{k=0}^{n}k\cdot p\left( Y=k\right)
=\sum_{k=1}^{n}k\cdot P(Y=k)+0\cdot p\left( Y=0\right) \\
&=&\sum_{k=1}^{n+1}k\left( \frac{1}{2}\right) ^{k}=\frac{n\left( \frac{1}{2}%
\right) ^{n+2}-(n+1)\left( \frac{1}{2}\right) ^{n+1}+\frac{1}{2}}{(1-\frac{1%
}{2})^{2}} \\
&=&4\left( n\left( \frac{1}{2}\right) ^{n+2}-(n+1)\left( \frac{1}{2}\right)
^{n+1}+\frac{1}{2}\right) \\
&=&-\left( n+2\right) \left( \frac{1}{2}\right) ^{n}+2
\end{eqnarray*}
\end{enumerate}

\paragraph{\'Etude du cas $c\neq 0$.}

On consid\`{e}re les variables al\'{e}atoires $\left( X_{i}\right)
_{1\leqslant i\leqslant n}$ d\'{e}finies par~:

\begin{itemize}
\item $X_{i}=1$ si on obtient une boule blanche au $i^{\grave{e}me}$tirage

\item $X_{i}=0$ sinon
\end{itemize}

On d\'{e}finit alors, pour $2\leqslant p\leqslant n$, la variable al\'{e}%
atoire $Z_{p}$, par~: 
\begin{equation*}
Z_{p}=\sum_{i=1}^{p}X_{i}.
\end{equation*}

\begin{enumerate}
\item $X_{i}$ compte le nombre de boule(s) balnches obtenue au $i^{\grave{e}%
me}$ tirage (uniquement). $Z_{p}$ est donc le nombre total de boules
blanches obtenues lors des $p$ premiers tirages.

\item Au premier tirages, les 2 boules sont \'{e}quiprobables. Donc $%
X_{1}\left( \Omega \right) =\left\{ 0,1\right\} $ et $p\left( X_{1}=1\right)
=p\left( X_{2}=1\right) =1/2$ et $X_{1}$ suit une loi de Bernouilli de param%
\`{e}tre $1/2.$ On a don $E\left( X\right) =1/2$ et $V\left( X\right) =1/4$

\item Il y a ici 4 probabilit\'{e}s \`{a} d\'{e}terminer en d\'{e}composant
en fonction du r\'{e}sultat de chacun des deux premiers tirages :

\begin{itemize}
\item $\left( X_{1}=0\cap X_{2}=0\right) =\left( N_{1}\cap N_{2}\right) $
donc $p\left( X_{1}=0\cap X_{2}=0\right) =p\left( N_{1}\cap N_{2}\right)
=p\left( N_{1}\right) p\left( N_{2}/N1\right) .$

Quand on a $N_{1}$ on rajoute alors $c$ boules Noires. Il y a donc $1$
blanche et c+1 noirs lors du second tirage. Ces boules \'{e}tant \'{e}%
quiprobables :

$p\left( X_{1}=0\cap X_{2}=0\right) =\displaystyle
\frac{1}{2}\cdot \frac{c+1}{c+2}$

\item De m\^{e}me $p\left( X_{1}=0\cap X_{2}=1\right) =p\left( N_{1}\cap
B_{2}\right) =p\left( N_{1}\right) p\left( B_{2}/N1\right) =\displaystyle
\frac{1}{2}\cdot \frac{1}{c+2}$

\item $p\left( X_{1}=1\cap X_{2}=0\right) =p\left( B_{1}\cap N_{2}\right)
=p\left( B_{1}\right) p\left( N_{2}/B_{1}\right) =\displaystyle
\frac{1}{2}\cdot \frac{1}{c+2}$

\item et enfin $p\left( X_{1}=1\cap X_{2}=1\right) =p\left( B_{1}\cap
B_{2}\right) =p\left( B_{1}\right) p\left( B_{2}/B1\right) =\displaystyle
\frac{1}{2}\cdot \frac{c+1}{c+2}$
\end{itemize}

La loi de $X_{2}$ est la loi marginale :

\begin{itemize}
\item $p\left( X_{2}=0\right) =p\left( X_{1}=1\cap X_{2}=0\right) +p\left(
X_{1}=0\cap X_{2}=0\right) =\displaystyle
\frac{1}{2}\cdot \frac{c+1}{c+2}+\frac{1}{2}\cdot \frac{1}{c+2}=\frac{1}{2}$

\item $p\left( X_{2}=1\right) =p\left( X_{1}=1\cap X_{2}=1\right) +p\left(
X_{1}=0\cap X_{2}=1\right) =\displaystyle
\frac{1}{2}\cdot \frac{c+1}{c+2}+\frac{1}{2}\cdot \frac{1}{c+2}=\frac{1}{2}$
\end{itemize}

La loi de $X_{2}$ est donc la m\^{e}me que celle de $X_{1}$ et $E\left(
X_{2}\right) =E\left( X_{1}\right) =1/2$

\item Ici $Z_{2}$ est la somme de deux variables al\'{e}atoires suivant des
lois binomiales de m\^{e}me param\`{e}tre de succ\`{e}s. \textbf{Mais }elles
ne sont pas ind\'{e}pendantes. On ne peut donc pas conclure que $%
Z_{2}\hookrightarrow \mathcal{B}\left( 2,1/2\right) $

\begin{itemize}
\item $Z_{2}\left( \Omega \right) =\left\{ 0,1,2\right\} $

\item $\left( Z_{2}=0\right) =\left( X_{1}=0\cap X_{2}=0\right) $ et $%
p\left( Z_{2}=0\right) =\displaystyle
\frac{1}{2}\cdot \frac{c+1}{c+2}$ (d'apr\`{e}s la loi du couple)

\item $\left( Z_{2}=1\right) =\left( X_{1}=0\cap X_{2}=1\right) \cup \left(
X_{1}=1\cap X_{2}=0\right) $ et comme ces deux parenth\`{e}ses sont
incompatibles :

$p\left( Z_{2}=1\right) =p\left( X_{1}=0\cap X_{2}=1\right) +p\left(
X_{1}=1\cap X_{2}=0\right) =\displaystyle
\frac{1}{2}\cdot \frac{1}{c+2}+\frac{1}{2}\cdot \frac{1}{c+2}=\frac{1}{c+2}$

\item $\left( Z_{2}=2\right) =\left( X_{1}=1\cap X_{2}=1\right) $ et $%
p\left( Z_{2}=2\right) =\displaystyle
\frac{1}{2}\cdot \frac{c+1}{c+2}$.
\end{itemize}

\item On peut avoir en $p$ tirages de 0 \`{a} $p$ boules blanches. Donc $%
Z_{p}\left( \Omega \right) =\left[ \left[ 0,p\right] \right] $

\item Soit $p\leqslant n-1$.

\begin{enumerate}
\item Quand $\left( Z_{p}=k\right) $ on a obtenu $k$ boules blanches et $p-k 
$ boules noires. On a donc rajout\'{e} lors de ces tirages $k\cdot c$ boules
blanches et $\left( p-k\right) c$ boules noires.

Il y a donc $k\cdot c+1$ blanches et $\left( p-k\right) c+1$ noires lors du $%
p+1^{i\grave{e}me}$ tirages$.$

Ces boules \'{e}tant \'{e}quiprobables 
\begin{equation*}
p(X_{p+1}=1\,/Z_{p}=k)=\frac{k\cdot c+1}{p\cdot c+2}
\end{equation*}

\item Les \'{e}v\'{e}nements $\left( Z_{p}=k\right) _{k\in \left[ \left[ 0,p%
\right] \right] }$ forment un syst\`{e}me complet d'\'{e}v\'{e}nements. Donc
d'apr\`{e}s la formule des probabilit\'{e}s totales :

\begin{eqnarray*}
p\left( X_{p+1}=1\right) &=&\sum_{k=0}^{p}p(X_{p+1}=1\,/Z_{p}=k)p\left(
Z_{p}=k\right) \\
&=&\sum_{k=0}^{p}\frac{k\cdot c+1}{p\cdot c+2}p\left( Z_{p}=k\right) \dots
\end{eqnarray*}

Mais on ne conna\^{\i}t pas la loi de $Z_{p}\dots $ Aussi ne fait on appara%
\^{\i}tre que son esp\'{e}rance : 
\begin{eqnarray*}
p\left( X_{p+1}=1\right) &=&\sum_{k=0}^{p}\frac{k\cdot c+1}{p\cdot c+2}%
p\left( Z_{p}=k\right) =\frac{1}{pc+2}\sum_{k=0}^{p}\left( k\cdot c+1\right)
p\left( Z_{p}=k\right) \\
&=&\frac{1}{pc+2}\left[ c\sum_{k=0}^{p}kp\left( Z_{p}=k\right)
+\sum_{k=0}^{p}p\left( Z_{p}=k\right) \right] \\
&=&\frac{1}{pc+2}\left[ cE\left( Z_{p}\right) +1\right] =\frac{cE\left(
Z_{p}\right) +1}{2+pc}
\end{eqnarray*}

\item On en d\'{e}duit par r\'{e}currence que $X_{p}$ est une variable al%
\'{e}atoire de Bernoulli de param\`{e}tre $\displaystyle \frac{1}{2}$.

\begin{itemize}
\item Pour $p=1,$ $X_{1}$ suit bien une loi de Bernouilli de param\`{e}tre $%
1/2$

\item Soit $p\ge 1$ tel que pour tout $k\in \left[ \left[ 1,p\right] \right] 
$, $X_{k}\hookrightarrow \mathcal{B}\left( 1/2\right) $

Alors $E\left( Z_{p}\right) =\sum_{k=1}^{p}E\left( X_{i}\right) =p/2$

et 
\begin{eqnarray*}
p\left( X_{p+1}=1\right) &=&\frac{cE\left( Z_{p}\right) +1}{2+pc}=\frac{%
\displaystyle \frac{cp}{2}+1}{2+pc}=\frac{cp+2}{2\left( cp+2\right) } \\
&=&\frac{1}{2}
\end{eqnarray*}
et donc $p\left( X_{p+1}=0\right) =1-p\left( X_{p}=1\right) =\frac{1}{2}$

Donc $X_{p+1}$ suit une loi binomiale de param\`{e}tre 1/2

\item Donc pour tout entier $p\geq 1:X_{p}$ suit une loi binomiale de param%
\`{e}tre 1/2.
\end{itemize}
\end{enumerate}
\end{enumerate}
\end{correction}









\end{document}