\documentclass[a4paper, 11pt,reqno]{article}
\input{/Users/olivierglorieux/Desktop/BCPST/2020:2021/preambule.tex}


\newif\ifshow
\showfalse

\input{/Users/olivierglorieux/Desktop/BCPST/2021:2022/ifshow.tex}


\geometry{hmargin=2.0cm, vmargin=2cm}

\author{Olivier Glorieux}
\begin{document}
\title{DM 14  }

\paragraph{Equation dans $\cL(E)$ }

Soit $E$ un  $\R$-espace vectoriel non réduit à son vecteur nul. On s'intéresse aux endomorphismes $f$ de $E$ vérifiant la relation 
$$f^2 =3f - 2\Id_E.\quad (*) $$
\paragraph{Un exemple}



On définit l'application : 
$$g \left| \begin{array}{ccl}
\R^2 &\tv& \R^2 \\
(x,y) &\mapsto & (3x+2y, -x)
\end{array}\right.$$
\begin{enumerate}

\item Montrer que $g$ est un endomorphisme de $\R^2$.
\item Donner la matrice de $g$, notée $A$,  dans la base canonique. 
\item Calculer $g\circ g$ et vérifier que $g$ est solution de $(*)$
\item Déterminer $F =\ker(g-\Id_{\R^2}) $ et $G =\ker(g- 2\Id_{\R^2}) $ et donner une base de $F$ et une base de $G$. 
\item Montrer que $F \cap G= \{ 0\}$ 
\item Soit $u =(1,-1) $ et  $v= (-2 , 1) $ Montrer que $B=(u,v) $ est une base de $\R^2$.\\
\textbf{Calcul de $g^n$}
\item Premiere méthode \begin{enumerate}
\item Donner la matrice de $g$, notée $D$,  dans la base $B$
\item Donner la matrice, notée $P$, de l'identité relativement aux bases $B$ au départ et la base canonique à l'arrivée. 
\item Montrer que $P$ est inversible et calculer $P^{-1} A P$. \footnote{J'ai pas encore tapé le corriger donc si vous obtenez pas $D$, essayez de calculer $PAP^{-1}$, mais je crois que c'est bon comme ca... }
\item En déduire, pour tout $n\in \N$,  $A^n$ et l'expression de $g^n$.
\end{enumerate} 
\item Deuxième Méthode

\begin{enumerate}
\item Soit $(x,y)\in \R^2$. Exprimer $(x,y) $ comme combinaison linéaire de $u$ et $v$.
\item Calculer $g^n(u)$ et $g^n(v)$. 
\item Donner finalement  l'expression de $g^n(x,y) $ en fonction de $x$ et $y$. 
\item Donner la matrice de $g^n$ dans la base canonique.

\end{enumerate}




\end{enumerate}
\paragraph{Etude de $f$}
On se place à nouveau dans le cas général et on s'intéresse à l'équation $(*)$.
\begin{enumerate}
%\item Montrer que $(*)$ possède une solution évidente. 
\item Montrer que si $f$ vérifie $(*)$ alors $f$ est bijective et exprimer $f^{-1}$ comme combinaison linéaire de $f$ et de $ \Id_E$. 
\item Déterminer les solutions de $(*)$ de la forme $\lambda \Id_E$ où $\lambda \in \R$. 
%\item L'ensemble des endomorphisme vérifiant $(*)$ est-il un sous-espace vectoriel de $\cL(E)$, espace des endomorphismes de $E$ ? 
\end{enumerate}


\paragraph{Etude des puissance de $f$}
On suppose dans la suite que $f$ est une solution de $(*)$ et que $f$ n'est pas de la forme $\lambda \Id_E$. 
\begin{enumerate}
%\item Montrer que $(f, \Id_E)$ est une famille libre de $\cL(E)$
\item \begin{enumerate}
\item Exprimer $f^3$ et $f^4$ comme combinaison linéaire de $\Id_E$ et $f$. 
\item Montrer que pour tout $n$ de $\N$, $f^n$ peut s'écrire sous la forme $f^n = a_n f +b_n \Id_E$ avec $(a_n,b_n) \in \R^2$
%\item Justifier que dans l'écriture précédente, le couple $(a_n,b_n) $ est unique.

\end{enumerate}
\item \begin{enumerate}
\item Montrer que pour tout entier $n\in \N$, $a_{n+1} -3a_n +2a_{n-1} = 0$
\item En déduire une expression de $a_n$ ne faisant intervenir que $n$. 
\item Calculer alors $b_n$.
\end{enumerate}

\end{enumerate}

\begin{correction}
\paragraph{Un exemple}
\begin{enumerate}
\item Soit $u=(x_1,y_1) $ et $v=(x_2,y_2) $ deux vecteurs de $\R^2$ et $\lambda\in \R.$
On a 
\begin{align*}
g(u+\lambda v) &= g((x_1,y_1)+\lambda (x_2,y_2)\\
						&=  g((x_1 +\lambda x_2,y_1+\lambda _2))\\
						&=(3(x_1 +\lambda x_2)+2(y_1 +\lambda y_2), -(x_1 +\lambda x_2))\\
						&=3(x_1+2y_1,-x_1) +\lambda (3x_2+2y_2,-x_2)\\
						&=g(x_1,y_1) +\lambda g(x_2,y_2)\\
						&= g(u) +\lambda g(v)
\end{align*}
Ainsi $g$ est linéaire. Comme l'espace de départ et d'arrivée de la fonction $g$ est $\R^2$,  $g$ est un endomorphisme de $\R^2$
\item On obtient $A=\begin{pmatrix}
3 & 2\\
-1 & 0
\end{pmatrix}$
\item $A^2 = \begin{pmatrix}
7& 6\\
-3 & -2 
\end{pmatrix}$
et $3A -2I_2 = \begin{pmatrix}
9 & 6\\
-3 & 0
\end{pmatrix} -\begin{pmatrix}
2 & 0\\
0 & 2
\end{pmatrix} = \begin{pmatrix}
7& 6\\
-3 & -2 
\end{pmatrix}$

Ainsi $$A^2 =3A-2I_2$$ 

L'endomorphisme $g$  vérifie donc la propriété $(*)$
\item 
$A-I_2 = \begin{pmatrix}
2 & 2\\
-1 & -1
\end{pmatrix}$
Soit $(x,y) \in F$ on a alors 
$$\begin{pmatrix}
2 & 2\\
-1 & -1
\end{pmatrix} \begin{pmatrix}
x\\
y
\end{pmatrix}=\begin{pmatrix}
0\\
0
\end{pmatrix}$$
C'est-à-dire 
$$\left\{ 
\begin{array}{ccc}
2x+2y &=&0\\
-x-y &=&0\\
\end{array}
\right.$$
On obtient donc $x=-y$ et 
$$F= \{ (-y,y) | y\in \R\} $$
Autrement dit 
$$F =\Vect((-1,1))$$

De même
$A-2I_2 = \begin{pmatrix}
1 & 2\\
-1 & -2
\end{pmatrix}$
Soit $(x,y) \in G$ on a alors 
$$\begin{pmatrix}
1 & 2\\
-1 & -2
\end{pmatrix} \begin{pmatrix}
x\\
y
\end{pmatrix}=\begin{pmatrix}
0\\
0
\end{pmatrix}$$
C'est-à-dire 
$$\left\{ 
\begin{array}{ccc}
x+2y &=&0\\
-x-2y &=&0\\
\end{array}
\right.$$
On obtient donc $x=-2y$ et 
$$G= \{ (-2y,y) | y\in \R\} $$
Autrement dit 
$$G =\Vect((-2,1))$$

\item Soit $u \in F\cap G$, 
On a alors 
$$(g-Id)(u) = (g-2Id)(u) = 0$$
D'où
$$g(u)=u \quadet g(u)=2u$$
Finalement 
$u=2u$ c'ets à dire $u=(0,0)$

Ainsi 
$$F\cap G =\{ (0,0)\}$$

\item $(u,v)$ forment une famille libre étant 2 vecteurs non propotionnels. Comme $\Card(u,v) =2 =\dim(R^2)$ c'est une base. 

\item 
\begin{enumerate}
\item On obtient 
$$D=\begin{pmatrix}
1&0\\
0& 2
\end{pmatrix}$$
\item 
Enoncé particuliérement alambiqué pour dire que l'on écrit les coordonnées de $u$ et $v$ dans la base canonique...

$Id(u) = u $ qui a pour coordoonées dans la base canonique $\begin{pmatrix}
1\\
-1
\end{pmatrix}$ donc la premiere coloone de $P$ est $\begin{pmatrix}
1\\
-1
\end{pmatrix}$. La deuxième colonne s'obtient de la meme facon avec $v$. On obtient 
$$P=\begin{pmatrix}
1 & -2\\
-1& 1
\end{pmatrix}$$
\item
$det(P) = 1-2= -1\neq 0$ donc $P$ ets inversible et la 
la formule de l'inverse d'une matrice donne 
$$P^{-1} = \begin{pmatrix}
-1 & -2\\
-1 & -1
\end{pmatrix}$$

\item Tout calcul fait, on obtient 
$$P^{-1} A P = D$$

\item On montre par récurrence que $A^n = P D^n P^{-1}$

On obtient 

$$A^n =- \begin{pmatrix}
1-2^{n+1}&2-2^{n+1}\\
1-2^{n}& -2-2^{n}
\end{pmatrix}$$

D'où 
$$g^n (x,y) = -((1-2^{n+1})x + (2-2^{n+1})y , (1-2^{n})x + (-2-2^{n+1})y )$$


\end{enumerate}
\item
\begin{enumerate}
On chercher $(a,b)\in \R^2$ tel que $au+bv=(x,y)$ 
La résolution du systéme donne 

$$a=-x-2y \quadet b= -x-y$$


\item Par récurrence on obtient $g^n(u)=u$ et $g^n (v)=2^n v$
\item 
\begin{align*}
g^n(x,y)& = g^n(au+bv)\\
				&= ag^n(u) +bg^n(v)\\
				&= (-x-2y)u +(-x-y) 2^n v\\
				&= ((-1+2^{n+1})x + (-2+2^{n+1})y , (-1+2^{n})x + (2+2^{n+1})y 
\end{align*}

\item On retrouve la matrice $A^n$ trouvée dans la question 7c




\end{enumerate}


\end{enumerate}

\begin{enumerate}
\item Si $f$ vérifie $(*)$, on a alors 

$$f\circ (f -3f) = -2Id_E$$
Ou encore 

$$f\circ \frac{-1}{2}(f -3f) =\frac{-1}{2}(f -3f)  \circ f = Id_E$$
Ainsi $f$ est bijective et $$f^{-1} =\frac{-1}{2}(f -3f) $$
\item Si $f$ est solution de $(*)$ de la forme $\lambda Id$ on a 
$$\lambda ^2 Id = 3\lambda Id -2 Id$$
Soit $$(\lambda^2 -3\lambda +2)Id =0$$
Comme $Id\neq 0$ on obtient $(\lambda^2 -3\lambda +2)=0$ dont les solutions sont $$\lambda =1 \quadet \lambda = 2$$

Ainsi les solutions de $(*)$ de la forme $\lambda Id$ sont $$f_1 = Id \quadet f_2 =2 Id$$
 \end{enumerate}
 
 \begin{enumerate}
 \item \begin{enumerate}
 \item Si $f$ vrifie $(*)$ on a alors en composant par $f$ : $$f^3= 3f^2-2f$$
 Or $f^2 =3f-2Id$ d'où
 $$f^3 = 7f -6Id$$
 
 De même on obtient $$f^4 =15f -14Id$$
 
 \item Par récurrence. 
 

 \end{enumerate}
 \item \begin{enumerate}
 \item Dans la récurrence précédente on obtient 
 $$a_{n+1}  = 3a_n+b_n$$
 et 
 $$b_{n+1} = -2a_n$$
 D'où 
 $$a_{n+2} =3a_{n+1} -2a_n$$
 C'est-à-dire 
 $$a_{n+1}-3a_n +2a_{n-1}=0$$
 \item 
C'est une suite arithmético-géométrique. On étudie le polynome caractérisque $X^2-3x+2$ de racines 
$1$ et $2$
On a donc 
$$a_n =C_1 +2^n C_2$$
où $C_1$ et $C_2$ sont deux réels à déterminer. 

On a $f^0 = Id$ d'où $a_0 =0$ et 
$f^0 = f$ d'où $a_1= 1$

En résolvant le systeme on obteint 
$C_1= -1 \quad C_2=1$
D'où 
$$a_n =-1 +2^n$$
et $$b_n = -2a_{n-1} =2 -2^{n}$$
 
 \end{enumerate}
 \end{enumerate}
\end{correction}





\end{document}