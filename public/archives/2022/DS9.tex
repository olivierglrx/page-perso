\documentclass[a4paper, 11pt,reqno]{article}
\input{/Users/olivierglorieux/Desktop/BCPST/2020:2021/preambule.tex}
\usepackage{enumitem}
\geometry{hmargin=2.5cm, vmargin=2.5cm}
\lstset{basicstyle=\ttfamily, keywordstyle=\rmfamily\bfseries}
\newif\ifshow
\showtrue

\input{/Users/olivierglorieux/Desktop/BCPST/2021:2022/ifshow.tex}

\author{Olivier Glorieux}


\begin{document}

\title{Correction - DS9}

\begin{exercice}

Pour un entier naturel non nul $n$ donné, on considère une urne contenant $2 n$ boules numérotées et indiscernables au toucher :
\begin{itemize}
\item  $n$ boules numérotées 0
\item  et les $n$ boules restantes numérotées de 1 à $n$.
\end{itemize}

On effectue au hasard et sans remise deux tirages successifs d'une boule dans cette urne. On note $X$ le plus grand des deux numéros obtenus et $Y$ le plus petit. Ce problème propose de  calculer les espérances de $X$ et $Y$.

\textbf{Partie 1 : Préliminaires. }

On note $A$ le premier numéro tiré et $B$ le deuxième.
\begin{enumerate}
\item Déterminer les univers images de $A$ et $B$.
\item Déterminer $P(A=0)$.
\item Déterminer la loi de probabilité de $A$ et en déduire que son espérance est égale à $\frac{n+1}{4}$.
\item  Justifier brièvement que $B$ a la même loi de probabilité que $A$ et en déduire son espérance.
\item  Exprimer la probabilité conditionnelle $P_{(A=a)}(B=b)$ en fonction de $n$ en distinguant plusieurs cas \footnote{on pourra distinguer 5 cas} selon les différentes valeurs de $(a, b) \in\{0,1, \ldots, n\}^2$.
\end{enumerate}

\textbf{Partie 2 : Calcul des espérances de $X$ et $Y$.}
\begin{enumerate}
\item  Déterminer l'univers image $X$.
\item Calculer la probabilité de l'événement $(X=0)$.
\item  On fixe un entier $k \in\{1,2, \ldots, n\}$ pour cette question.
\begin{enumerate}
\item Soit $E_k$ l'événement$ \{$ la premiere boule tirée vaut $k$ et la seconde boule est inférieure à $k\}$. Montrer  que 
$$P(E_k) =P(A=k) \sum_{b=0}^{k-1} P_{(A=k)}(B=b)$$

\item Soit $F_k $ l'événement $\{$ la seconde boule tirée vaut $k$ et la première boule est inférieure à $k\}$. Montrer  que 
\item En déduire que :
$$P(F_k)=\sum_{a=0}^{k-1} P(A=a) P_{(A=a)}(B=k) $$

\item En utilisant les résultats du préliminaire et les deux questions précédentes, en déduire que $P(X=k)=\frac{n+k-1}{n(2 n-1)}$.
\end{enumerate}

\item  Déduire des résultats précédents l'expression de l'espérance de $X$ en fonction de $n$. (On rappelle que 
$\ddp \sum_{k=0}^n k^2 =\frac{n(n+1)(2n+1)}{6}$)
%\item  En remarquant que $X+Y=A+B$, montrer que $E(Y)=\frac{n^2-1}{6(2 n-1)}$.
%\item  Donner un équivalent simple de $\frac{E(X)}{E(Y)}$ quand $n$ tend vers l'infini et interpréter ce résultat.
\end{enumerate}


\end{exercice}

\begin{correction}
Partie 1 
\begin{enumerate}
\item $A(\Omega) = B(\Omega) = \intent{0,n}$
\item Pour $k=0$ on obtient 
$$P(A=0)  = \frac{n}{2n}= \frac{1}{2}$$

\item $\forall  k \in \intent{1,n}$ on obtient 
$$P(A=k) = \frac{1}{2n}$$
On a alors 
\begin{align*}
E(A) &=\sum_{k=0}^n k P(A=k)\\
		&=\sum_{k=1}^n k  \frac{1}{2n}\\
		&= \frac{1}{2n} \frac{n(n+1)}{2}\\
		&=\frac{n+1}{4}
\end{align*}

\item 

\item Comme proposer dans l'énoncé on va distinguer les différents cas. 

\underline{Cas $a= 0 , b=0$}
Alors  $$P_{[A=a]}(B=b) = \frac{n-1}{2n-1}$$

\underline{Cas $a= 0 , b\neq0$}
Alors  $$P_{[A=a]}(B=b) = \frac{1}{2n-1}$$

\underline{Cas $a\neq 0 , b=0$}
Alors  $$P_{[A=a]}(B=b) = \frac{n}{2n-1}$$

\underline{Cas $a\neq 0 , b\neq 0, a=b$}
Alors  $$P_{[A=a]}(B=b) = 0$$

\underline{Cas $a\neq 0 , b\neq 0, a\neq b$}
Alors  $$P_{[A=a]}(B=b) = \frac{1}{2n-1}$$


\end{enumerate}
Partie 2 
\begin{enumerate}
\item $X(\Omega)= \intent{1,n}$
\item \begin{align*}
P(X=0)& = P(A=0 \cap B=0)\\
		&= P_{[A=0]} P(B=0) P(B=0)\\
		&= \frac{n-1}{2n-1}\frac{1}{2}
\end{align*}

\item On traduit tout d'abord l'événement $E_k$ à l'aide des VAR $A$ et $B$. On obtient :
$$E_k = [A=k] \cap \bigcup_{i=0}^{k-1} [B=i]$$
d'où 
$$P(E_k) = P( [A=k] \cap \bigcup_{i=0}^{k-1} [B=i])$$
Donc 
$$P(E_k) = P_{[A=k] }( \bigcup_{i=0}^{k-1} [B=i]))  P([A=k] )$$

Or les événements $[B=i]$ sont disjoints donc 
$$ P_{[A=k] }( \bigcup_{i=0}^{k-1} [B=i]) = \sum_{i=0}^{k-1}   P_{A=k }( B=i)$$
On obtient bien 
$$P(E_k)  =P(A=k)  \sum_{i=0}^{k-1}   P_{A=k }( B=i)$$

\item Soit $F_k $ l'événement $\{$ la seconde boule tirée vaut $k$ et la première boule est inférieure à $k\}$

On a de même 

$$F_k = [B=k] \cap \bigcup_{i=0}^{k-1} [A=i] = \bigcup_{i=0}^{k-1} [B=k] \cap [A=i]$$
d'où 
$$P(F_k) = P(\bigcup_{i=0}^{k-1} [B=k] \cap [A=i])$$
Comme les événémentes  $[B=k] \cap [A=i]$ sont disjoints on a alors 
\begin{align*}
P(F_k)&= \sum_{i=0}^{k-1} P(B=k \cap A=i)\\
		&=  \sum_{i=0}^{k-1} P_{A=i}(B=k)P ( A=i)\\
\end{align*}
\item 
$P(X=k) = P(E_k \cup F_k)$ et donc 
$$P(X=k) = P(A=k)\sum_{i=0}^{k-1} P_{A=k}(B=i)  + \sum_{a=0}^{k-1} P(A=a) P_{(A=a)}(B=k) $$
En remplacant par les valeurs obtenues au I-5) on obtient 
$$P(X=k) = \frac{1}{2n} \left(\frac{n}{2n-1}+ \sum_{i=1}^{k-1}   \frac{1}{2n-1}\right) + ( P(A=0) P_{(A=0)}(B=k)  +\sum_{a=1 }^{k-1}   P(A=a) P_{(A=a)}(B=k) $$

D'une part 
$$\frac{1}{2n} \left(\frac{n}{2n-1}+ \sum_{i=1}^{k-1}   \frac{1}{2n-1}\right)= \frac{n+k-1}{2n(2n-1)}$$

D'autre part 
\begin{align*}
 P(A=0) P_{(A=0)}(B=k)  +\sum_{a=1 }^{k-1}   P(A=a) P_{(A=a)}(B=k)  & = \frac{1}{2}\frac{1}{2n-1} + \sum_{a=1 }^{k-1} \frac{1}{2n} \frac{1}{2n-1}\\
 &= \frac{n+k-1}{2n(2n-1)}
\end{align*}
D'où 
$$P(X=k) = 2  \frac{n+k-1}{2n(2n-1)}=  \frac{n+k-1}{n(2n-1)}$$


\item 

\begin{align*}
E(X) &= \sum_{k=0}^n kP(X=k)\\
		&=\frac{1}{n(2n-1)} \sum_{k=1}^n k(n+k-1)\\
		&= \frac{1}{n(2n-1)} (n-1) \frac{n}{•}
\end{align*}

\end{enumerate}

\end{correction}
\newpage

\begin{exercice}

On définit l'application : 
$$g \left| \begin{array}{ccl}
\R^2 &\tv& \R^2 \\
(x,y) &\mapsto & (-4x+3y, -6x+5y)
\end{array}\right.$$
\begin{enumerate}

\item Montrer que $g$ est un endomorphisme de $\R^2$.
\item Donner la matrice de $g$, notée $A$,  dans la base canonique. 

\item Déterminer $F =\ker(g-2\Id_{\R^2}) $ et $G =\ker(g+\Id_{\R^2}) $ et donner une base de $F$ et une base de $G$. 
\item Montrer que $F \cap G= \{ 0\}$ 
\item Soit $u =(1,2) $ et  $v= (1 , 1) $ Montrer que $B=(u,v) $ est une base de $\R^2$.
\item Donner la matrice de $g$, notée $D$,  dans la base $B$
\item Soit $P= \begin{pmatrix}
1 & 1\\
2 & 1\\
\end{pmatrix}$
Montrer que $P$ est inversible et calculer $P^{-1} A P$. 
\item En déduire, pour tout $n\in \N$ l'expression de  $A^n$.
\end{enumerate} 
\end{exercice}


\begin{correction}
\paragraph{Un exemple}
\begin{enumerate}
\item Soit $u=(x_1,y_1) $ et $v=(x_2,y_2) $ deux vecteurs de $\R^2$ et $\lambda\in \R.$
On a 
\begin{align*}
g(u+\lambda v) &= g((x_1,y_1)+\lambda (x_2,y_2)\\
						&=  g((x_1 +\lambda x_2,y_1+\lambda _2))\\
						&=(-4(x_1 +\lambda x_2)+3(y_1 +\lambda y_2), -6(x_1 +\lambda x_2) + 5(y_1 +\lambda y_2))\\
						&=(-4x_1+3y_1,-6x_1+5y_1) +\lambda (-4x_2+3y_2,-6x_2+5y_2)\\
						&=g(x_1,y_1) +\lambda g(x_2,y_2)\\
						&= g(u) +\lambda g(v)
\end{align*}
Ainsi $g$ est linéaire. Comme l'espace de départ et d'arrivée de la fonction $g$ est $\R^2$,  $g$ est un endomorphisme de $\R^2$
\item On obtient $A=\begin{pmatrix}
-4 &3\\
-6 & 5
\end{pmatrix}$

\item 
$A-2I_2 = \begin{pmatrix}
-6 & 3\\
-6 & 3
\end{pmatrix}$
Soit $(x,y) \in F$ on a alors 
$$\begin{pmatrix}
-6 & 3\\
-6 & 3
\end{pmatrix} \begin{pmatrix}
x\\
y
\end{pmatrix}=\begin{pmatrix}
0\\
0
\end{pmatrix}$$
C'est-à-dire 
$$\left\{ 
\begin{array}{ccc}
-6x+3y &=&0\\
-6x+3y &=&0\\
\end{array}
\right.$$
On obtient donc $y=2x$ et 
$$F= \{ (x,2x) | x\in \R\} $$
Autrement dit 
$$F =\Vect((1,2))$$

De même
$A+I_2 = \begin{pmatrix}
-3 & 3\\
-6 & 6
\end{pmatrix}$
Soit $(x,y) \in G$ on a alors 
$$\begin{pmatrix}
-3 & 3\\
-6 & 6
\end{pmatrix} \begin{pmatrix}
x\\
y
\end{pmatrix}=\begin{pmatrix}
0\\
0
\end{pmatrix}$$
C'est-à-dire 
$$\left\{ 
\begin{array}{ccc}
-3x+3y &=&0\\
-6x-6y &=&0\\
\end{array}
\right.$$
On obtient donc $x=y$ et 
$$G= \{ (y,y) | y\in \R\} $$
Autrement dit 
$$G =\Vect((1,1))$$

\item Soit $u \in F\cap G$, 
On a alors 
$$(g+Id)(u) = (g-2Id)(u) = 0$$
D'où
$$g(u)=-u \quadet g(u)=2u$$
Finalement 
$-u=2u$ c'est-à-dire $u=(0,0)$

Ainsi 
$$F\cap G =\{ (0,0)\}$$

\item $(u,v)$ forment une famille libre étant 2 vecteurs non propotionnels. Comme $\Card(u,v) =2 =\dim(R^2)$ c'est une base de $\R^2$
\item On obtient 
$$D=\begin{pmatrix}
2&0\\
0& -1
\end{pmatrix}$$
\item $det(P) = 1-2= -1\neq 0$ donc $P$ est inversible et la 
la formule de l'inverse d'une matrice donne 
$$P^{-1} = \begin{pmatrix}
-1 & 1\\
2 & -1
\end{pmatrix}$$

\item Tout calcul fait, on obtient 
$$P^{-1} A =\begin{pmatrix}
-1 & 1\\
2 & -1
\end{pmatrix}  \begin{pmatrix}
-4 &3\\
-6 & 5
\end{pmatrix} = \begin{pmatrix}
-2&2\\
-2& 1
\end{pmatrix}$$


Puis $$P^{-1} A P = \begin{pmatrix}
-2&2\\
-2& 1
\end{pmatrix} \begin{pmatrix}
1 & 1\\
2 & 1\\
\end{pmatrix}= 
\begin{pmatrix}
2& 0\\
0 & -1\\
\end{pmatrix}$$
  
\item On montre par récurrence que $A^n = P D^n P^{-1}$ et  on sait d'apres le cours que 
$$D^n  =\begin{pmatrix}
2^n & 0\\
0 & (-1)^n\\
\end{pmatrix}$$
et donc $A^n = PD^n P^{-1}$ donne 
$$A^n  = $$

 \end{enumerate}
\end{correction}



\vspace{1cm}
\end{document}