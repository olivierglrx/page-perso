\documentclass[a4paper, 11pt,reqno]{article}
\input{/Users/olivierglorieux/Desktop/BCPST/2020:2021/preambule.tex}
\usepackage{enumitem}
\geometry{hmargin=2.5cm, vmargin=2.5cm}
\lstset{basicstyle=\ttfamily, keywordstyle=\rmfamily\bfseries}
\newif\ifshow
\showfalse
\input{/Users/olivierglorieux/Desktop/BCPST/2021:2022/ifshow.tex}

\author{Olivier Glorieux}


\begin{document}

\title{DS 7\\
\Large{Durée 3h00}
}




\vspace{1cm}
\begin{center}

\begin{description}
\item$\bullet$ Les calculatrices sont \underline{interdites} durant les cours, TD et \emph{a fortiori} durant les DS de mathématiques. \\

\item $\bullet $ Si vous pensez avoir découvert une erreur, indiquez-le clairement sur la copie et justifiez les initiatives que vous êtes amenés à prendre. \\

\item $\bullet$ Une grande attention sera apportée à la clarté de la rédaction et à la présentations des solutions. (Inscrivez clairement en titre le numéro de l'exercice, vous pouvez aussi encadrer les réponses finales.)  \\

\item $\bullet$ Vérifiez vos résultats. \\

\item $\bullet$ Le résultat d'une question peut être admis et utilisé pour traiter les questions suivantes en le signalant explicitement sur la copie. 
\end{description}

\end{center} 

\newpage
\begin{exercice}
Calculer les limites suivantes : 


$$1) \lim\limits_{x\to 0} \ddp\frac{(\cos{x})^2\sin{x^2}}{\sqrt{1-x^2}-1} \quad \hspace{1.5cm }2) \lim\limits_{x\to +\infty} \ddp\frac{\cos{x}}{x+1} \quad \hspace{1.5cm } 3)  \ddp \lim_{x \tv 1} \frac{\sqrt{x}-1}{ \ln(x)}$$

\end{exercice}

\vspace{1cm}


\begin{exercice}

Calculer la limite de $S_n$ quand $n\tv +\infty$ où $S_n$ est définie par 
$$S_n=\sum_{k=1}^n \frac{k+n}{k^2+n^2}$$
\end{exercice}

\vspace{1cm}




\begin{exercice}
Soit $a\in ]-1,1[. $ On suppose l'existence d'une application $f$, continue sur $\R$, telle que :
$$\forall x\in \R, \quad f(x) =\int_0^{ax} f(t)dt.$$
\begin{enumerate}
\item Calcul des dérivées successives de $f$. 
\begin{enumerate}
\item Justifier l'existence d'une primitive $F$ de $f$ sur $\R$ et écrire alors, pour tout nombre réel $x,$
$f(x)$ en fonction de $x, a$ et $F$. 
%En déduire une expression de $f(x)$  en fonction de $x, a$ et $F$. 
\item Justifier la dérivabilité de $f$ sur $\R$ et exprimer, pour tout nombre réel $x$, $f'(x)$  en fonction de $x, a$ et $f$. 
\item Démontrer que $f$ est de classe $\cC^\infty $ sur $\R$ et que pour tout nombre entier naturel $n,$ on a 
$$\forall x\in \R \quad  f^{(n)} (x) =a^{n(n+1)/2} f(a^nx).$$
\item En déduire, pour tout nombre entier naturel $n$ la valeur de $f^{(n)} (0)$. 
\end{enumerate}
\item Démontrer que, pour tout nombre réel $x$ et tout nombre entier $n$, on a :
$$f(x) = \int_0^x \frac{(x-t)^n}{n!} f^{(n+1)} (t) dt.$$
\footnotesize{ \textit{On pourra faire une récurrence et utiliser une intégration par parties}}
\normalsize{}
\item Soit $A$ un nombre réel strictement positif. 
\begin{enumerate}
\item Justifier l'existence d'un nombre réel positif ou nul $M$ tel que : 
$$\forall x\in [-A,A], \quad |f(x) | \leq M$$
et en déduire que pour tout nombre entier naturel $n$, on a :
$$\forall x\in [-A,A], \quad |f^{(n)}(x) | \leq M$$.
\item Soit $x$ un nombre réel apartenant à $[-A,A].$ Démontrer que, pour tout nombre entier naturel $n$, on  a 
$$|f(x)| \leq M\frac{A^{n+1}}{(n+1)!}.$$
\item En déduire que $f(x) = 0$ pour tout $x\in [-A,A]$
\item Que peut-on en déduire sur la fonction $f$ ? 
\end{enumerate} 
\end{enumerate}
\end{exercice}








\begin{exercice}
On dispose d'une urne contenant initialement $b$ boules blanches et $r$ boules rouges. On fait des tirages successifs dans cette urne en respectant à chaque fois le protocole suivant : 
\begin{itemize}
\item Si la boule tirée est de couleur blanche, on la remet et on ajoute une boule blanche 
\item Si la boule tirée est de couleur rouge, on la remet et on ajoute une boule rouge.  
\end{itemize}

On appelle $B_i$ l'événement "tirer une boule blanche au $i$-iéme tirage" et on note $p_i =P(B_i)$. 

\begin{enumerate}
\item Calculer $p_1$ en fonction de $b$ et $r$.
\item Montrer que $p_2= \frac{b}{b+r}  $.
\item On a tiré une boule blanche au deuxième tirage. Donner alors la probabilité que l'on ait tiré une boule blanche au premier tirage  en fonction de $b$ et $r$. 
\item On appelle $E_n$  l'événément 
\begin{center}
$E_n$ : " On tire que des boules blanches sur les $n$ premiers tirages "
\end{center}

et $F_n$ l'événement
\begin{center}
$F_n$ : " On tire  pour la première fois une boule rouge au $n$-ième tirage"
\end{center}

\begin{enumerate}
\item Exprimer $E_n$  à l'aide des événements $(B_k)_{k\in \intent{1,n}} $ 
\item Exprimer $F_n$  à l'aide de $E_{n-1}$ et $B_n$ 
\end{enumerate}


\item Pour tout $k\geq 2$ calculer $P_{E_{k-1}}(B_k)$.
\item Calculer $P(E_n)$ en fonction de $b, r$ et $n$ puis $P(F_n)$.

\item On souhaite modéliser informatiquement cette expérience. On va utiliser la lettre 'B' pour désigner les boules blanches et 'R' pour les rouges. 
\begin{enumerate}
\item Créer une fonction \texttt{urne} qui prend en paramètres le nombre de boules blanches et rouges, et retourne une liste correspondant à l'urne initiale. (Cette  liste n'a pas à être "mélangée")
%\item Créer une fonction \texttt{shuffle} qui prend en argument une liste et retourne une autre liste contenant les mêmes élèments que le première mélangés aléatoirement. 
\item Créer une fonction \texttt{tirage} qui prend en argument une liste correspondant à une urne, modélise le tirage d'une boule alétoirement dans cette urne, affiche la couleur de la boule tirée et retourne une liste correspondant à l'urne aprés l'ajout de la boule de la couleur tirée. 
\item Créer une fonction \texttt{compte} qui prend une liste correspondant à une urne et retourne   le nombre de  boules blanches  contenues dans l'urne. 
\item Créer une fonction \texttt{expérience} qui prend en argument le nombre de boules blanches et rouges et $N$ le nombre de tirages effectués et retourne le nombre de boules blanches dans l'urne aprés $N$ tirages. 
\end{enumerate}
 
\end{enumerate}
\end{exercice}

\end{document}