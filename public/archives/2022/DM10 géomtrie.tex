\documentclass[a4paper, 11pt,reqno]{article}
\input{/Users/olivierglorieux/Desktop/BCPST/2020:2021/preambule.tex}
\newif\ifshow
\showfalse
\geometry{hmargin=2cm,vmargin=2cm }
\input{/Users/olivierglorieux/Desktop/BCPST/2021:2022/ifshow.tex}
\newcommand\vv[1]{\overrightarrow{#1}}

\author{Olivier Glorieux}


\begin{document}

\title{DM 9 - Géométrie 
}
\begin{exercice}%https://math1a.bcpsthoche.fr/docs/DS_1415.pdf DS5
On munit le plan d’un repère orthonormé dont l’origine est placée en $A(0, 0).$
Soit $B, C$ deux points du plan de coordonnées respectives 
$B=(x_B,y_B)$ et $C=(x_C,y_C)$.

Soit $A'$ (respectivement $B'$ et $C'$) le milieu de $[BC]$ (respectivement $[AC]$ et $[AB]$)
\begin{enumerate}
\item Déterminer  les coordonnées de $A', B' $ et $C'$
\item Soit $G$ le point vérifiant $\vv{GA} = \frac{1}{3} (\vv{BA} +\vv{CA})$. 
\begin{enumerate}
\item Déterminer les coordonnées de $G$. 
\item Montrer que $\vv{GA}+\vv{GB}+\vv{GC}=\vv{0}$
\item Montrer que $\vv{GA} =\frac{2}{3}\vv{A'A}$
\item Pourquoi a-t-on aussi :  $\vv{GB} =\frac{2}{3}\vv{B'B}$ et  $\vv{GC} =\frac{2}{3}\vv{C'C}$ ?
\item Justifier alors que les droites $(AA')$, $(BB')$ et $(CC')$ sont concourantes en $G$. 
\end{enumerate}
\item Soit $D$ la médiatrice de $[AB]$ et $D'$ la médiatrice de $[AC]$ et $\Omega(x_\Omega,y_\Omega)$ l'intersection de $D$ et $D'$. 
\begin{enumerate}
\item Donner les équations cartésiennes des droites $D$ et $D'$ 
\item Montrer que les coordonnées de $\Omega$ vérifie 
$$M \begin{pmatrix}
x_\Omega\\
y_\Omega\\
\end{pmatrix}= \frac{1}{2}\begin{pmatrix}
x_B^2+y_B^2\\
x_C^2+y_C^2\\
\end{pmatrix}$$
\item En utilisant le fait que $\vv{AB}$ et $\vv{AC}$ ne sont pas colinéaires justifier que $M$ est inversible et donner son inverse. 
\item En déduire que 
$$ \begin{pmatrix}
x_\Omega\\
y_\Omega\\
\end{pmatrix} = \frac{1}{2\left(x_B y_C-x_C y_B\right)}\left(\begin{array}{c}
y_C\left(x_B^2+y_B^2\right)-y_B\left(x_C^2+y_C^2\right) \\
-x_C\left(x_B^2+y_B^2\right)+x_B\left(x_C^2+y_C^2\right)
\end{array}\right)$$
\item Montrer que $\vv{A' \Omega}\cdot \vv{BC}=0$ et justifier alors que 
$\Omega $ appartient à la médiatrice de $[BC]$
\end{enumerate}

\item 
 Soit $H\left(x_H, y_H\right)$ le point d'intersection de la hauteur issue de $C$ et de la hauteur issue $B$.
 \begin{enumerate}
 \item  Déterminer des représentations cartésiennes des hauteurs issues de $C$ et $B$.
 \item  Montrer que les coordonnées de $H$ vérifient l'équation
$$
M\left(\begin{array}{l}
x_H \\
y_H
\end{array}\right)=\left(\begin{array}{c}
x_B x_C+y_B y_C \\
x_B x_C+y_B y_C
\end{array}\right) .
$$
 \item En déduire les coordonnées de $H$.
 \item Montrer que $\overrightarrow{A H} \cdot \overrightarrow{B C}=0$ et justifier que $H$ appartient à la hauteur issue de $A$.
 \end{enumerate}

% \item  Montrer que $\overrightarrow{G H}$ et $\overrightarrow{G \Omega}$ sont colinéaires.
% \item Conclure  que $G, \Omega$ et $H$ sont alignés. (On pourra distinguer le cas où au moins deux des trois points $G, \Omega$ et $H$ sont confondus.)

\end{enumerate}
\end{exercice}

\end{document}