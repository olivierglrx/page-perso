\documentclass[a4paper, 11pt,reqno]{article}
\input{/Users/olivierglorieux/Desktop/BCPST/2020:2021/preambule.tex}
\newif\ifshow
\showtrue
\geometry{hmargin=2cm,vmargin=2cm }
\input{/Users/olivierglorieux/Desktop/BCPST/2021:2022/ifshow.tex}
\newcommand\vv[1]{\overrightarrow{#1}}
\usepackage{cancel}
\author{Olivier Glorieux}


\begin{document}

\title{Correction - DM 9 - Géométrie 
}
\begin{exercice}%https://math1a.bcpsthoche.fr/docs/DS_1415.pdf DS5
On munit le plan d’un repère orthonormé dont l’origine est placée en $A(0, 0).$
Soit $B, C$ deux points du plan de coordonnées respectives 
$B=(x_B,y_B)$ et $C=(x_C,y_C)$.

Soit $A'$ (respectivement $B'$ et $C'$) le milieu de $[BC]$ (respectivement $[AC]$ et $[AB]$)
\begin{enumerate}
\item Déterminer  les coordonnées de $A', B' $ et $C'$
\item Soit $G$ le point vérifiant $\vv{GA} = \frac{1}{3} (\vv{BA} +\vv{CA})$. 
\begin{enumerate}
\item Déterminer les coordonnées de $G$. 
\item Montrer que $\vv{GA}+\vv{GB}+\vv{GC}=\vv{0}$
\item Montrer que $\vv{GA} =\frac{2}{3}\vv{A'A}$
\item Pourquoi a-t-on aussi :  $\vv{GB} =\frac{2}{3}\vv{B'B}$ et  $\vv{GC} =\frac{2}{3}\vv{C'C}$ ?
\item Justifier alors que les droites $(AA')$, $(BB')$ et $(CC')$ sont concourantes en $G$. 
\end{enumerate}
\item Soit $D$ la médiatrice de $[AB]$ et $D'$ la médiatrice de $[AC]$ et $\Omega(x_\Omega,y_\Omega)$ l'intersection de $D$ et $D'$. 
\begin{enumerate}
\item Donner les équations cartésiennes des droites $D$ et $D'$ 
\item Montrer que les coordonnées de $\Omega$ vérifie 
$$M \begin{pmatrix}
x_\Omega\\
y_\Omega\\
\end{pmatrix}= \frac{1}{2}\begin{pmatrix}
x_B^2+y_B^2\\
x_C^2+y_C^2\\
\end{pmatrix}$$
où $M\in M_2(\R)$ est une matrice à déterminer.
\item En utilisant le fait que $\vv{AB}$ et $\vv{AC}$ ne sont pas colinéaires justifier que $M$ est inversible et donner son inverse. 
\item En déduire que 
$$ \begin{pmatrix}
x_\Omega\\
y_\Omega\\
\end{pmatrix} = \frac{1}{2\left(x_B y_C-x_C y_B\right)}\left(\begin{array}{c}
y_C\left(x_B^2+y_B^2\right)-y_B\left(x_C^2+y_C^2\right) \\
-x_C\left(x_B^2+y_B^2\right)+x_B\left(x_C^2+y_C^2\right)
\end{array}\right)$$
\item Montrer que $\vv{A' \Omega}\cdot \vv{BC}=0$ et justifier alors que 
$\Omega $ appartient à la médiatrice de $[BC]$
\end{enumerate}

\item 
 Soit $H\left(x_H, y_H\right)$ le point d'intersection de la hauteur issue de $C$ et de la hauteur issue $B$.
 \begin{enumerate}
 \item  Déterminer des représentations cartésiennes des hauteurs issues de $C$ et $B$.
 \item  Montrer que les coordonnées de $H$ vérifient l'équation
$$
M\left(\begin{array}{l}
x_H \\
y_H
\end{array}\right)=\left(\begin{array}{c}
x_B x_C+y_B y_C \\
x_B x_C+y_B y_C
\end{array}\right) .
$$
 \item En déduire les coordonnées de $H$.
 \item Montrer que $\overrightarrow{A H} \cdot \overrightarrow{B C}=0$ et justifier que $H$ appartient à la hauteur issue de $A$.
 \end{enumerate}

% \item  Montrer que $\overrightarrow{G H}$ et $\overrightarrow{G \Omega}$ sont colinéaires.
% \item Conclure  que $G, \Omega$ et $H$ sont alignés. (On pourra distinguer le cas où au moins deux des trois points $G, \Omega$ et $H$ sont confondus.)

\end{enumerate}
\end{exercice}


\begin{correction}
\begin{enumerate}
\item Les coordonnées de $A'$ sont $(\frac{x_B+x_C}{2} , \frac{y_B+y_C}{2})$

Les coordonnées de $B'$ sont $(\frac{x_C}{2} , \frac{y_C}{2})$

Les coordonnées de $C'$ sont $(\frac{x_B}{2} , \frac{y_B}{2})$

\item \begin{enumerate}
\item Les coordonnées de $G$ sont $ \left( \frac{x_B+x_C}{3}, \frac{y_B+y_C}{3}\right)$
\item On a $\vv{GA}=\frac{1}{3}( \vv{BA} +\vv{CA})$ donc en utilisant la relation de Chasles on obtient : 
$$\vv{GA}  =\frac{1}{3}( \vv{BG}+\vv{GA} +\vv{CG}+\vv{GA})$$
En mulitpliant par 3 et en mettant tous les termes à gauche, cela donne : 
$$\vv{GA} - \vv{BG} -\vv{CG}=\vv{0}$$
On obtient le résultat demandé en remarquant que $\vv{BG}=-\vv{GB}$ et $\vv{CG}=-\vv{GC}$


\item On peut utiliser les coordonées des différents points (on peut aussi répondre purement avec des arguments vectoriels) 
Les coordonnées du vecteur $\vv{GA}$ sont $\frac{1}{3}\begin{pmatrix}
x_B+x_C\\
y_B+y_C
\end{pmatrix}$
Et les coordonnées du vecteur $\frac{2}{3}\vv{A'A}$ sont 

$$\frac{2}{3}\begin{pmatrix}
\frac{x_B+x_C}{2} \\
\frac{y_B+y_C}{2}
\end{pmatrix} = \frac{2}{3}\begin{pmatrix}
x_B+x_C \\
y_B+y_C
\end{pmatrix} $$

\conclusion{Ainsi $\vv{GA} = \frac{2}{3}\vv{A'A}$}

\item Le probleme est strictement symétrique. Il suffit juste de changer les lettres $A$ en $B$ (ou $C$)  pour obtenir les égalités annoncées. 

\item D'aprés les deux questions précédentes :
\begin{itemize}
\item $\vv{GA}$ est colinéaire à $\vv{A'A}$, donc $ G\in (A'A)$
\item $\vv{GB}$ est colinéaire à $\vv{B'B}$, donc $G \in (B'B)$
\item $\vv{GC}$ est colinéaire à $\vv{C'C}$, donc $G \in (C'C)$
\end{itemize} 

\conclusion{Ainsi $G \in (A'A) \cap (B'B) \cap (C'C)$. }
\conclusion {Les trois médianes sont concourantes en $G$.}

\end{enumerate}
\item \begin{enumerate}
\item Les coordonnées de $\vv{AB} $ sont $\begin{pmatrix}
 \frac{x_B}{2} \\ 
 \frac{y_B}{2}
\end{pmatrix}$
Remarquons que par définition  $\vv{AB}$ est normal à $D$ donc 

Donc $D$ admet une équation de la forme $\frac{x_B}{2}
x +\frac{y_B}{2}y + c=0$ et comme $C'\in D$ on a 
$\frac{x_B}{2}\frac{x_B}{2}+\frac{y_B}{2}\frac{y_B}{2} +c=0$ soit 
$c= -(\frac{x_B^2}{4}+\frac{y_B^2}{4}$
Une équation de $D$ est donnée par  (en mutlipliant par 2)
\conclusion{$ (D)$ : $x_B x +y_B y =\frac{x_B^2+y_B^2}{2}$ }

Des calculs strictement similaires donnent une équation de $D'$ :
\conclusion{$ (D)$ : $x_C x +y_C y =\frac{x_C^2+y_C^2}{2}$ }

\item Les coordonnées de $\Omega$ vérifient les deux équations cartésiennes de $D$ et $D'$ on a donc 
$$\left\{ \begin{array}{ccc}
x_B x_\Omega +y_B y_\Omega &=&\frac{x_B^2+y_B^2}{2}\\
x_C x_\Omega +y_C y_\Omega &=&\frac{x_C^2+y_C^2}{2}
\end{array}\right.$$
Soit matriciellement : 

\conclusion{$ \begin{pmatrix}
x_B & y_B\\
x_C & y_C
\end{pmatrix}\begin{pmatrix}
x_\Omega\\
y_\Omega
\end{pmatrix}= \frac{1}{2}\begin{pmatrix}
x_B^2+y_B^2\\
x_C^2+y_C^2
\end{pmatrix}$}


Donc $M=\begin{pmatrix}
x_B & y_B\\
x_C & y_C
\end{pmatrix} $ 

\item Comme $\vv{AB}$ et $\vv{AC}$ ne sont pas colinéaires 
les couples $(x_C,y_C) $ et $(x_B,y_B)$ ne sont pas proportionnels donc $x_By_C -x_Cy_B\neq 0$. Le determinant de $M$ est non nul donc la matrice $M$ est inversible et on a 
$$M^{-1} = \frac{1}{x_By_C -x_Cy_B }\begin{pmatrix}
y_C & -y_B\\
-x_C & x_B
\end{pmatrix}$$

Finalement les coordonnées de $\Omega$ vérifient : 
\begin{align*}
\begin{pmatrix}
x_\Omega\\
y_\Omega
\end{pmatrix}&= M^{-1} \frac{1}{2}\begin{pmatrix}
x_B^2+y_B^2\\
x_C^2+y_C^2
\end{pmatrix} \\
&=\frac{1}{2(x_By_C -x_Cy_B )}\begin{pmatrix}
y_C & -y_B\\
-x_C & x_B
\end{pmatrix}\begin{pmatrix}
x_B^2+y_B^2\\
x_C^2+y_C^2
\end{pmatrix} \\
&= 
\frac{1}{2\left(x_B y_C-x_C y_B\right)}\left(\begin{array}{c}
y_C\left(x_B^2+y_B^2\right)-y_B\left(x_C^2+y_C^2\right) \\
-x_C\left(x_B^2+y_B^2\right)+x_B\left(x_C^2+y_C^2\right)
\end{array}\right)
\end{align*}


\item Les coordonnées de $\vv{A'\Omega}$ sont 

$$\vv{A'\Omega} =\frac{1}{2}\begin{pmatrix}
x_\Omega-(x_B+x_C)\\
y_\Omega -(y_B+y_C)
\end{pmatrix}$$

Les coordonnées de $\vv{BC}$ sont 
$$\begin{pmatrix}
x_C-x_B\\
y_C-y_B
\end{pmatrix}$$

Ainsi 
\begin{align*}
\vv{A'\Omega } \cdot \vv{BC} 
&=\begin{pmatrix}
x_\Omega-\frac{1}{2}(x_B+x_C)\\
y_\Omega -\frac{1}{2} (y_B+y_C)
\end{pmatrix} \cdot \begin{pmatrix}
x_C-x_B\\
y_C-y_B
\end{pmatrix}\\
&= (x_\Omega-\frac{1}{2}(x_B+x_C) )(x_C-x_B)+(y_\Omega -\frac{1}{2}(y_B+y_C) )(y_C-y_B) \\
&=x_\Omega(x_C-x_B) -\frac{1}{2} (x_C^2 -x_B^2) + y_\Omega (y_C-y_B) - \frac{1}{2}(y_C^2 -y_B^2)\\
&=x_\Omega x_C+y_\Omega y_C - ( x_\Omega x_B+y_\Omega y_B)  -\frac{1}{2} (x_C^2 -x_B^2) - \frac{1}{2}(y_C^2 -y_B^2)
\end{align*}

Remarquons maintenant que 
\begin{align*}
x_\Omega x_C +  y_\Omega y_C  &=\frac{1 }{2(x_By_C-x_Cy_B)}\left(
y_Cx_C( x_B^2 +y_B^2) -y_Bx_C (x_C^2+y_C^2)  -x_Cy_C\left(x_B^2+y_B^2\right)+x_By_C\left(x_C^2+y_C^2\right) \right)\\
&=\frac{1 }{2(x_By_C-x_Cy_B)}\left(
\cancel{y_Cx_C( x_B^2 +y_B^2)} -y_Bx_C (x_C^2+y_C^2)  -\cancel{x_Cy_C\left(x_B^2+y_B^2\right)}+x_By_C\left(x_C^2+y_C^2\right) \right)\\
&=\frac{1 }{2(x_By_C-x_Cy_B)}\left(
-y_Bx_C (x_C^2+y_C^2) +x_By_C\left(x_C^2+y_C^2\right) \right)\\
&=\frac{1 }{2(x_By_C-x_Cy_B)}\left(
(x_By_C-y_Bx_C )(x_C^2+y_C^2) \right)\\
&=\frac{1 }{2\cancel{(x_By_C-x_Cy_B)}}\left(
\cancel{(x_By_C-y_Bx_C )}(x_C^2+y_C^2) \right)\\
&=\frac{1 }{2}
(x_C^2+y_C^2)\\
\end{align*}



et de la même manière 
$$x_\Omega x_B +  y_\Omega y_B   =  \frac{1 }{2}
(x_B^2+y_B^2)$$

On obtient alors : 
\conclusion{$ \vv{A'\Omega } \cdot \vv{BC} =0$ donc les $(A'\Omega)$ est orthogonal à $(BC)$. Ainsi $\Omega$ appartient à la médiatrice de $[BC]$}


\conclusion{ Les trois médiatrices sont concourantes en $\Omega$}
\end{enumerate}

\item  Soit $\Delta$ la hauteur issue de $B$. $\Delta$ est orthogonale à $[AC]$ donc admet une équation cartésienne de la forme
$$x_C x + y_C y +c=0$$ où $c$ est un réel à déterminer. 

Comme $B\in \Delta$ on a $x_C x_B + y_C y_B +c=0$
Donc La hauteur issue de $B$ admet pour équation :
\conclusion{ $\Delta$  : $x_C x + y_C y  = x_C x_B + y_C y_B $}

Des calculs similaires montrent que la hauteur issue de $C$ admet pour équation : 
\conclusion{ $\Delta'$  : $x_B x + y_B y  = x_B x_C + y_B y_C $}

\item Les coordonnées de $H$ vérifients les équations cartésiennes des deux hauteurs prémentionnées. On a donc : 
$$\left\{ 
\begin{array}{ccc}
x_B x_H + y_B y_H  &=& x_B x_C + y_B y_C\\
x_C x_H + y_C y_H  &=& x_B x_C + y_B y_C \\

\end{array}
\right.$$
Soit matriciellement : 

\conclusion{$ M \begin{pmatrix}
x_H\\
y_H
\end{pmatrix}= \begin{pmatrix}
 x_B x_C + y_B y_C\\
 x_B x_C + y_B y_C
\end{pmatrix}$}

où $M=\begin{pmatrix}
x_B & y_B\\
x_C & y_C
\end{pmatrix}$ 

\item On a déjà calculé l'inverse de $M$ à savoir $M^{-1}= \frac{1}{x_By_C -x_Cy_B }\begin{pmatrix}
y_C & -y_B\\
-x_C & x_B
\end{pmatrix}$

On en déduit les coordonées de $H$: 

\begin{align*}
\begin{pmatrix}
x_H\\
y_H 
\end{pmatrix} &= \frac{1}{x_By_C -x_Cy_B }\begin{pmatrix}
y_C & -y_B\\
-x_C & x_B
\end{pmatrix}\begin{pmatrix}
 x_B x_C + y_B y_C \\
 x_B x_C + y_B y_C
\end{pmatrix}\\
&= \frac{1}{x_By_C -x_Cy_B }\begin{pmatrix}
(y_C-y_B) ( x_B x_C + y_B y_C)\\
(-x_C+x_B) ( x_B x_C + y_B y_C)\\
\end{pmatrix}\\
\end{align*}

\item Enfin on calcule $\vv{AH} \cdot \vv{BC}$ : 
\begin{align*}
\vv{AH} \cdot \vv{BC} &=  \begin{pmatrix}
x_H\\
y_H 
\end{pmatrix}\cdot \begin{pmatrix}
x_C-x_B\\
y_C-y_B
\end{pmatrix}\\
&= x_H(x_C-x_B) +y_H(y_C-y_B)\\
&=\frac{(y_C-y_B) ( x_B x_C + y_B y_C)(x_C-x_B)  + (-x_C+x_B) ( x_B x_C + y_B y_C)(y_C-y_B)}{x_By_C -x_Cy_B }\\
&= \frac{ ( x_B x_C + y_B y_C)(y_C-y_B)(x_C-x_B)  - ( x_B x_C + y_B y_C)(y_C-y_B) (x_C-x_B)}{x_By_C -x_Cy_B }\\
&=0
\end{align*}

\conclusion{ $(AH)$ est orthogonal à $(BC)$, donc $H$ appartient à la hauteur issue de $A$}
\conclusion{Les trois hauteurs sont concourantes en $H$ }





\end{enumerate}
\end{correction}

\end{document}