\documentclass[a4paper, 11pt,reqno]{article}
\input{/Users/olivierglorieux/Desktop/BCPST/2020:2021/preambule.tex}
\newif\ifshow
\showtrue
\input{/Users/olivierglorieux/Desktop/BCPST/2021:2022/ifshow.tex}


\geometry{hmargin=2.0cm, vmargin=2.5cm}

%\newcommand{\implique}{\Longrightarrow}
%\newcommand{\equivaut}{\Longleftrightarrow}
%\renewcommand{\ddp}{\displaystyle}
%\fancyhead[R]{TD 1 : Nombres}


\author{Olivier Glorieux}


\begin{document}
\title{DM 8
 }

 
\vspace{0.7cm}
\begin{exercice}
\begin{enumerate}
\item A quelle condition sur $X,Y\in \R$ a-t-on 
$$X=Y \Longleftrightarrow X^2=Y^2  $$

\item On se propose de résoudre l'équation :
\begin{equation}
|\cos(x)|=|\sin(x)|.\\
\end{equation}
\begin{enumerate}
\item Montrer que $(1)$ est équivalent à $\cos(2x)=0$. 
\item En déduire les solutions  dans $\R$ puis dans $[-\pi, \pi[$ 
\end{enumerate}

\end{enumerate}
\end{exercice}


\begin{correction}
\begin{enumerate}
\item On a  $X=Y \Longleftrightarrow X^2=Y^2  $ si $X$ et $Y$ sont de même signe.
\item Comme $|\cos(x)|\geq 0$ et $|\sin(x)|\geq 0$ l'équation est équivalente à $\cos^2(x) =\sin^2(x)$, soit encorrectione 
$$\cos(2x)=0.$$
On a donc $2x\equiv \frac{\pi}{2}\quad [\pi]$ ou encorrectione 
$$x\equiv \frac{\pi}{4}\quad [\frac{\pi}{2}]$$
 Les solutions sur $\R$ sont 
 $$\cS =\bigcup_{k\in Z} \{ \frac{\pi}{4}+\frac{\pi k}{2}\}$$
 Sur $[-\pi, \pi[$  les solutions sont :
 $$\cS\cap [-\pi, \pi[ =  \{ \frac{\pi}{4}, \frac{3\pi}{4}, \frac{-\pi}{4}, \frac{-3\pi}{4}\}$$
 




\end{enumerate}
\end{correction}


\begin{exercice}
Soit $z_1 $ et $z_2$ deux complexes tel que $|z_1|<1 $ et $|z_2|<1$. A l'aide d'un raisonnement par l'absurde montrer que $z_1+z_2\neq 2$
\end{exercice}

\begin{correction}
Soient $z_1$, $z_2$ deux complexes tel que  $|z_1|<1 $ et $|z_2|<1$ et supposons par l'absurde que $z_1+z_2=2$

On a alors 
$$|z_1+z_2|=|2|=2$$

D'après l'inégalité triangulaire on a :
$$|z_1+z_2|\leq |z_1|+|z_2|$$
et donc comme  $|z_1|<1 $ et $|z_2|<1$ on a  :
$$|z_1+z_2|<1+1=2$$
On obtient 
$$2<2$$ ce qui est absurde. 
La proposition est donc démontrée. 




\end{correction}

\begin{exercice}
Soit $E$ un ensemble et $A,B$ deux sous-ensembles de $E$. On appelle \emph{différence symétrique } de $A$ et $B$, notée $A\Delta B$ le sous-ensemble de $E$ définie par :
$$A \Delta  B =  (A\cap \bar{B})\cup \left(\bar{A}\cap B\right).$$
\begin{enumerate}
\item Calculer $A\Delta A$, $A\Delta \emptyset$, $A\Delta E$ et $A\Delta \overline{A}$.
\item Montrer que $A\Delta B= A$ si et seulement si $B=\emptyset.$
\item Montrer que pour tout $A,B,C$ sous-ensembles de $E$ on a :
$$(A\Delta B) \cap C = (A\cap C)\Delta (B\cap C).$$
\end{enumerate}
\end{exercice}
\begin{correction}
\begin{enumerate}
\item On obtient 
$$A\Delta A = \emptyset \quadet A\Delta \emptyset = A$$
$$A\Delta E  = \overline{A} \quadet A\Delta \overline{A} =E$$
\item Si $B=\emptyset$ on vient de voir que $A\Delta B= A$.  Montrons maintenant l'implication réciproque, on suppose donc que $A\Delta B =A$, c'est à dire :
$$ (A\cap \bar{B})\cup \left(\bar{A}\cap B\right) = A \quad (H) $$
On a donc $A \cap \left(  (A\cap \bar{B})\cup \left(\bar{A}\cap B\right)  \right) =A\cap A $
et par distributivé de l'intersection vis-à-vis de l'union : 
$$ \left( A\cap  (A\cap \bar{B})\right) \cup \left( A\cap \left(\bar{A}\cap B\right) \right)=A \quad (*) $$
Or  on a d'une part $$A\cap  (A\cap \bar{B}) = (A\cap A)\cap \bar{B}= A\cap \bar{B} $$ et  d'autre part 
$$A\cap \left(\bar{A}\cap B\right)  = (A\cap \bar{A} ) \cap B= \emptyset$$
En injectant ces deux égalités dans $(*)$ on obtient 
$$A\cap \bar{B} = A$$
D'où \conclusion{$A\subset \bar{B}$}

En revenant à l'hypothèse $(H)$ on a donc 
$$A \cup  \left(\bar{A}\cap B\right) = A$$
d'où $ \left(\bar{A}\cap B\right) \subset A$.  En prenant l'union avec $A$ et en utilisant la dsitributé de l'union on obient 
\begin{align*}
A \cup\left(\bar{A}\cap B\right) &\subset A \cup A\\
(A\cup \bar{A})\cap (A\cap B) &\subset A \\
A\cap B& \subset A
\end{align*}
Finalement on obtient 
\conclusion{$B\subset A$}

On a donc obtenu $B\subset A\subset \bar{B}$, donc 
$$B\subset \bar{B}$$
Soit en prenant l'intersection avec $B$ : 
$$B\cap B\subset B\cap \bar{B} =\emptyset$$

\conclusion{$B=\emptyset$}

\item 
On a d'une part :
\begin{align*}
(A\Delta B)\cap C &= (A\cap \bar{B})\cup \left(\bar{A}\cap B\right) \cap C \\
							&= \Big((A\cap \bar{B})\cap C\Big)  \cup \Big( \left(\bar{A}\cap B\right) \cap C \Big) \\
							&=  \Big((A\cap C)\cap  \bar{B}\Big)  \cup \Big( \bar{A}\cap (B \cap C) \Big) 
\end{align*}

On a d'autre part 
\begin{align*}
(A\cap C)\Delta (B\cap C) &=  ((A\cap C)\cap \overline{ (B\cap C)})\cup \left(\overline{(A\cap C)}\cap  (B\cap C)\right) 
\end{align*}
Simplifions la première parenthèse: 

\begin{align*}
(A\cap C)\cap \overline{ (B\cap C)} &= (A\cap C)\cap \left(\overline{ B}\cup \overline{C}\right)\\
\end{align*}
On utilise la distributivité de l'union vis-à-vis de l'intersection 
($E\cap (F\cup G) = (E\cap F)\cup   (E\cap G)$, avec $E=A\cap C,  F=\overline{ B}$ et $G=\overline{ C}$)
On obtient : 

\begin{align*}
(A\cap C)\cap \overline{ (B\cap C)} &= \Big((A\cap C)\cap \overline{ B}\Big)\cup \Big((A\cap C) \cap \overline{C}\Big) 
\end{align*}
Comme $C\cap \overline{C} = \emptyset $ on a donc 
\begin{align*}
(A\cap C)\cap \overline{ (B\cap C)} &= \Big((A\cap C)\cap \overline{ B}\Big)
\end{align*}

Le même raisonnement conduit à la simplification de la deuxième parenthèse : 
\begin{align*}
\left(\overline{(A\cap C)}\cap  (B\cap C)\right)  &=  \Big( \bar{A}\cap (B \cap C) \Big) 
\end{align*}

On obtient donc bient 
\conclusion{$(A\Delta B) \cap C = (A\cap C)\Delta (B\cap C).$} 


\end{enumerate}
\end{correction}






\end{document}