\documentclass[a4paper, 11pt,reqno]{article}
\input{/Users/olivierglorieux/Desktop/BCPST/2020:2021/preambule.tex}
\geometry{hmargin=1.5cm,vmargin=2.5cm }
\usepackage{enumitem}
\newif\ifshow
\showtrue
\input{/Users/olivierglorieux/Desktop/BCPST/2021:2022/ifshow.tex}



\author{Olivier Glorieux}


\begin{document}

\title{Correction DM1 \\
}
\begin{exercice}
Soit $f$ définie par $f(x) = \frac{x+2}{2x+1}$.\\
Exprimer $A=\{ x\in \R^+\,| \, f(x) \geq 1\}$ sous la forme d'un intervalle puis donner $\inf(A)$. \\
Faire de même avec $B=\{ f(x)\,| \, x \in\R^+\}$.

\end{exercice}


\begin{correction}
Résolvons $f(x) \geq 1$ qui donne la condition sur $x$ d'appartenance à l'ensemble $A$:
\begin{align*}
f(x) &\geq 1 \\
\frac{x+2}{2x+1}-1 &\geq 0\\
\frac{-x+1}{2x+1} &\geq 0
\end{align*}
Les solutions sont donc 
$\cS = \left]\frac{-1}{2}, 1\right]$

Donc 
\begin{center}
\fbox{$A=\R^+\cap  \left]\frac{-1}{2}, 1\right] = [0,1]$ et $\inf A =0$}
\end{center}
%$A=\R^+\cap  \left]\frac{-1}{2}, 1\right] = [0,1]$.

Etudions $f$ afin d'expliciter $B$. 
$f$ est définie et  dérivable  sur $\R\setminus\{ \frac{-1}{2}\}$ et 
$$f'(x) = \frac{2x+1 - 2(x+2)}{(2x+1)^2}= \frac{-3}{(2x+1)^2}$$
Ainsi $f$ est décroissante sur $]-\infty, \frac{-1}{2}[ $ et sur $]\frac{-1}{2}, +\infty[$. 
Par ailleurs, $f(0) = 2$ (on s'intéresse à $f(0)$  car la condition dans $B$ est $x\in \R^+$) et $\ddp \lim_{x\tv +\infty } f(x) = \frac{1}{2}$
donc

\begin{center}
\fbox{$B= \left]\frac{1}{2},2\right]$ et $\inf(B) = \frac{1}{2}$}
\end{center}


\end{correction}


%%%%%%%%%%%%
%%%%%%%%%%%%



\begin{exercice}
Simplifier au maximum
$$\frac{\left(1-\frac{y^2}{x^2}\right)^a\left(1+\frac{x}{y}\right)^a }{(x+y)^{2a}}$$
\end{exercice}


\begin{correction}
\begin{align*}
\frac{\left(1-\frac{y^2}{x^2}\right)^a\left(1+\frac{x}{y}\right)^a }{(x+y)^{2a}} &=\frac{\left(x^2-y^2\right)^a\left(y+x\right)^a }{x^{2a}y^a(x+y)^{2a}}\\
&=\frac{\left(x-y\right)^a\left(y+x\right)^a\left(y+x\right)^a }{x^{2a}y^a(x+y)^{2a}}\\
&=\frac{\left(x-y\right)^a }{x^{2a}y^a}
\end{align*}

\begin{center}
\fbox{$\frac{\left(1-\frac{y^2}{x^2}\right)^a\left(1+\frac{x}{y}\right)^a }{(x+y)^{2a}} = \frac{\left(x-y\right)^a }{x^{2a}y^a}$}
\end{center}

\end{correction}


%%%%%%%%%%%%
%%%%%%%%%%%%

\begin{exercice}
Calculer $1001^2-999^2$ (sans calculette)
\end{exercice}


\begin{correction}
$$1001^2-999^2 =  (1001-999)(1001+999) = 2 (2000)=\fbox{4000}$$
\end{correction}


%%%%%%%%%%%%
%%%%%%%%%%%%

%\begin{exercice}
%Résoudre pour $x \in \R$ l'inéquation $$\frac{1}{x+1}\leq \frac{x}{x+2}.$$
%\end{exercice}








\begin{exercice}
Résoudre pour $x \in \R$ l'inéquation $$\frac{1}{x+1}\leq \frac{x}{x+2}.$$
\end{exercice}



\begin{correction}
Le fomaine de définition est $\R\setminus\{ -1, -2\}$ . Sur ce domaine l'équation est équivalente à 
\begin{align*}
\frac{1}{x+1}-\frac{x}{x+2}&\leq 0\\
\frac{x+2-x(x+1)}{(x+1)(x+2)}&\leq 0\\
\frac{-x^2+2}{(x+1)(x+2)}&\leq 0\\
\frac{x^2-2}{(x+1)(x+2)}&\geq 0\geq\\
\frac{(x-\sqrt{2})(x+\sqrt{2})}{(x+1)(x+2)}&\geq 0
\end{align*}
Tableau de signe. 
Les solutions sont 
\begin{center}
\fbox{$\cS =]-\infty, -2[\cup [-\sqrt{2}, -1[\cup [\sqrt{2}, +\infty[$. }
\end{center}


\end{correction}

%%%%%%%%%%%%
%%%%%%%%%%%%


\begin{exercice}
Donner l'ensemble de définition de $f(x) = \sqrt{ (x^2-4)\ln\left(\frac{1}{x}\right)}$
\end{exercice}


\begin{correction}


Le logarithme est défini sur $\R^*_+$, on obtient donc comme condition 
$$\frac{1}{x}>0$$
Cette  condition équivaut à $x>0$.

La racine est définie sur $\R^+$ donc on obtient comme deuxième condition 
$$(x^2-4)\ln(\frac{1}{x}) \geq 0.$$
Ceci équivaut à $(x-2)(x+2) \ln(x)\leq 0$ et un tableau de signe donne comme solution $[1,2]$.
Ce dernier ensemble est donc l'ensemble de définition de $f$ :

\begin{center}
\fbox{$D_f =[1,2]$}
\end{center}
\end{correction}

\end{document}