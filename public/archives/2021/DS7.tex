\documentclass[a4paper, 11pt,reqno]{article}
\input{/Users/olivierglorieux/Desktop/BCPST/2020:2021/preambule.tex}
\newif\ifshow
\showfalse

\input{/Users/olivierglorieux/Desktop/BCPST/2021:2022/ifshow.tex}

\geometry{hmargin=2.0cm, vmargin=2cm}
\newenvironment{amatrix}[1]{%
  \left(\begin{array}{@{}*{#1}{c}|c@{}}
}{%
  \end{array}\right)
}

\author{Olivier Glorieux}


\begin{document}

\title{DS 7\\
\Large{Durée 3h00}
}

\vspace{1cm}
\begin{center}

\begin{description}
\item$\bullet$ Les calculatrices sont \underline{interdites} durant les cours, TD et \emph{a fortiori} durant les DS de mathématiques. \\

\item $\bullet $ Si vous pensez avoir découvert une erreur, indiquez-le clairement sur la copie et justifiez les initiatives que vous êtes amenés à prendre. \\

\item $\bullet$ Une grande attention sera apportée à la clarté de la rédaction et à la présentations des solutions. (Inscrivez clairement en titre le numéro de l'exercice, vous pouvez aussi encadrer les réponses finales.)  \\

\item $\bullet$ Vérifiez vos résultats. \\

\item $\bullet$ Le résultat d'une question peut être admis et utilisé pour traiter les questions suivantes en le signalant explicitement sur la copie. 
\end{description}

\end{center} 
\vspace{1cm}




\newpage


\begin{exercice}
\noindent

\begin{enumerate}
\item \begin{enumerate}
\item Donner un DL à l'ordre $2$ de $(1+x)^x-1$ en $0$. 
\item Donner un DL à l'ordre $2$ de $\sqrt{1-x}-\cos(x) +\frac{x}{2}$ en $0$. 
\item En déduire la limite suivante $\ddp \lim_{x\tv 0}   \frac{(1+x)^x-1}{\sqrt{1-x}-\cos(x) +\frac{x}{2}}$.
\end{enumerate}

\item Calculer la limite suivante : $\ddp \lim_{x\tv 1}\frac{\ln{x}}{x^2-1}$
\item 
\begin{enumerate}
\item Montrer que le $DL_2(0) $ de $\ddp e^{t}-e^{\frac{t}{t+1}} = t^2+o(t^2)$
\item Soit $f(x)=  x^2\left( e^{\frac{1}{x}}-e^{\frac{1}{x+1}} \right)$.
A l'aide du changment de variable $X=\frac{1}{x}$ calculer 
$$\lim_{x\tv +\infty} f(x)$$
\item Interpréter géométriquement le résultat. 
\end{enumerate}
\end{enumerate}






\end{exercice}

\begin{correction}
\begin{enumerate}
\item 
Tout d'abord on rappelle que $(1+x)^x =\exp(x\ln(1+x))$. 
Donnons ensuite les DL à l'ordre 2 en 0 des fonctions présentes dans l'expression : 
$$\ln(1+x)= x -\frac{x}{2}+o(x^2)$$
$$\exp(x) =1+x+\frac{x^2}{2}+o(x^2)$$
$$\cos(x)= 1-\frac{x^2}{2}+o(x^2)$$
$$\sqrt{1-x} = 1-\frac{x}{2}-\frac{x^2}{8} +o(x^2)$$

On a donc : 
$$\sqrt{1-x}-\cos(x) +\frac{x}{2}=  \frac{5}{8}x^2+o(x^2)$$

et 
$x\ln(1+x) =x^2 +o(x^2)$ donc 
$$(1+x)^x-1 =   x^2 +o(x^2)$$

Finalement $$\frac{(1+x)^x-1}{\sqrt{1-x}-\cos(x) +\frac{x}{2}}  = \frac{x^2+o(x^2)}{ \frac{5}{8}x^2+o(x^2)} = \frac{1+o(1)}{ \frac{5}{8}+o(1)} $$

Et donc \conclusion{$L_1 = \frac{8}{5}$.}

\item  
On pose le changement de variable $X= x-1$  ie. $x=X+1$ et on étudie la fonction $g_2(X) =\frac{\ln(X+1)}{(X+1)^2-1}$ en $0$. Calculons sa limite en $0$ à l'aide des développement limités. 

$$g_2(X) = \frac{X+o(X)}{X^2+2X} = \frac{1+o(1) }{2+X}$$

Ainsi $\lim_{X\tv 0} g_2(X) =\frac{1}{2}$ ce qui donne grâce au changment de variable : 
\conclusion{ $L_2= \frac{1}{2}$}

\item On fait le changement de variable $X=\frac{1}{x}$ et  et on étudie la fonction $g_3(X) =\frac{1}{X^2} ( e^X-e^{\frac{X}{X+1}}) $ en $0$. Calculons sa limite en $0$ à l'aide des développement limités. 

$$e^{X} =1 +X+\frac{1}{2}X^2 +o(X^2)$$
et 
$$\frac{X}{X+1} = X \frac{1}{1+X} = X (1-X+X^2 +o(X^2) = X-X^2 +o(X^2)$$
 D'où 
 \begin{align*}
 e^{\frac{X}{X+1}} &= 1 + X-X^2 +\frac{1}{2} \left(  X-X^2 \right)^2+o(X^2) \\
 							&= 1 +X -\frac{1}{2}X^2+o(X^2)
 \end{align*}

On a finalement :
\begin{align*}
e^X-e^{\frac{X}{X+1}} &= 1 +X+\frac{1}{2}X^2 - ( 1 +X -\frac{1}{2}X^2) +o(X^2) \\						&= X^2 +o(X^2) 				
\end{align*}
et donc 
$$g_3(X) =\frac{X^2+o(X^2)}{X^2} =1+o(1)$$
Le changement de variable donne 
$\lim_{x\tv +\infty} f_3(x) =\lim_{X\tv+0 } g_3(X)  =1$

\conclusion{ $L_3=1$}


\end{enumerate}

\end{correction}


\vspace{1cm}
%------------------------------------------------------------------------------------
%------------------------------------------------------------------------------------
%------------------------------------------------------------------------------------
\begin{exercice}[D'après Agro 2015]

\noindent

\begin{enumerate}
\item  Montrer que la fonction sinus réalise une bijection de $[-\pi / 2, \pi / 2]$ dans $[-1,1]$. On note alors $A$ la réciproque de la fonction 
$\left|\begin{array}{ccc}
[-\pi / 2, \pi / 2] &\rightarrow& [-1,1]\\
x &\longmapsto &\sin x.
\end{array}\right.$


\item Déterminer $A(1 / 2)$ et $A(0)$.
%\item Tracer le graphe de la fonction $A$ dans le plan usuel muni d'un repère orthonormé $(O, \vec{i}, \vec{j})$. (ou proposer un script Python qui permet de tracer cette courbe) 
\item  Soit $x$ appartenant à $[-1,1]$, montrer que $\cos (A(x))=\sqrt{1-x^{2}}$.
\item On admet que $A$ est dérivable sur $]-1,1[$. Montrer que pour tout $x\in]-1,1[$ on a :
$$A'(x) =\frac{1}{\sqrt{1-x^2}}$$
\item 
\begin{enumerate}
\item Déterminer le développement limité à l'ordre un de la fonction $t \mapsto \frac{1}{\sqrt{1+t}}$.
\item Montrer que la fonction $A$ admet un développement limité à l'ordre 3 en 0 donné par
$$
A(x)=x+\frac{x^{3}}{6}+o\left(x^{3}\right)
$$
\end{enumerate}

\end{enumerate}


\end{exercice}

\begin{correction}
\begin{enumerate}
\item  La fonction $\sin$ est continue et dérivable sur $[-\pi/2,\pi/2]$ et on a pour tout $x\in [-\pi/2,\pi/2]$, $\sin'(x) =\cos(x)$. De plus sur $]-\pi/2,\pi/2[$, $cos(x)> 0$, donc $\sin$ est strictement croissante sur $[-\pi/2,\pi/2]$. 
Le théorème de la bijeciton assure que $\sin$ réalise une bijection de $[-\pi/2,\pi/2]$  sur l'ensemble image, ici $\sin(-\pi/2) =-1$ et $\sin(\pi/2) =1$ 
donc $\sin([-\pi/2,\pi/2] ) =[-1,1]$ 
\item $A(1/2)$ est la valeur $\theta$ tel que $\sin(\theta) =\frac{1}{2}$ et $\theta \in [-\pi/2,\pi/2]$ on obtient donc 
$$A(1/2) =\frac{\pi}{6}$$

De  même $A(0)$ est la valeur $\theta$ tel que $\sin(\theta) =0$ et $\theta \in [-\pi/2,\pi/2]$ on obtient donc 
$$A(0) =0$$

\item Pour tout $u\in \R$ on a $\cos^2(u)+\sin^2(u)=1$. En appliquant cette égalité à $A(x)$ pour $x\in [-1,1]$ on obtient :

$$\cos^2(A(x)) +\sin^2(A(x))=1$$
Or par définition de $A$, $\sin(A(x))=x$ donc 
$$\cos^2(A(x)) = 1-x^2$$
Finalement comme $A(x)\in [-\pi/2,\pi/2]$, $\cos(A(x))>0$ et ainsi 
\conclusion{$\cos(A(x)) =\sqrt{1-x^2}$}

\item
\begin{enumerate}
\item $\frac{1}{\sqrt{1+t}} = (1+t)^{-1/2} = 1-\frac{1}{2} t+o(t)$ 
\item 
Le DL de $A'$ à l'ordre $2$ est donné par 
$$\frac{1}{\sqrt{1-x^2}} =1+\frac{1}{2}x^2+o(x^2)$$

En intégrant on obtient 
$$A(x) =A(0)+x+ \frac{1}{2}\frac{x^2}{3} +o(x^3)$$
Comme $A(0) =0$ on a bien
\conclusion{ $A(x) =x+\frac{x^3}{6}+o(x^3)$}

 
\end{enumerate}

\end{enumerate}
\end{correction}
\vspace{1cm}


%------------------------------------------------------------------------------------
%------------------------------------------------------------------------------------





\begin{exercice}
Roudoudou le hamster vit une vie paisible de hamster. Il a deux activités : manger et  dormir... 
On va voir Roudoudou à 00h00 ($n=0$). Il est en train de dormir. 
\begin{itemize}
\item Quand Roudoudou dort à l'heure $n$, il y a 7 chances sur 10 qu'il dorme à l'heure suivante et 3 chances sur 10 qu'il mange à l'heure suivante. 
\item Quand Roudoudou mange à l'heure $n$, il y a 2 chances sur 10 qu'il dorme à l'heure suivante et 8 chances sur 10 qu'il mange à l'heure suivante. 
\end{itemize}


On note $D_n$ l'événement 'Roudoudou dort à l'heure $n$' et $M_n$ 'Roudoudou mange à l'heure $n$'. On note $d_n =P(D_n)$ et $m_n=P(M_n)$ les probabilités respectives. 


\begin{enumerate}
\item Justifier que $d_n+m_n=1$. 
\item Montrer rigoureusement que $$d_{n+1} =  0,7d_n+0,2m_n$$
\item Exprimer de manière similaire $m_{n+1} $ en fonction de $d_n$ et $m_n$. 

\item Soit $A$ la matrice $$A=\frac{1}{10}\left(\begin{array}{ccc}
7 & 2\\
3 & 8
\end{array}
\right).$$
Résoudre en fonction de $\lambda \in \R$ l'équation $AX = \lambda X$ d'inconnue $\ddp X =\left(\begin{array}{c}
x \\
y 
\end{array}
\right)$. 
\item Soit $P = 
\left(\begin{array}{cc}
1 & 2\\	
-1 & 3
\end{array}
\right)$ Montrer que $P$ est inversible et calculer $P^{-1}$. 
\item Montrer que $P^{-1} A P =\frac{1}{5} \left(\begin{array}{cc}
 \frac{1}{2}& 0\\
0 &  1 
\end{array}
\right)$
\item Calculer $D^n$ où $D=\left(\begin{array}{cc}
 \frac{1}{2}& 0\\
0 &  1 
\end{array}
\right)$

\item En déduire que pour tout  $n\in \N$, $\ddp A^n=\left(\begin{array}{ccc}
3\left( 1/2\right)^n +2 & -2\left( 1/2\right)^n +2\\
-3\left( 1/2\right)^n +3& 2\left( 1/2\right)^n +3
\end{array}
\right)$.
\item En déduire la valeur de $d_n$ en fonction de $n$. 
\end{enumerate}
\end{exercice}



\begin{correction}
\begin{enumerate}
\item $D_n$ et $M_n$ forment un système complet d'événements donc $
d_n+m_n=1$. 
\item On cherche à calculer $d_{n+1} =P(D_{n+1})$ 
On applique la formule des probabilités totales avec le SCE $(M_N,D_N)$
\begin{align*}
d_{n+1} &= P(D_{n+1}\, |\, M_n) P(M_n) +P(D_{n+1}\, |\, D_n) P(D_n)\\
			&= P(D_{n+1}\, |\, M_n) m_n +P(D_{n+1}\, |\, D_n) d_n
\end{align*}
L'énoncé donne : $ P(D_{n+1}\, |\, M_n) = \frac{2}{10}$ et  $ P(D_{n+1}\, |\, D_n) = \frac{7}{10}$
et donc 
$$d_{n+1} = 0,7 d_n  +0,2 m_n$$

\item On cherche à calculer $m_{n+1} =P(M_{n+1})$ 
On applique la formule des probabilités totales avec le SCE $(M_N,D_N)$
\begin{align*}
m_{n+1} &= P(M_{n+1}\, |\, M_n) P(M_n) +P(M_{n+1}\, |\, D_n) P(D_n)\\
			&= P(M_{n+1}\, |\, M_n) m_n +P(M_{n+1}\, |\, D_n) d_n
\end{align*}
L'énoncé donne : $ P(M_{n+1}\, |\, M_n) = \frac{8}{10}$ et  $ P(M_{n+1}\, |\, D_n) = \frac{3}{10}$
et donc 
$$m_{n+1} = 0,3 d_n  +0,8 m_n$$

\item 
On obtient le système d'équations
$$\left\{  
\begin{array}{cc}
7x +2y  &=10\lambda x\\
3x +8y  &=10\lambda y
\end{array}\right.$$



$
\equivaut
\left\{  
\begin{array}{cc}
(7-10\lambda) x +2y  &=0\\
3x +(8-10\lambda)y  &=0
\end{array}\right.
\equivaut 
\left\{  
\begin{array}{cc}
3x +(8-10\lambda)y  &=0\\
(7-10\lambda) x +2y  &=0
\end{array}\right.$

$L_2 \leftarrow3*L_2- (7-10\lambda)L_1$

$
\equivaut 
\left\{  
\begin{array}{cc}
3x +(8-10\lambda)y  &=0\\
(-100\lambda^2 +150\lambda -50 )y  &=0
\end{array}\right.
\equivaut
\left\{  
\begin{array}{cc}
(7-10\lambda) x +2y  &=0\\
(2\lambda^2 -3\lambda +1) y  &=0
\end{array}\right.
\equivaut
\left\{  
\begin{array}{cc}
(7-10\lambda) x +2y  &=0\\
(2\lambda-1)(\lambda-1) y  &=0
\end{array}\right.
$

Le système est de Cramer pour $(2\lambda-1)(\lambda-1)\neq 0$ et l'unique solution est alors $(0,0)$. 

Pour $\lambda=1$ on obtient 
$\equivaut
\left\{  
\begin{array}{cc}
-3 x +2y  &=0\\
0 &=0
\end{array}\right.
$
et les solutions sont de la forme : 
$$\{ (2a,3a ) \, |\, a\in \R\} $$

Pour $\lambda=\frac{1}{2}$ on obtient 
$\equivaut
\left\{  
\begin{array}{cc}
2 x +2y  &=0\\
0 &=0
\end{array}\right.
$
et les solutions sont de la forme : 
$$\{ (a,-a ) \, |\, a\in \R\} $$

\item Le determinant de $P$ vaut $det(P) = 3+2 = 5 \neq 0$ donc $P$ est inversible. 
Son inverse vaut 
$$P^{-1} = \frac{1}{5} \left( 
\begin{array}{cc}
3 & -2 \\
1 & 1
\end{array}
\right)$$

\item Ce n'est que du calcul. 

\item $$D^n =  \left( 
\begin{array}{cc}
\frac{1}{2^n}& 0 \\
0 & 1
\end{array}
\right)$$
A prouver par récurrence ou  dire que c'est du cours pour des matrices diagonales. 
\item 
On prouve tout d'abord par récurrence que pour tout n :
$Q(n) : " A^n = P D^n P^{-1} "$.
Initialisation. La proposition est vraie pour $n=0$ les deux cotés valent l'identité. 

On suppose $Q(n) $ vraie pour un $n \in \N$ fixé. On a 
$A^{n} =   P D^n P^{-1}$ et donc
\begin{align*}
A^{n+1} &=A P D^n P^{-1}\\
&=PDP^{-1} P D^n P^{-1} \\
&= PD \Id D^n P^{-1}  \\
&=PD  D^n P^{-1}\\
&=PD^{n+1} P^{-1}
\end{align*}

Ensuite c'est du calcul. 
\item 
Et d'après les questions 2 et 3 on  a
$$A \binom{d_n}{m_n}= \binom{d_{n+1}}{m_{n+1}}$$
et par récurrence 
$$A^n \binom{d_0}{m_0}= \binom{d_n}{m_n}$$
D'après l'énoncé $d_0= 1$ c'est l'événement certain. 
et donc $$ \binom{d_n}{m_n}= 
\frac{1}{5} \binom{3\left( 1/2\right)^n +2 }{-3\left( 1/2\right)^n +3}$$
En particulier 

$$d_n=\frac{1}{5} (3\left( 1/2\right)^n +2) $$

\end{enumerate}
\end{correction}




%%%%------------------------------------------------------------------
%------------------------------------------------------------------
%------------------------------------------------------------------
%------------------------------------------------------------------

\vspace{1cm}

\begin{exercice}
On dispose d'une urne contenant initialement $b$ boules blanches et $r$ boules rouges. On fait des tirages successifs dans cette urne en respectant à chaque fois le protocole suivant : 
\begin{itemize}
\item Si la boule tirée est de couleur blanche, on la remet et on ajoute une boule blanche 
\item Si la boule tirée est de couleur rouge, on la remet et on ajoute une boule rouge.  
\end{itemize}

On appelle $B_i$ l'événement "tirer une boule blanche au $i$-iéme tirage" et on note $p_i =P(B_i)$. 

\begin{enumerate}
\item Calculer $p_1$ en fonction de $b$ et $r$.
\item Montrer que $p_2= \frac{b}{b+r}  $.
\item On a tiré une boule blanche au deuxième tirage. Donner alors la probabilité que l'on ait tiré une boule blanche au premier tirage  en fonction de $b$ et $r$. 
\item On appelle $E_n$  l'événément 
\begin{center}
$E_n$ : " On tire que des boules blanches sur les $n$ premiers tirages "
\end{center}

et $F_n$ l'événement
\begin{center}
$F_n$ : " On tire  pour la première fois une boule rouge au $n$-ième tirage"
\end{center}

\begin{enumerate}
\item Exprimer $E_n$  à l'aide des événements $(B_k)_{k\in \intent{1,n}} $ 
\item Exprimer $F_n$  à l'aide de $E_{n-1}$ et $B_n$ 
\end{enumerate}


\item Pour tout $k\geq 2$ calculer $P_{E_{k-1}}(B_k)$.
\item Calculer $P(E_n)$ en fonction de $b, r$ et $n$ puis $P(F_n)$.

\item On souhaite modéliser informatiquement cette expérience. On va utiliser la lettre 'B' pour désigner les boules blanches et 'R' pour les rouges. 
\begin{enumerate}
\item Créer une fonction \texttt{urne} qui prend en argument le nombre de boules blanches et rouges, et retourne une liste correspondant à l'urne initiale. (Cette  liste n'a pas à être "mélangée")
%\item Créer une fonction \texttt{shuffle} qui prend en argument une liste et retourne une autre liste contenant les mêmes élèments que le première mélangés aléatoirement. 
\item Créer une fonction \texttt{tirage} qui prend en argument une liste correspondant à une urne, modélise le tirage d'une boule alétoirement dans cette urne, affiche la couleur de la boule tirée et retourne une liste correspondant à l'urne aprés l'ajout de la boule de la couleur tirée. 
\item Créer une fonction \texttt{compte} qui prend une liste correspondant à une urne et retourne   le nombre de  boules blanches  contenues dans l'urne. 
\item Créer une fonction \texttt{expérience} qui prend en argument le nombre de boules blanches et rouges initial et $N$ le nombre de tirages effectués et retourne le nombre de boules blanches dans l'urne aprés $N$ tirages. 
\end{enumerate}
 
\end{enumerate}
\begin{correction}
\begin{enumerate}


\item On a $p_1=P(B_1)$. Comme il y a $b$ boules et $b+r$ boules en tout, on en déduit que $P(B_1) =\frac{b}{b+r}$ 
\conclusion{$p_1=\frac{b}{b+r}$}

\item On utilise le système complet d'événements $(B_1, \overline{B_1})$, la formule des probabilités totales donnent : 
$$p_2=P(B_2) =P(B_2|B_1) P(B_1) +P(B_2 |\overline{B_1})P(\overline{B_1})$$

Si on a tiré une boule blanche au tirage 1, il y a $b+1$ boules blanches dans l'urne et $r$ boules rouges, donc 
$$P(B_2|B_1)  =\frac{b+1}{b+r+1}$$

De même, si on a tiré une boule rouge au tirage 1, il y a $b$ boules blanches dans l'urne et $r+1$ boules rouges, donc 
$$P(B_2|\overline{B_1})  =\frac{b}{b+r+1}$$

D'après le calcul de $p_1$, on sait que $P(\overline{B_1})= 1-P(B_1) =1-\frac{b}{b+r}= \frac{r}{b+r}$

Ainsi 
\begin{align*}
p_2&=\frac{b+1}{b+r+1} \frac{b}{b+r} + \frac{b}{b+r+1} \frac{r}{b+r}\\
	&=\frac{(b+1)b}{(b+r+1)(b+r)} + \frac{br}{(b+r+1)(b+r)} \\
	&=\frac{(b+1+r)b}{(b+r+1)(b+r)}  \\
	&=\frac{b}{b+r}  
\end{align*}

\conclusion{ $p_2= \frac{b}{b+r}  $}

\item On cherche à calculer $P(B_1|B_2)$ et on utilise pour cela la formule de Bayes :
$$P(B_1|B_2) =\frac{P(B_2|B_1) P(B_1)}{P(B_2)}$$
Or on a vu que $P(B_1) =P(B_2)$ donc 
$$P(B_1|B_2) =P(B_2|B_1)$$

Ainsi \conclusion{ $P(B_1|B_2) = \frac{b+1}{b+r+1}$}
\item \begin{enumerate}
\item $E_n = B_1 \cap B_2 \cap \dots \cap B_{n} $. 
\item $F_n =E_{n-1} \cap \overline{B_n}$. 
\end{enumerate}
\item Si l'événement $E_{k-1} $ est réalisé, on a tiré que des boules blanches sur les $k-1$ premiers tirages. Il y a donc $b+k-1$ boules blanches et $b+k-1+r$ boules au total. 
Donc \conclusion{ $P(B_k |E_{k-1}) = \frac{b+k-1}{b+k-1+r}$} 

\item On utilise la formule des probabilités conditionnelles et on obtient : 
$$P(E_n) = P(B_1) P(B_2|B_1) P(B_3|B_2\cap B_1) \dots  P(B_{n}|B_{n-1}\cap \dots \cap B_2\cap B_1) $$
Remarquons que les termes du produit sont de la forme $P(B_k |E_{k-1})$ que l'on a calculé à la question précédente. On a donc 
\begin{align*}
P(E_n) &=P(B_1) \prod_{k=2}^{n} P(B_k |E_{k-1})\\
			&=\frac{b}{b+r}\frac{b+2-1}{b+2-1+r} \frac{b+3-1}{b+2-1+r}\dots \frac{b+n-1}{b+n-1+r}\\
			&=\frac{b}{b+r}\frac{b+1}{b+1+r} \frac{b+2}{b+2+r}  \dots \frac{b+n-1}{b+n-1+r}\\
			&= \frac{(b+n-1)!}{(b-1)!} \frac{(b+r-1)!}{(b+n-1+r)!} 
\end{align*}


\item On en déduit la valeur de $P(F_n) $ de nouveau en utilisant la formule des proababilités conditionnelles : 
$$P(F_n) =P(E_{n-1}\cap \overline{B_n}) =P(E_{n-1} ) P(\overline{B_n}|E_{n-1})$$

Si l'événement $E_{n-1}$ est réalisé, il y a  $r$ boules rouges et $b+n-1+r$ boules au total. 
Donc 
$$P(F_n) = \frac{r}{b+n-1+r}$$

\item \begin{enumerate}
\item
\begin{lstlisting}
def urne(b,r):
  L=['B' for i in range(b)] +['R' for i in range(n)]
  return(L)
\end{lstlisting}

\item
\begin{lstlisting}
from random import randint
def tirage(L):
	nouvel_urne=L[:]
	k=randint(0,len(L)-1)
	
	if L[k]=='B':
	  print('Boule blanche')
	  nouvel_urne=nouvel_urne+['B']
	else:
	  print('Boule Rouge')
	  nouvel_urne=nouvel_urne+['R']
	return(nouvel_urne)
\end{lstlisting}

\item 
\begin{lstlisting}

def compte(L):
  b=0
  for e in L:
    if e=='B':
      b=b+1
  return(b)
\end{lstlisting}

\item 
\begin{lstlisting}
def experience(b,r,N):
  U=urne(b,r)
  for k in range(N):
    U=tirage(U)
  boule_b=compte(U)
  return(boule_b)
    
\end{lstlisting}

\end{enumerate}

\end{enumerate}

\end{correction}


\end{exercice}





\end{document}