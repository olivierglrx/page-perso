\documentclass[a4paper, 11pt,reqno]{article}
\input{/Users/olivierglorieux/Desktop/BCPST/2020:2021/preambule.tex}
\newif\ifshow
\showtrue
\usepackage{eurosym}
\input{/Users/olivierglorieux/Desktop/BCPST/2021:2022/ifshow.tex}

\geometry{hmargin=2.0cm, vmargin=2cm}
\newenvironment{amatrix}[1]{%
  \left(\begin{array}{@{}*{#1}{c}|c@{}}
}{%
  \end{array}\right)
}

\author{Olivier Glorieux}


\begin{document}

\title{Correction : DS 9 
}


%\subsection{Probabilité, VAR, tirage boules et urnes. (ECRICOME 2002) }
\begin{exercice}




Une urne contient une boule blanche et une boule noire, les boules \'etant
indiscernables au toucher. On y pr\'el\`eve une boule, chaque boule ayant la m\^{e}me probabilit\'e d'%
\^{e}tre tir\'ee, on note sa couleur, et on la remet dans l'urne avec $c$
boules de la couleur de la boule tir\'ee. On r\'ep\`ete cette \'epreuve, on
r\'ealise ainsi une succession de $n$ tirages ($n\geqslant 2$).

\paragraph{\small{A - \'Etude du cas $c=0$.}\\}
Dans cette partie et dans cette partie seulement, on suppose que $c=0$. C'est-à-dire que l'on  effectue  ici $n$ tirages avec remise.

On note $X$ la variable al\'{e}atoire r\'{e}elle \'{e}gale au nombre de
boules blanches obtenues au cours des $n$ tirages et $Y$ la variable al\'{e}%
atoire r\'{e}elle d\'{e}finie par~: 
\begin{equation*}
\begin{cases}
Y=k & \text{si l'on obtient une boule blanche pour la premi\`{e}re fois au }%
k^{i\grave{e}me}\text{ tirage.} \\ 
Y=0 & \text{si les $n$ boules tir\'{e}es sont noires.}%
\end{cases}%
\end{equation*}

\begin{enumerate}
\item D\'{e}terminer la loi de $X$. Donner la valeur de $E(X)$ et de $V(X)$.

\item 
\begin{enumerate}
\item Soit $N_i$ l'événement : " tirer une boule blanche au tirage $i$ ". Pour $k\in \intent{1,n}$, exprimer l'événément $Y=k$ en fonction de $(N_i)_{i\in \intent{1,k}}$
\item En déduire la pour  $k\in \intent{1,n}$, la probabilit\'{e} $P(Y=k) $
\item Calculer enfin $P(Y=0)$
\end{enumerate}


%\item V\'{e}rifier que~: 
%\begin{equation*}
%\sum_{k=0}^{n}P(Y=k)=1.
%\end{equation*}
\item \begin{enumerate}
\item Soit $x\neq 1$ et $n\in \N$  Rappeler la valeur de $\ddp \sum_{k=0}^{n+1} x^k $.
 \item En dérivant l'expression obtenue à la question précédente, montrer que : 
\begin{equation*}
\sum_{k=1}^{n}kx^{k}=\frac{nx^{n+2}-(n+1)x^{n+1}+x}{(1-x)^{2}}.
\end{equation*}
\end{enumerate}


\item En d\'{e}duire $E(Y)$.
\end{enumerate}

\paragraph{\small{B - \'Etude du cas $c\neq 0$.}\\}
On revient au cas général et on suppose donc que l'on remet à chaque tirage $c$ boules de la couleur tirée. 
On consid\`{e}re les variables al\'{e}atoires $\left( X_{i}\right)
_{1\leqslant i\leqslant n}$ d\'{e}finies par~: 
\begin{equation*}
\begin{cases}
X_{i}=1 & \text{si on obtient une boule blanche au }i^{\grave{e}me}\text{
tirage.} \\ 
X_{i}=0 & \text{sinon.}%
\end{cases}%
\end{equation*}%
On d\'{e}finit alors, pour $2\leqslant p\leqslant n$, la variable al\'{e}%
atoire $Z_{p}$, par~: 
\begin{equation*}
Z_{p}=\sum_{i=1}^{p}X_{i}.
\end{equation*}

\begin{enumerate}
\item Que repr\'{e}sente la variable $Z_{p}$~?

\item Donner la loi de $X_{1}$ et l'esp\'{e}rance $E(X_{1})$ de $X_{1}$.

\item D\'{e}terminer l'univers image de $X_2$ et pour tout $(x,y)\in X_1(\Omega) \times X_2(\Omega)$ la valeur de $P(X_1=x \text{ et } X_2 =y)$. 

\item  En d\'{e}duire la loi de $X_{2}$ puis l'esp\'{e}rance $E(X_{2})$.

\item D\'{e}terminer la loi de probabilit\'{e} de $Z_{2}$.

\item D\'{e}terminer l'univers image $Z_{p}\left( \Omega \right) $ de $Z_{p} 
$.

\item Soit $p\leqslant n-1$.

\begin{enumerate}
\item D\'{e}terminer $P_{Z_{p}=k}(X_{p+1}=1)$ pour $k\in Z_{p}\left( \Omega
\right) $.

\item En utilisant la formule des probabilit\'{e}s totales, montrer que~: 
\begin{equation*}
P(X_{p+1}=1)=\frac{1+cE(Z_{p})}{2+pc}.
\end{equation*}

\item En d\'{e}duire que $X_{p}$ est une variable al\'{e}atoire de Bernoulli
de param\`{e}tre $\displaystyle\frac{1}{2}$.

(On raisonnera par r\'{e}currence forte sur $p$ : les variables $X_{1}$, $X_{2}$,
...., $X_{p}$ \'{e}tant suppos\'{e}es suivre une loi de de Bernoulli de param%
\`{e}tre $\displaystyle\frac{1}{2}$, et on calculera $E(Z_{p})$).
\end{enumerate}
\end{enumerate}

\end{exercice}


\begin{correction}


Une urne contient une boule blanche et une boule noire, les boules \'etant
indiscernables au toucher.

On y pr\'el\`eve une boule, chaque boule ayant la m\^{e}me probabilit\'e d'%
\^{e}tre tir\'ee, on note sa couleur, et on la remet dans l'urne avec $c$
boules de la couleur de la boule tir\'ee. On r\'ep\`ete cette \'epreuve, on
r\'ealise ainsi une succession de $n$ tirages ($n\geqslant 2$).

\paragraph{\'Etude du cas $c=0$.}

On effectue donc ici $n$ tirages avec remise de la boule dans l'urne.

On note $X$ la variable al\'{e}atoire r\'{e}elle \'{e}gale au nombre de
boules blanches obtenues au cours des $n$ tirages et $Y$ la variable al\'{e}%
atoire r\'{e}elle d\'{e}finie par~:

\begin{itemize}
\item $Y=k$ si l'on obtient une boule blanche pour la premi\`{e}re fois au $%
k^{i\grave{e}me}$ tirage.

\item $Y=0$ si les $n$ boules tir\'{e}es sont noires.
\end{itemize}

\begin{enumerate}
\item On effectue $n$ tirages ind\'{e}pendants (le contenu de l'urne ne
change pas) pour lesquels la probabilit\'{e} d'obtenir $blanc$ est toujours
1/2 (boules \'{e}quiprobables). Donc $X\hookrightarrow \mathcal{B}\left(
n,1/2\right) $ et $E\left( X\right) =n/2$ et $V\left( x\right) =n/4$

\item Pour $k\in \{1,\ldots ,n\}$, $\left( Y=k\right) $ signifie qu'on
obtient $B$ pour la premi\`{e}re fois au $k^{i\grave{e}me}$ tirage. Donc que
l'on a eu $N$ pour les tirages pr\'{e}c\'{e}dents

\begin{equation*}
\left( Y=k\right) =\bigcap_{i=1}^{k-1}N_{i}\cap B_{k}
\end{equation*}
et les tirages \'{e}tants ind\'{e}pendants, . 
\begin{equation*}
p\left( Y=k\right) =\prod_{i=1}^{k-1}p\left( N_{i}\right) \cdot p\left(
B_{k}\right) =\left( \frac{1}{2}\right) ^{k}
\end{equation*}

$\left( Y=0\right) $ signifie qu'il n'y a eu que des $N$ lors des $n$
tirages. Et donc $\displaystyle P(Y=0)=\left( \frac{1}{2}\right) ^{n}$

\item Pour calculer cette somme, il faut traiter \`{a} part la valeur $k=0$
: 
\begin{eqnarray*}
\sum_{k=0}^{n}p\left( Y=k\right) &=&\sum_{k=1}^{n}P(Y=k)+p\left( Y=0\right)
\\
&=&\sum_{k=1}^{n}\left( \frac{1}{2}\right) ^{k}+\left( \frac{1}{2}\right)
^{n}=\sum_{k=0}^{n}\left( \frac{1}{2}\right) ^{k}-\left( \frac{1}{2}\right)
^{0}+\left( \frac{1}{2}\right) ^{n} \\
&=&\frac{\left( \frac{1}{2}\right) ^{n+1}-1}{\frac{1}{2}-1}-1+\left( \frac{1%
}{2}\right) ^{n}=\frac{\left( \frac{1}{2}\right) ^{n}-1+\frac{1}{2}-\frac{1}{%
2}\left( \frac{1}{2}\right) ^{n}}{-\frac{1}{2}} \\
&=&1
\end{eqnarray*}

\item On le d\'{e}montre par r\'{e}currence : Pour $x\neq 1$

\begin{itemize}
\item Pour $n=1$ on a : 
\begin{eqnarray*}
\sum_{k=1}^{1}kx^{k} &=&x\mbox{ et } \\
\frac{1x^{1+2}-(1+1)x^{1+1}+x}{(1-x)^{2}} &=&x\frac{x^{2}-2x+1}{(1-x)^{2}}=x
\end{eqnarray*}
d'o\`{u} l'\'{e}galit\'{e}.

\item Soit $n\in \mathbb{N}^{*}$ tel que 
\begin{equation*}
\sum_{k=1}^{n}kx^{k}=\frac{nx^{n+2}-(n+1)x^{n+1}+x}{(1-x)^{2}}.
\end{equation*}

alors 
\begin{eqnarray*}
\sum_{k=1}^{n+1}kx^{k} &=&\sum_{k=1}^{n}kx^{k}+\left( n+1\right) x^{n+1} \\
&=&\left( n+1\right) x^{n+1}+\frac{nx^{n+2}-(n+1)x^{n+1}+x}{(1-x)^{2}} \\
&=&\frac{\left( n+1\right) x^{n+1}(1-x)^{2}+nx^{n+2}-(n+1)x^{n+1}+x}{%
(1-x)^{2}} \\
&=&\frac{\left( n+1\right) x^{n+1}-2\left( n+1\right) x^{n+2}+\left(
n+1\right) x^{n+3}+nx^{n+2}-(n+1)x^{n+1}+x}{(1-x)^{2}} \\
&=&\frac{\left( n+1\right) x^{n+3}+-\left( n+2\right) x^{n+2}+x}{(1-x)^{2}}
\end{eqnarray*}

Ce qu'il fallait d\'{e}montrer

\item Donc la propri\'{e}t\'{e} est vraie pour tout entier $n\ge 1$
\end{itemize}

\item On a alors 
\begin{eqnarray*}
E\left( Y\right) &=&\sum_{k=0}^{n}k\cdot p\left( Y=k\right)
=\sum_{k=1}^{n}k\cdot P(Y=k)+0\cdot p\left( Y=0\right) \\
&=&\sum_{k=1}^{n+1}k\left( \frac{1}{2}\right) ^{k}=\frac{n\left( \frac{1}{2}%
\right) ^{n+2}-(n+1)\left( \frac{1}{2}\right) ^{n+1}+\frac{1}{2}}{(1-\frac{1%
}{2})^{2}} \\
&=&4\left( n\left( \frac{1}{2}\right) ^{n+2}-(n+1)\left( \frac{1}{2}\right)
^{n+1}+\frac{1}{2}\right) \\
&=&-\left( n+2\right) \left( \frac{1}{2}\right) ^{n}+2
\end{eqnarray*}
\end{enumerate}

\paragraph{\'Etude du cas $c\neq 0$.}

On consid\`{e}re les variables al\'{e}atoires $\left( X_{i}\right)
_{1\leqslant i\leqslant n}$ d\'{e}finies par~:

\begin{itemize}
\item $X_{i}=1$ si on obtient une boule blanche au $i^{\grave{e}me}$tirage

\item $X_{i}=0$ sinon
\end{itemize}

On d\'{e}finit alors, pour $2\leqslant p\leqslant n$, la variable al\'{e}%
atoire $Z_{p}$, par~: 
\begin{equation*}
Z_{p}=\sum_{i=1}^{p}X_{i}.
\end{equation*}

\begin{enumerate}
\item $X_{i}$ compte le nombre de boule(s) balnches obtenue au $i^{\grave{e}%
me}$ tirage (uniquement). $Z_{p}$ est donc le nombre total de boules
blanches obtenues lors des $p$ premiers tirages.

\item Au premier tirages, les 2 boules sont \'{e}quiprobables. Donc $%
X_{1}\left( \Omega \right) =\left\{ 0,1\right\} $ et $p\left( X_{1}=1\right)
=p\left( X_{2}=1\right) =1/2$ et $X_{1}$ suit une loi de Bernouilli de param%
\`{e}tre $1/2.$ On a don $E\left( X\right) =1/2$ et $V\left( X\right) =1/4$

\item Il y a ici 4 probabilit\'{e}s \`{a} d\'{e}terminer en d\'{e}composant
en fonction du r\'{e}sultat de chacun des deux premiers tirages :

\begin{itemize}
\item $\left( X_{1}=0\cap X_{2}=0\right) =\left( N_{1}\cap N_{2}\right) $
donc $p\left( X_{1}=0\cap X_{2}=0\right) =p\left( N_{1}\cap N_{2}\right)
=p\left( N_{1}\right) p\left( N_{2}/N1\right) .$

Quand on a $N_{1}$ on rajoute alors $c$ boules Noires. Il y a donc $1$
blanche et c+1 noirs lors du second tirage. Ces boules \'{e}tant \'{e}%
quiprobables :

$p\left( X_{1}=0\cap X_{2}=0\right) =\displaystyle
\frac{1}{2}\cdot \frac{c+1}{c+2}$

\item De m\^{e}me $p\left( X_{1}=0\cap X_{2}=1\right) =p\left( N_{1}\cap
B_{2}\right) =p\left( N_{1}\right) p\left( B_{2}/N1\right) =\displaystyle
\frac{1}{2}\cdot \frac{1}{c+2}$

\item $p\left( X_{1}=1\cap X_{2}=0\right) =p\left( B_{1}\cap N_{2}\right)
=p\left( B_{1}\right) p\left( N_{2}/B_{1}\right) =\displaystyle
\frac{1}{2}\cdot \frac{1}{c+2}$

\item et enfin $p\left( X_{1}=1\cap X_{2}=1\right) =p\left( B_{1}\cap
B_{2}\right) =p\left( B_{1}\right) p\left( B_{2}/B1\right) =\displaystyle
\frac{1}{2}\cdot \frac{c+1}{c+2}$
\end{itemize}

La loi de $X_{2}$ est la loi marginale :

\begin{itemize}
\item $p\left( X_{2}=0\right) =p\left( X_{1}=1\cap X_{2}=0\right) +p\left(
X_{1}=0\cap X_{2}=0\right) =\displaystyle
\frac{1}{2}\cdot \frac{c+1}{c+2}+\frac{1}{2}\cdot \frac{1}{c+2}=\frac{1}{2}$

\item $p\left( X_{2}=1\right) =p\left( X_{1}=1\cap X_{2}=1\right) +p\left(
X_{1}=0\cap X_{2}=1\right) =\displaystyle
\frac{1}{2}\cdot \frac{c+1}{c+2}+\frac{1}{2}\cdot \frac{1}{c+2}=\frac{1}{2}$
\end{itemize}

La loi de $X_{2}$ est donc la m\^{e}me que celle de $X_{1}$ et $E\left(
X_{2}\right) =E\left( X_{1}\right) =1/2$

\item Ici $Z_{2}$ est la somme de deux variables al\'{e}atoires suivant des
lois binomiales de m\^{e}me param\`{e}tre de succ\`{e}s. \textbf{Mais }elles
ne sont pas ind\'{e}pendantes. On ne peut donc pas conclure que $%
Z_{2}\hookrightarrow \mathcal{B}\left( 2,1/2\right) $

\begin{itemize}
\item $Z_{2}\left( \Omega \right) =\left\{ 0,1,2\right\} $

\item $\left( Z_{2}=0\right) =\left( X_{1}=0\cap X_{2}=0\right) $ et $%
p\left( Z_{2}=0\right) =\displaystyle
\frac{1}{2}\cdot \frac{c+1}{c+2}$ (d'apr\`{e}s la loi du couple)

\item $\left( Z_{2}=1\right) =\left( X_{1}=0\cap X_{2}=1\right) \cup \left(
X_{1}=1\cap X_{2}=0\right) $ et comme ces deux parenth\`{e}ses sont
incompatibles :

$p\left( Z_{2}=1\right) =p\left( X_{1}=0\cap X_{2}=1\right) +p\left(
X_{1}=1\cap X_{2}=0\right) =\displaystyle
\frac{1}{2}\cdot \frac{1}{c+2}+\frac{1}{2}\cdot \frac{1}{c+2}=\frac{1}{c+2}$

\item $\left( Z_{2}=2\right) =\left( X_{1}=1\cap X_{2}=1\right) $ et $%
p\left( Z_{2}=2\right) =\displaystyle
\frac{1}{2}\cdot \frac{c+1}{c+2}$.
\end{itemize}

\item On peut avoir en $p$ tirages de 0 \`{a} $p$ boules blanches. Donc $%
Z_{p}\left( \Omega \right) =\left[ \left[ 0,p\right] \right] $

\item Soit $p\leqslant n-1$.

\begin{enumerate}
\item Quand $\left( Z_{p}=k\right) $ on a obtenu $k$ boules blanches et $p-k 
$ boules noires. On a donc rajout\'{e} lors de ces tirages $k\cdot c$ boules
blanches et $\left( p-k\right) c$ boules noires.

Il y a donc $k\cdot c+1$ blanches et $\left( p-k\right) c+1$ noires lors du $%
p+1^{i\grave{e}me}$ tirages$.$

Ces boules \'{e}tant \'{e}quiprobables 
\begin{equation*}
p(X_{p+1}=1\,/Z_{p}=k)=\frac{k\cdot c+1}{p\cdot c+2}
\end{equation*}

\item Les \'{e}v\'{e}nements $\left( Z_{p}=k\right) _{k\in \left[ \left[ 0,p%
\right] \right] }$ forment un syst\`{e}me complet d'\'{e}v\'{e}nements. Donc
d'apr\`{e}s la formule des probabilit\'{e}s totales :

\begin{eqnarray*}
p\left( X_{p+1}=1\right) &=&\sum_{k=0}^{p}p(X_{p+1}=1\,/Z_{p}=k)p\left(
Z_{p}=k\right) \\
&=&\sum_{k=0}^{p}\frac{k\cdot c+1}{p\cdot c+2}p\left( Z_{p}=k\right) \dots
\end{eqnarray*}

Mais on ne conna\^{\i}t pas la loi de $Z_{p}\dots $ Aussi ne fait on appara%
\^{\i}tre que son esp\'{e}rance : 
\begin{eqnarray*}
p\left( X_{p+1}=1\right) &=&\sum_{k=0}^{p}\frac{k\cdot c+1}{p\cdot c+2}%
p\left( Z_{p}=k\right) =\frac{1}{pc+2}\sum_{k=0}^{p}\left( k\cdot c+1\right)
p\left( Z_{p}=k\right) \\
&=&\frac{1}{pc+2}\left[ c\sum_{k=0}^{p}kp\left( Z_{p}=k\right)
+\sum_{k=0}^{p}p\left( Z_{p}=k\right) \right] \\
&=&\frac{1}{pc+2}\left[ cE\left( Z_{p}\right) +1\right] =\frac{cE\left(
Z_{p}\right) +1}{2+pc}
\end{eqnarray*}

\item On en d\'{e}duit par r\'{e}currence que $X_{p}$ est une variable al%
\'{e}atoire de Bernoulli de param\`{e}tre $\displaystyle \frac{1}{2}$.

\begin{itemize}
\item Pour $p=1,$ $X_{1}$ suit bien une loi de Bernouilli de param\`{e}tre $%
1/2$

\item Soit $p\ge 1$ tel que pour tout $k\in \left[ \left[ 1,p\right] \right] 
$, $X_{k}\hookrightarrow \mathcal{B}\left( 1/2\right) $

Alors $E\left( Z_{p}\right) =\sum_{k=1}^{p}E\left( X_{i}\right) =p/2$

et 
\begin{eqnarray*}
p\left( X_{p+1}=1\right) &=&\frac{cE\left( Z_{p}\right) +1}{2+pc}=\frac{%
\displaystyle \frac{cp}{2}+1}{2+pc}=\frac{cp+2}{2\left( cp+2\right) } \\
&=&\frac{1}{2}
\end{eqnarray*}
et donc $p\left( X_{p+1}=0\right) =1-p\left( X_{p}=1\right) =\frac{1}{2}$

Donc $X_{p+1}$ suit une loi binomiale de param\`{e}tre 1/2

\item Donc pour tout entier $p\geq 1:X_{p}$ suit une loi binomiale de param%
\`{e}tre 1/2.
\end{itemize}
\end{enumerate}
\end{enumerate}
\end{correction}




%\subsection{Sujet Révisions Algébre linéaire - (Pb ) Corrigé non écrit !  }
\vspace{2cm}
\begin{exercice}


Soit $E$ un  $\R$-espace vectoriel non réduit à son vecteur nul. On s'intéresse aux endomorphismes $f$ de $E$ vérifiant la relation 
$$f^2 =3f - 2\Id_E.\quad (*) $$
\paragraph{A- Etude d'un exemple}



On définit l'application : 
$$g \left| \begin{array}{ccl}
\R^2 &\tv& \R^2 \\
(x,y) &\mapsto & (3x+2y, -x)
\end{array}\right.$$
\begin{enumerate}
\item Montrer que $g$ est un endomorphisme de $\R^2$.
\item Calculer $g\circ g$ et vérifier que $g$ est solution de $(*)$
\item Soit $F =\ker(g-\Id_{\R^2}) $. Montrer que $F$ est un sous-espace vectoriel de $\R^2$ et  en  donner une base.
\item Faire de même avec $G =\ker(g- 2\Id_{\R^2}) $. 
\item Montrer que $F \cap G= \{ 0\}$ 
\item Soit $u =(1,-1) $ et  $v= (-2 , 1) $ Montrer que $B=(u,v) $ est une base de $\R^2$.
\item Soit $(x,y)\in \R^2$. Exprimer $(x,y) $ comme combinaison linéaire de $u$ et $v$.
\item Calculer $g^n(u)$ et $g^n(v)$. 
\item Donner finalement  l'expression de $g^n(x,y) $ en fonction de $x$ et $y$. 
\item Soit $A$ la matrice de $g$ dans la base canonique. Donner la matrice $A$, puis à l'aide de la question précédente déterminer  $A^n$ en fonction de $n$. 
\end{enumerate}
\paragraph{B- Cas général}
On se place maitenant dans le cas général et on s'intéresse à l'équation $(*)$.
\begin{enumerate}
%\item Montrer que $(*)$ possède une solution évidente. 
\item Montrer que si $f$ vérifie $(*)$ alors $f$ est bijective et exprimer $f^{-1}$ comme combinaison linéaire de $f$ et de $ \Id_E$. 
\item Déterminer les solutions de $(*)$ de la forme $\lambda \Id_E$ où $\lambda \in \R$. 
%\item L'ensemble des endomorphisme vérifiant $(*)$ est-il un sous-espace vectoriel de $\cL(E)$, espace des endomorphismes de $E$ ? 



%\paragraph{Etude des puissance de $f$}
On suppose dans la suite que $f$ est une solution de $(*)$ et que $f$ n'est pas de la forme $\lambda \Id_E$. 

%\item Montrer que $(f, \Id_E)$ est une famille libre de $\cL(E)$
\item \begin{enumerate}
\item Exprimer $f^3$ comme combinaison linéaire de $\Id_E$ et $f$. 
\item Montrer que pour tout $n$ de $\N$, $f^n$ peut s'écrire sous la forme $f^n = a_n f +b_n \Id_E$ avec $(a_n,b_n) \in \R^2$
%\item Justifier que dans l'écriture précédente, le couple $(a_n,b_n) $ est unique.

\end{enumerate}
\item \begin{enumerate}
\item Montrer que pour tout entier $n\in \N$, $a_{n+1} -3a_n +2a_{n-1} = 0$
\item En déduire une expression de $a_n$ ne faisant intervenir que $n$. 
\item Calculer alors $b_n$.
\end{enumerate}
\item Déterminer enfin $f^n$ en fonction de $f$ et $\Id$. 

\end{enumerate}


\end{exercice}

 
\begin{correction}
\begin{enumerate}
\item $g$ est une fonction de $\R^2$ dans $\R^2$, il suffit donc de vérifier que $g$ est linéaire. Pour cela on considère $(x_1,y_1)\in R^2,(x_2,y_2) \in \R^2, \lambda \in \R$ et 
\begin{align*}
g( (x_1,y_1) + \lambda (x_2,y_2))& = g( x_1 +\lambda x_2 , y_1 +\lambda y_2) \\
												&= (3(x_1 +\lambda x_2)  +2 (y_1 +\lambda y_2) , - (x_1 +\lambda x_2)) \\
												&= (3x_1 + 2y_1 +\lambda (3 x_2 +2y_2),  -x_1 -lambda x_2))\\
												&= (3x_1 + 2y_1, -x_1) +  \lambda (3 x_2 +2y_2,  -x_2))\\
												&= g(x_1,y_1) +\lambda g(x_2,y_2)
\end{align*}
Ainsi $g$ est linéaire, 
\conclusion{ $g$ est donc un endomorphisme de $\R^2$}
\item Soit $(x,y)\in \R^2$ 
\begin{align*}
g\circ g(x,y) &=g(3x+2y,-x)\\
					&= (3 (3x+2y) +2 (-x) , - (3x+2y))\\
					&=( 7x +6y, -3x-2y)\\
					&= (9x+6y, -3x) + (-2x,-2y)\\
					&= 3 (3x+2y,-x) -2(x,y)\\
					&= 3 g(x,y) - 2\Id (x,y)
\end{align*}

\conclusion{ On a bien $g^2 = 3g -2\Id$}
\item 
\begin{align*}
F&=\ker( g-\Id)\\
  &= \{ (x,y)\in \R^2\, |\, g(x,y)-(x,y) =(0,0)\}\\
  &= \{ (x,y)\in \R^2\, |\, (2x+2y,-x-y) =(0,0)\}\\
  &= \{ (x,y)\in \R^2\, |\, 2x+2y=0 \text{ et } x+y = 0 \}\\
  &= \{ (x,y)\in \R^2\, |\, x+y=0\}\\
  &= \{ (x,y)\in \R^2\, |\, x=-y\}\\
  &= \{ (-y,y)\, |\,y \in \R\}\\
  &=\{ y(-1,1)\, |\,y \in \R\}\\
  &= Vect(( -1,1))
\end{align*}

\conclusion{ Ainsi $F$ est un sev de $\R^2$ de dimension $1$, et $(-1,1)$ est une base de $F$}
\item 
\begin{align*}
G&=\ker( g-2\Id)\\
  &= \{ (x,y)\in \R^2\, |\, g(x,y)-2(x,y) =(0,0)\}\\
  &= \{ (x,y)\in \R^2\, |\, (x+2y,-x-2y) =(0,0)\}\\
  &= \{ (x,y)\in \R^2\, |\, x+2y=0 \text{ et } -x-2y = 0 \}\\
  &= \{ (x,y)\in \R^2\, |\, x+2y=0\}\\
  &= \{ (x,y)\in \R^2\, |\, x=-2y\}\\
  &= \{ (-2y,y)\, |\,y \in \R\}\\
  &=\{ y(-2,1)\, |\,y \in \R\}\\
  &= Vect(( -2,1))
\end{align*}

\conclusion{ Ainsi $G$ est un sev de $\R^2$ de dimension $1$, et $(-2,1)$ est une base de $G$}

\item  Comme $F$ et $G$ sont des espaces vectoriels, $0 \in F$ et $0\in G$ donc, 
$\{ (0,0) \} \subset F\cap G$. 


Soit $u \in F\cap G$, comme $u \in F$ on a $g(u)-u=0$ et  donc $g(u)=u$.  Comme $u \in G$ on a $g(u)-2u =0$ donc $g(u) =2u$. Ainsi 
$u=2u$ et donc $u=0$. On a donc $F\cap G\subset \{ (0,0)\}$

\conclusion{ Par double inclusion $F\cap G= \{ (0,0)\}$}

\item $u$ et $v$ ne sont pas proportionnels et forment donc une famille libre. Comme $Card((u,v)) = 2 = \dim(\R^2)$, $(u,v)$ est aussi une famille génératrice de $\R^2$. 
\conclusion{ $(u,v)$ est une base de $\R^2$}

\item On cherche $\lambda, \mu \in \R^2$ tel que 
$$\lambda u+\mu v =(x,v)$$
On obtient le système suivant : 

$$\left\{ \begin{array}{cc}
\lambda -2\mu &= x\\
-\lambda +\mu &=y
\end{array}
\right. \equivaut\left\{ \begin{array}{cc}
\lambda -2\mu &= x\\
-\mu &=y+ x
\end{array}
\right. \equivaut\left\{ \begin{array}{cc}
\lambda &= -x-2y\\
\mu &=-y- x
\end{array}
\right.$$

\conclusion{ Pour tout $(x,y) \in \R^2 $ on a $(-x-2y)u + (-y-z) v= (x,y)$}

\item 
Prouvons par récurrence la proposition  $P(n):"g^n(u)  =u \text{ et } g^n (v) = 2^n v"$

\paragraph{Initialisation }
$g(u)= u$ et $ g(v) =2v$ car $u\in F$ et $v\in G$. donc $P(1) $ est vraie. 

\paragraph{Hérédité}
On suppose qu'il existe $n\in \N^*$ tel que $P(n)$ soit vraie. On a alors par HR, 
$g^n(u) = u$ et $g^n(v) =2^n v$ donc 
$$g^{n+1} (u) =g (g^n(u)) = g(u) =u$$
et $$g^{n+1} (u) =g (g^n(v)) = g(2^nv) =2^n g(v) = 2^{n+1} v$$

Ainsi la propriété $P$ est héréditaire, 

\conclusion{ Pour tout $n\in N^*,  g^n(u)  =u \text{ et } g^n (v) = 2^n v$}
\item 
D'après la question  7: 
$$g^n (x,y) = g^n((-x-2y)u + (-y-z) v) = (-x-2y) g^(u) +(-y-z)g^n (v)$$

D'après la question 8, on a donc 
$$g^n (x,y) = (-x-2y) u + (-x-y) 2^n v$$

Ainsi 
\begin{align*}
g^n (x,y) &= (-x-2y) (1,-1) + (-2^n x -2^n y) (-2,1)\\
				&=( -x-2y +2^{n+1} x +2^{n+1} y , x+2y -2^n x-2^n y)
\end{align*}
Pour tout $(x,y) \in \R^2$ pour tout $n\in \N^* $ on a : 
\conclusion{  $g^n(x,y)=  (  (-1 +2^{n+1}) x +(-2+2^{n+1}) y , (1-2^n) x +(2-2^n) y )$}

\item 
La matrice de $g$ dans la base canonique est la matrice 
$$A =\begin{pmatrix}
3 & 2\\
-1 & 0
\end{pmatrix}$$
D'après la question précédente, la matrice de $g^n$ est 
\conclusion{$A^n = \begin{pmatrix}
-1+2^{n+1} & -2+2^{n+1}\\
 1-2^n & 2-2^n
\end{pmatrix}$}





\end{enumerate}
B - Cas général. 
\begin{enumerate}
\item Si $f$ vérifie $(*)$ on a 
$f^2 = 3f- 2\Id_E$ donc $-f^2+3f =2\Id_E$ soit encore 
$$f\circ \frac{1}{2}(-f +3\Id) = \Id_E$$
\conclusion{ 
$f$ est bijective et $f^{-1} =  \frac{1}{2}(-f +3\Id) $}

\item Soit $f =\lambda \Id_E$ une solution de $(*)$ on a alors 
$f^2 = \lambda^2 \Id_E$   et donc 
$$\lambda^2 \Id_E = 3\lambda  \Id_E - 2\Id_E$$
Donc $$(\lambda^2 - 3\lambda +2) \Id_E =0$$
Comme $\Id_E$ n'est pas l'application nulle, on a $(\lambda^2 - 3\lambda +2)=0$ ainsi 
$$(\lambda -1) (\lambda -2) =0$$
 Finalement $\lambda \in \{1,2\}$
 
 \conclusion{ Les seules solutions de $(*)$ de la forme $\lambda \Id_E$ sont $\Id_E $ et $2\Id_E$} 
 
 \item \begin{enumerate}
 \item Soit $f$ solution   de $(*)$ on a donc $f^2 =3f-2\Id_E$, en composant par $f$ on obtient 
 $$f^3 =3f^2 -2f$$
 Or $f^2 =3f-2\Id_E$ donc 
\begin{align*}
f^3 &= 3 (3f-2\Id_E) -2f \\
		&= 7f -6\Id_E
\end{align*}

\conclusion{ $f^3 =7f-6\Id_E$}

\item Montrons la propriété par récurrence. 

Pour $n=1$ ,  $f^1 =f$ donc $a_1 =1 $ et $b_1 =0$ satisfont la condition demandée. 

Supposons donc qu'il existe $n\in \N$ tel que $P(n) $ soit vraie. Il existe donc $(a_n,b_n)$ tel que 
$f^n =a_n f +b_n \Id_E$
En composant par $f$ on obtient 

$$f^{n+1} =a_n f^2 +b_n f$$
 Or $f^2 =3f-2\Id_E$ donc 
\begin{align*}
f^{n+1}  &= a_n (3f-2\Id_E) +b_nf  \\
		&= (3a_n +b_n) f -2a_n\Id_E\\
		&=a_{n+1} f +b_{n+1}\Id_E\\
\end{align*}
avec $a_{n+1} = 3a_n +b_n$ et $b_{n+1} = -2a_n$
\conclusion{ Pour tout $n\in \N$, il existe $(a_n,b_n)$ tel que $f^n=a_n f +b_n \Id_E$}






 \end{enumerate}
\item \begin{enumerate}
\item D'après la question précédente  $b_{n+1} = -2a_n$ donc $b_n =-2a_{n+1}$. En remettant dans l'équation  $a_{n+1} = 3a_n +b_n$, on obtient 
$$a_{n+1} =3 a_n -2a_{n-1}$$

\conclusion{ $\forall n\in \N, \, a_{n+1} -3 a_n +2a_{n-1}=0$}

\item  On reconnait une suite récurrente linéaire d'ordre 2 à coefficients constants. Son équation caractérisitique est $X^2 -3X +2 =0$ dont les racines sont $1$ et $2$. 

Ainsi il existe $\alpha, \beta \in \R$ tel que pour tout $n\in \N$ 

$$a_{n} = \alpha +\beta 2^n $$ 

D'après l'initialisation de la récurrence on sait que $a_0 = 0 $ et $a_1 =1$, donc 
$\alpha +\beta =0$ et $\alpha +2 \beta =1$. Tout calcul fait, on obtient : 
\conclusion{ $\forall n\in \N, a_n = -1 +2^n$} 

\item On sait que $b_n = -2a_{n+1}$ donc 
\conclusion{$b_n = -1 +2^{n+1}$}


\end{enumerate}

\end{enumerate}

\end{correction}









\begin{exercice} [Extrait du Concours Agro-Veto 2019]
Des éléments de syntaxe Python, et en particulier l'usage du module numpy, sont donnés en annexe . Dans tout ce qui suit, les variables $n, p, A, M, i, j$ et $c$ vérifient les conditions suivantes qui ne seront pas rappelées à chaque question :
\begin{itemize}


\item $n$ et $p$ sont des entiers naturels tels que $p \geq n \geq 2$;
\item $A$ est une matrice carrée à $n$ lignes inversible;
\item $M$ est une matrice à $n$ lignes et $p$ colonnes telle que la sous-matrice carrée constituée des $n$ premières colonnes de $M$ est inversible;
\item  $i$ et $j$ sont des entiers tels que $0 \leq i \leq n-1$ et $0 \leq j \leq n-1$;
\item  $c$ est un réel non nul.
\end{itemize}
On note $L_{i} \leftarrow L_{i}+c L_{j}$ l'opération qui ajoute à la ligne $i$ d'une matrice la ligne $j$ multipliée par c.
\begin{enumerate}


\item Soit la fonction initialisation:
\begin{lstlisting}{python}
def initialisation(A):
    n = np.shape(A)[0]
    mat = np.zeros((n,2*n))
    for i in range(0, n):
        for j in range(0, n):
            mat[i,j] = A[i,j]
    return(mat)
\end{lstlisting}
Pour chacune des affirmations suivantes, indiquer si elle est vraie ou fausse, en justifiant. L'appel initialisation (A) renvoie:
\begin{enumerate}


\item une matrice rectangulaire à $n$ lignes et $2 n$ colonnes remplie de zéros;
\item une matrice de même taille que $A$;
\item une erreur au niveau d'un range;
\item une matrice rectangulaire telle que les $n$ premières colonnes correspondent aux $n$ colonnes de $A$, et les autres colonnes sont nulles.\end{enumerate}
\item  Les trois fonctions multip, ajout et permut suivantes ne renvoient rien : elles modifient les matrices auxquelles elles s'appliquent.
\begin{enumerate}


\item Que réalise la fonction multip?
\begin{lstlisting}{python}
def multip(M, i, c):
    p = np.shape(M)[1]
    for k in range(0, p):
        M[i,k] = c*M[i,k]
\end{lstlisting}


\item Compléter la fonction ajout, afin qu'elle effectue l'opération $L_{i} \leftarrow L_{i}+c L_{j}$.
\begin{lstlisting}{python}
def ajout(M, i, j, c):
    p = np.shape(M)[1]
    for k in range(0, p):
        _____ ligne(s) a completer _____ 
\end{lstlisting}

\item Écrire une fonction permut prenant pour argument $M, i$ et $j$, et qui modifie $M$ en échangeant les valeurs des lignes $i$ et $j$.
\end{enumerate}
Dans la suite du sujet, l'expression "opération élémentaire sur les lignes" fera référence à l'utilisation de permut, multip ou ajout.
\item  Soit la colonne numéro $j$ dans la matrice $M$. On cherche le numéro $r$ d'une ligne où est situé le plus grand coefficient (en valeur absolue) de cette colonne parmi les lignes $j$ à $n-1$. Autrement dit, $r$ vérifie :
$$
|A[r, j]|=\max \{|A[i, j]| \text { pour } i \text { tel que } j \leq i \leq n-1\} .
$$
Écrire une fonction \texttt{rang\_pivot} prenant pour argument $M$ et $j$, et qui renvoie cette valeur de $r$.

 Lorsqu'il y a plusieurs réponses possibles pour $r$, dire (avec justification) si l'algorithme renvoie le plus petit $r$, le plus grand $r$ ou un autre choix.
 
  (L'utilisation d'une commande max déjà programmée dans Python est bien sûr proscrite.)
  
\item Soit la fonction mystere:
\begin{lstlisting}{python}
def mystere(M):
    n = np.shape(M)[0]
    for j in range(0, n):
        r = rang_pivot(M, j)
        permut(M, r, j)
        for k in range(j+1, n):
            ajout(M, k, j, -M[k,j]/M[j,j])
        print(M) 
\end{lstlisting} 

\begin{enumerate}
\item On considère dans cette question l'algorithme mystere appliqué à la matrice \[M_{1}=\left(\begin{array}{ccc}3 & 2 & 2 \\ -6 & 0 & 12 \\ 1 & 1 & -3\end{array}\right)\] Indiquer combien de fois la ligne print (M) est exécutée ainsi que les différentes valeurs qu'elle affiche
\item De façon générale, que réalise cet algorithme?
\end{enumerate} 
\item On considère la fonction reduire:
\begin{lstlisting}{python}
def reduire(M):
    n = np.shape(M)[0]
    mystere(M)
    for i in range(0, n):
        multip(M, i, 1/M[i,i])
    #Les lignes suivantes sont \'a compl\'eter :
        __________________________        
\end{lstlisting}
Compléter la fonction afin que la portion de code manquante effectue les opérations élémentaires suivantes sur les lignes :

pour $j$ prenant les valeurs $n-1, n-2, \ldots, 1$, faire :


\hspace{2cm}	pour $k$ prenant les valeurs $j-1, j-2, \ldots, 0$, faire :

$$
\mathrm{L}_{\mathrm{k}} \leftarrow \mathrm{L}_{\mathrm{k}}-\mathrm{M}[\mathrm{k}, \mathrm{j}] \mathrm{L}_{\mathrm{j}}
$$



Indiquer ce que réalise cette fonction.
\end{enumerate}

\begin{center}
\textbf{\huge{Annexe}}
\end{center}
On considère que le module numpy, permettant de manipuler des tableaux à deux dimensions, est importé via import numpy as np. Pour une matrice $M$ à $n$ lignes et $p$ colonnes, les indices vont de 0 à $n-1$ pour les lignes et de 0 à $p-1$ pour les colonnes.

\begin{center}
\begin{tabular}{ll}
\hline \multicolumn{1}{c}{ Python } & \multicolumn{1}{c}{ Interprétation } \\
\hline $\mathrm{abs}(\mathrm{x})$ & Valeur absolue du nombre $x$ \\
\hline $\mathrm{M}[\mathrm{i}, \mathrm{j}]$ & Coefficient d'indice $(i, j)$ de la matrice $M$ \\
\hline $\mathrm{np} \cdot$ zeros $((\mathrm{n}, \mathrm{p}))$ & Matrice à $n$ lignes et $p$ colonnes remplie de zéros \\
\hline $\mathrm{T}=\mathrm{np} . \mathrm{shape}(\mathrm{M})$ & Dimensions de la matrice $M$ \\
$\mathrm{~T}[0]$ ou np.shape $(\mathrm{M})[0]$ & Nombre de lignes \\
$\mathrm{T}[1]$ ou np.shape $(\mathrm{M})[1]$ & Nombre de colonnes \\
\hline $\mathrm{M}[\mathrm{a}: \mathrm{b}, \mathrm{c}: \mathrm{d}]$ & Matrice extraite de $M$ constituée des lignes $a$ à $b-1$ et des \\
& colonnes $c$ à $d-1:$ \\
& si $a$ (resp. $c)$ n'est pas précisé, l'extraction commence à la \\
& première ligne (resp. colonne) \\
& si $b$ (resp. $d)$ n'est pas précisé, l'extraction finit à la dernière \\
& ligne (resp. colonne) incluse \\
\hline
\end{tabular}
\end{center}
\end{exercice}



\end{document}