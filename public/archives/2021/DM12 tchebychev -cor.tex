\documentclass[a4paper, 11pt,reqno]{article}
\input{/Users/olivierglorieux/Desktop/BCPST/2020:2021/preambule.tex}
\geometry{hmargin=2.0cm, vmargin=3.5cm}

\author{Olivier Glorieux}


\begin{document}

\title{DM 11 : Correction}



\begin{exercice}
Pour tout réel $t>0, $ on note $P_t$ le polynôme $X^5 +tX-1 \in \R_5[X]$. Le but de ce problème est d'étudier les racines de $P_t$ en fonction de $t>0$. 
\begin{enumerate}
\item On fixe $t>0$ pour cette question. Prouver que $P_t$ admet une unique racine notée $f(t)$. 
\item Montrer que $f(t) \in ]0,1[$ pour tout $t>0.$
\item Montrer que $f$ est strictement décroissante sur $]0,+\infty[$.
\item En déduire que $f$ admet des limites finies en $0^+$ et en $+\infty$.

\item Déterminer $\lim_{t\tv 0^+} f(t)$. 

\item Déterminer $\lim_{t\tv+\infty} f(t)$. 
\item En déduire  $\lim_{t\tv +\infty} tf(t)= 1$. (Comment noter ce résultat avec le signe équivalent : $\sim$) 

\item Justifier que $f$ est la bijection réciproque de $g : ]0,1[\tv ]0,+\infty[$ 
$x \mapsto\frac{1-x^5}{x}$
\item \begin{enumerate}
\item Justifier que $f$ est dérivale sur $]0,+\infty[ $ et exprimer $f'(t)$ en fonction de $f(t)$ pour tout $t>0$.
\item En déduire la limite de $f'(t)$ en $0$. Calculer la limite de $t^2 f'(t)$ en $+\infty$ (Comment noter ce résultat avec le signe équivalent : $\sim$) 
\end{enumerate}
\end{enumerate}
\end{exercice}

\begin{cor}
\begin{enumerate}
\item On considère la dérivée de la fonction polynomiale. On a $P'_t(X) = 5X^4 +t $. Ainsi pour tout $x\in \R$ et pour tout $t>0$ 
$P'_t(x) \geq 0$. La fonction polynomiale $x\mapsto P_t(x)$ est donc strictement croissante  sur $\R$, par ailleurs elle est continue. On peut appliquer le théorème de la bijection à $P_t$ pour la valeur  
$0 \in ]\lim_{x\tv +\infty} P_t(x) =+\infty $, $\lim_{x\tv -\infty} P_t(x) =-\infty [$.  Il existe donc une unique valeur, notée $f(t)$ par l'énoncé, telle que $P'_t(f(t)) =0$. 

\item Par définition de $P_t$ on a $P_t(0) = -1<0$ et $P_t(1) = t >0$. Comme $x\mapsto P_t(x)$ est strictement croissante et $P_t(f(f))=0$ on  obtient $f(t) \in ]0,1[$. 

\item Soit $t_1>t_2 $, on a $P_{t_1} (X)-P_{t_2} (X) = X^5 +t_1X-1 - (X^5 +t_2 X -1) = (t_1-t_2)X$
Donc pour $x>0$ on a 
$$P_{t_1} (x)-P_{t_2} (x)>0$$
On applique ce résultat à $f(t_2)$ on obtient 
$$P_{t_1} (f(t_2))-P_{t_2} (f(t_2))>0$$
$$P_{t_1} (f(t_2))>0$$
Comme $x\mapsto P_{t_1}(x)$ est une fonction croissant et que $P_{t_1}(f(t_1)) =0$ on obtient $f(t_2)> f(t_1)$
Finalement $t\mapsto f(t) $ est décroissante. 

\item $f$ est montone et bornée. Le théorème des limites monotones assure que  $f$ admet des limites finies en $0^+$ et en $+\infty$. 


\item Notons $\ell$ la limite  $\lim_{t\tv 0^+}f(t)= \ell$. Par définition de $f$ on a $f(t)^5 +t f(t)-1= 0$. Cette expression admet une limite quand $t\tv 0$, on  a $\lim_{t\tv 0^+} f(t)^5 +t f(t)-1 =\ell^5-1$. Par unicité de la limite on a donc $\ell^5-1 =0$. Et donc $\ell =1$ (car $\ell$ est réel).  

\item Notons $\ell'$  la limite  $\lim_{t\tv +\infty }f(t)= \ell'$.
Supposons par l'absurde que cette limite soit non nulle. On a alors $\lim_{t\tv +\infty } tf(t) =+\infty$. En passant à la limite dans l'égalité 
$f(t)^5 +tf(t) -1 =0$ on obtient 
$+\infty =0$ ce qui est absurde. 
Donc $$\lim_{t\tv +\infty }f(t)=0.$$

\item En repartant de l'égalité $f(t)^5 +tf(t)-1=0$ on obtient 
$$tf(t) = 1 -f(t)^5$$
Comme $lim_{t\tv +\infty} f(t)=0$ on a 
$$\lim_{t\tv +\infty} tf(t)=1$$
En d'autres termes $\ddp f(t)\sim_{+\infty} \frac{1}{t}$ 

\item $f$ est strictement  montone sur $]0,+\infty[$ donc $f$ est une bijection $]0,+\infty[$ sur son image. $\lim_{t\tv 0 }f(t)=1$ et  $\lim_{t\tv +\infty }f(t)=0$. Donc $f( ]0,+\infty[) =]0,1[$ et $f$ est une bijection de $]0,+\infty[$ sur $]0,1[$. 

Par définition de $f$ on  a
$f(t)^5 +tf(t)-1=0$
Donc 
$tf(t) =  -f(t)^5 +1$. Comme $f(t)>0$, on  a :
$$t =\frac{1-f(t)^5}{f(t)}$$
Soit $g(x) = \frac{1-x^5}{x}$ on a bien $g(f(t)) =t$ Donc $g\circ f =\Id$. Ainsi  la réciproque de $f$  est bien la fonction $g : ]0,1[\tv ]0,\infty[$. 
\item 
\begin{enumerate}
\item $g$ est dérivable et pour tout $x\in ]0,1[$ 
$$g'(x) = \frac{-1-4x^5}{x^2}.$$
$g'(x) $ est différent de $0$ car $-1-4x^5$ est différent de $0$ sur $]0,1[$, donc $f$ est dérivable et 
$$f'(t) =\frac{1}{g'(f(t)} =  \frac{f(t)^2}{ -1-4f(t)^5}.$$

\item $\lim_{t\tv 0} f(t) = 1$ donc $$\lim_{t\tv 0} f'(t) = \frac{1^2}{-1-4\times 1 }= \frac{-1}{5}$$

On  a aussi 
$t^2 f'(t) = \frac{(tf(t)^2}{-1-4 f(t)^5}$ Comme $\lim_{t\tv \infty } tf(t)=1 $ et $\lim_{t\tv \infty } f(t)=0 $  en passant à la limite dans l'égalité précédente on obtient : 
$$\lim_{t\tv \infty } t^2f'(t)=\frac{1}{-1}=-1 $$ 
En d'autres termes : 
$$f'(t) \sim_{+\infty} \frac{-1}{t^2}$$

\end{enumerate}



\end{enumerate}
\end{cor}




\begin{exercice}
On considère la suite de polynômes $(T_n)_{n\in \N}$ définie par 
$$ T_0=1 \quadet T_1=X \quadet \forall n\in \N,\, T_{n+2}=2X T_{n+1}-T_n$$
\begin{enumerate}
\item \begin{enumerate}
\item Calculer $T_2$, $T_3$ et $T_4$.
\item Calculer le degré et le coefficient de $T_n$ pour tout $n\in\N$. 
\item Calculer le coefficient constant de $T_n$. 
\end{enumerate}
\item \begin{enumerate}
\item Soit $\theta \in \R$. Montrer que pour tout $n\in \N$ on  a  $T_n(\cos(\theta)) =\cos(n\theta)$.
\item En déduire que $\forall x\in [-1,1], $ on a $T_n(x) =\cos(n \arccos(x))$. 
\end{enumerate}
\item \begin{enumerate}
\item En utilisant la question 2a), déterminer les racines de $T_n$ sur $[-1,1]$. 
\item Combien de racines distinctes a-t-on ainsi obtenues ? Que peut on en déduire ? 
\item Donner la factorisaiton de $T_n$ pour tout $n\in \N^*$. 
\end{enumerate}
\end{enumerate}


\end{exercice}
\begin{cor}
\begin{enumerate}
\item \begin{enumerate}
\item $T_2 = 2X^2-1$, $T_3 =4X^3-3X$, $T_4 = 8X^4 -8X^2+1 $
\item Montrons par récurrence que $\deg(T_n) =n$. 
Comme la suite est une suite récurrente d'ordre 2, on va poser comme proposition de récurrence $$P(n) : ' \deg(T_n) =n \text{ ET } \deg(T_{n+1}) =n+1'$$
C'est vrai pour $n=0,1,2$ et $3$.  On suppose qu'il existe un entier $n_0\in \N$ tel que $P(n_0)$ soit vrai et montrons $P(n_0+1)$. 
On cherche donc à vérifier $\deg(T_{n_0+1}) =n_0+1 \text{ ET } \deg(T_{n_0+2}) =n_0+2'$. La première égalité est vraie par hypothèse de récurrence. 
La seconde vient de la relation $T_{n_0+2} =2 X T_{n_0+1} -T_{n_0}$
En effet, par hypothèse de récurrence $T_{n_0+1}$ est de degré $n_0+1$ donc $2 X T_{n_0+1}$ est de degrés $n_0 +2$. Comme $\deg(T_{n_0}) =n_0<n_0+2$, on a 
$$\deg(T_{n_0+2} )=\max(\deg(2 X T_{n_0+1}),\deg(T_{n_0})) = n_0+2$$

Ainsi par récurrence pour tout $n\in \N$, $\deg(T_n)= n$.
\item La récurrence précédente montre que le coefficient dominant, notons le $c_n$ vérifie $c_{n+2} = 2 c_{n+1}$. Ainsi $c_n = 2^nc_0 =2^n$. 

\end{enumerate}
\item 
\begin{enumerate}
\item Montrons le résultat par récurrence. On pose 
$$Q(n) : " \forall \theta\in \R, T_n(\cos(\theta)) =\cos(n\theta)  \text{ ET } T_{n+1}(\cos(\theta)) =\cos((n+1)\theta)"$$

$Q(0)$ est vraie par définition de $T_0 $ et $T_1$ 

Supposons qu'il existe $n\in \N$ tel que $Q(n)$ soit vrai et montrons $Q(n+1)$. Il suffit de montrer que $\forall \theta \in \R$
$$T_{n+2}(\cos(\theta) )=\cos((n+2)\theta)$$

On a par définition de $T_{n+2}$ 
$$T_{n+2} (\cos(\theta))  = 2\cos(\theta) T_{n+1}(\cos(\theta)) -T_n(\cos(\theta))$$
Par hypothèse de récurrence on a 
$T_{n+1}(\cos(\theta)) =\cos((n+1)\theta)$ et 
$T_{n}(\cos(\theta))=\cos(n\theta) $ donc  
$$T_{n+2} (\cos(\theta)) =2 \cos(\theta) \cos((n+1) \theta) - \cos(n \theta)$$
Les formules trigonométriques donnent : 
\begin{align*}
2 \cos(\theta) \cos((n+1) \theta)   &=\cos(\theta+(n+1) \theta) +\cos(\theta-(n+1) \theta)\\
&=\cos((n+2) \theta) + \cos(-n\theta)\\
&=\cos((n+2) \theta) + \cos(n\theta)
\end{align*}
Donc 
$$T_{n+2} (\cos(\theta))  = \cos((n+2) \theta) + \cos(n\theta)-\cos(n\theta) = \cos((n+2)\theta)
$$


Par récurrence, pour tout $\theta \in \R$ et tout $n\in \N$: 
$$T_n(\cos(\theta ) ) =\cos(n\theta)$$

On peut répondre maintenant facilement à la question 2c) (avec la faute de frappe coefficient constant au lieu de coefficient dominant). Le coefficient constant vaut $T_n(0) = T_n(\cos(\pi/2))  =\cos(n \pi/2) $ 

Donc $T_n(0) = 0$ pour $n=2k$, $k\in \Z$. Pour $n= 2k+1$, $k\in \Z$, on a alors $T_{2k+1}(0) = \cos(( 2k+1 ) \pi/2 ) =\frac{(-1)^k}{2}$

\item Soit $x\in [-1,1]$ on  note $x =\cos(\theta)$, avec $\theta \in [0,\pi]$ on a  alors 
$\theta =\arccos( x) $. D'après la question précédente on a donc pour tout $x\in [-1,1]$: 
$$T_n(x) = \cos( n \arccos(x))$$



\end{enumerate}
\item 
\begin{enumerate}
\item Pour tout $\theta $ tel que $n\theta  \equiv \frac{\pi}{2}[\pi]$,  on a $\cos(n\theta) =0$ 

Ainsi pour tout $\theta $ tel que $\theta \equiv \frac{\pi}{2n}[\frac{\pi}{n}]$, 
$$T_n(\cos(\theta) ) =0$$

On obtient ainsi $n$ racines entre $[-1,1]$ données par 
$$\{  \cos\left(\frac{\pi+2k\pi }{2n}\right)\, |  \,  k\in [0,n-1] \}$$

\item On a obtenu $n$ racines. Comme $T_n$ est de degrés $n$, on a obtenu toutes les racines, ainsi $T_n$ se factorise de la manière suivante\footnote{ sans oublier le coéfficient dominant, merci Marie.}  : 
\item 
$$T_ n (X)  =2^n\prod_{k=0}^{n-1} \left(X - \cos\left(\frac{\pi+2k\pi }{2n}\right) \right) $$

\end{enumerate}

\end{enumerate}
\end{cor}




\end{document}