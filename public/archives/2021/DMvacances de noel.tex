\documentclass[a4paper, 11pt,reqno]{article}
\input{/Users/olivierglorieux/Desktop/BCPST/2020:2021/preambule.tex}
\newif\ifshow
\showfalse
\input{/Users/olivierglorieux/Desktop/BCPST/2021:2022/ifshow.tex}




\author{Olivier Glorieux}


\begin{document}

\title{DM de Noël 
}

\begin{exercice}
Soit $$A=\left(\begin{array}{rr} 2&1\\0&2\\-1&0 \end{array}\right) \quadet B=\left(\begin{array}{rrr} 0&2&1\\2&2&1 \end{array}\right)$$
\begin{enumerate}
\item Calculer $B^T$.
\item Calculer $-2A$
\item Calculer $-2A+B^T$
\end{enumerate}
\end{exercice}


\begin{exercice}
Soit $$A=\left(\begin{array}{rr} 2&1\\0&2 \end{array}\right) \quadet B=\left(\begin{array}{rrr} 0&2&1\\2&2&1 \end{array}\right)$$
\begin{enumerate}
\item Calculer $A^2$.
\item Calculer si cela est possible $AB$ et $BA$
\end{enumerate}
\end{exercice}




\begin{exercice}
Soit $$B=\left(\begin{array}{rrr} 0&2&-1\\-2&0&1 \end{array}\right)$$
\begin{enumerate}
\item Calculer $BB^T$ puis $B^T B$
\end{enumerate}
\end{exercice}


\begin{exercice}
Soit $$J=\left(\begin{array}{rrr} 1&1&1\\1&1&1 \\1&1&1 \end{array}\right)$$
\begin{enumerate}
\item Calculer $J^2$ et $J^3$
\item Montrer que pour tout $n\in \N$, $$J^n =\left(\begin{array}{rrr} 3^{n-1}&3^{n-1}&3^{n-1}\\3^{n-1}&3^{n-1}&3^{n-1} \\3^{n-1}&3^{n-1}&3^{n-1} \end{array}\right)$$
\end{enumerate}
\end{exercice}




\begin{exercice}
Soit $$A=\left(\begin{array}{rr} 2&1\\0&-3 \end{array}\right) \quadet B=\left(\begin{array}{rrr} 0&2&1\\2&2&1 \end{array}\right)$$
\begin{enumerate}
\item Calculer l'inverse de $A$ et $B$ si cela est possible. 
\item Déterminer $A^n$.
\end{enumerate}
\end{exercice}


\begin{exercice}
Soit $A$ une matrice et $P$ une matrice inversible. 
\begin{enumerate}
\item A-t-on $(AP)^2 =A^2 P^2 $?
\item Montrer qu'on a en revanche : 
$$(P^{-1}A P )^2 =P^{-1} A^2 P$$
\item Puis par récurrence  que  pour tout $n\in \N$
$$(P^{-1}A P )^n =P^{-1} A^n P$$
\end{enumerate}


\end{exercice}

\begin{exercice}

 Soit $$P= \left(
\begin{array}{ccc}
1&1&0\\
1&0&1\\
2&2&1
\end{array}
 \right) \quadet A=\left(
\begin{array}{ccc}
0&-1&1\\
4&1&-2\\
2&-2&1
\end{array}
 \right)$$

\begin{enumerate}
\item  Calculer $P^{-1}$
\item Calculer $P^{-1}AP$
\end{enumerate}
\end{exercice}

\begin{exercice}
Soit $$ A_x=\left(\begin{array}{rrr}  0&2&x\\0&2&1\\2&2&1 \end{array}\right)$$
\begin{enumerate}
\item Calculer le rang de $A_x$ en fonction de $x$.
\item Donner l'inverse de $A_x$ quand cela a un sens.
\end{enumerate}
\end{exercice}


\begin{exercice}
Soit $$ A_\lambda=\left(\begin{array}{cc}  1-\lambda&2\\0&2-\lambda \end{array}\right)$$
\begin{enumerate}
\item Calculer le rang de $A_\lambda$ en fonction de $\lambda$.
\item Donner l'inverse de $A_\lambda$ quand cela a un sens.
\end{enumerate}
\end{exercice}







\begin{exercice}
Soit $A$ la matrice 
$$A=\left(
\begin{array}{ccc}
0&-1&1\\
4&1&-2\\
2&-2&1
\end{array}
 \right)$$
 
 
\begin{enumerate}
\item Résoudre le système $AX=\lambda X$ d'inconnue $X =\left(
\begin{array}{c}
x\\
y\\
z
\end{array}
 \right)$ où $\lambda$ est un paramètre réel. 
 
 \item Soit $e_1= \left(
\begin{array}{c}
1\\
1\\
2
\end{array}
 \right)$,  $e_1= \left(
\begin{array}{c}
1\\
0\\
2
\end{array}
 \right)$, et  $e_1= \left(
\begin{array}{c}
0\\
1\\
1
\end{array}
 \right)$.
 Calculer $Ae_1, Ae_2$ et $Ae_3$. 
 
\item Montrer par récurrence que $A^ne_1= e_1$. 
\item Donner de même la valeur de $A^n e_2 $ et $A^n e_3$.
\item Soit $P= \left(
\begin{array}{ccc}
1&1&0\\
1&0&1\\
2&2&1
\end{array}
 \right)$ 
 
 Montrer que $P$ est inversible et calculer son inverse. 
 \item Soit $D=P^{-1}AP$. Calculerr $D$. 
 \item Montrer par récurrence que $D^n = P^{-1}A^n P$
 \item En déduire la valeur de $A^n$. 
\item Soit $\suite{x}, \suite{y} $ et $\suite{z}$ les suites définies par : 
$x_0=1, y_0=1 $ et $z_0=1$ et pour tout $n\in \N$ :
$$\left\{
\begin{array}{cll}
x_{n+1} &= &-y_n+z_n\\
y_{n+1}&=4x_n&+y_n-2z_n\\
z_{n+1}&=2x_n&-2x_n+z_n
\end{array}
 \right.$$ 
Soit $X_n = \left(
\begin{array}{c}
x_{n}\\
y_{n}\\
z_{n}
\end{array}
 \right)$
 
Montrer que $X_{n+1} = A X_n$. 
\item Montrer par récurrence que pour tout $n\in \N$, $$X_n = A^n X_0$$
\item En déduire le terme général de $\suite{x}$ en fonction de $n$. 
\end{enumerate} 
\end{exercice}




\end{document}