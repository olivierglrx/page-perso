\documentclass[a4paper, 11pt,reqno]{article}
\input{/Users/olivierglorieux/Desktop/BCPST/2020:2021/preambule.tex}
\newif\ifshow
\showtrue
\usepackage{eurosym}
\input{/Users/olivierglorieux/Desktop/BCPST/2021:2022/ifshow.tex}

\geometry{hmargin=2.0cm, vmargin=2cm}
\newenvironment{amatrix}[1]{%
  \left(\begin{array}{@{}*{#1}{c}|c@{}}
}{%
  \end{array}\right)
}

\author{Olivier Glorieux}


\begin{document}

\title{Correction : DS 8 - Concours Blanc
}

%\begin{exercice}
%Soit $f$ la fonction définie pour tout $x$ par : 
%$$f(x)=e^{-1/x^2} $$
%
%\begin{enumerate}
%\item Donner l'ensemble de définition de $f$.
%\item Etudier les limites de $f$ et montrer que $f$ est prolongeable par continuité en $0$. 
%\item On appelle encore $f$ la fonction prolongée en $0$. Montrer alors que $f$ est dérivable sur $\R$. 
%\item Justifier que $f$ est de classe $\cC^\infty$ sur $\R^*$ et que pour tout $n\in \N$ il existe $P_n \in \R[X]$ (que l'on ne cherchera pas à calculer) tel que pour tout $x\in \R^*$ : 
%$$f^{(n)} (x) =P_n\left(\frac{1}{x^2}\right) e^{-1/x^2}$$ 
%\item En déduire que $f$ est de classe $\cC^\infty $ sur $\R$ et expliciter la valeur de $f^{(n)}(0)$
%\item Justifier que $f$  admet un développement limité à tout ordre en $0$ et donner son DL à l'ordre $n\in \N$.
%\end{enumerate}
%
%\end{exercice}

\begin{exercice}
\begin{enumerate}
\item Enoncer le théorème des accroissements finis avec ses hypothéses. 
\item A l'aide de ce théorème prouver que pour tout $x>0$:
$$\frac{1}{x+1} <\ln(x+1) -\ln(x) < \frac{1}{x}$$
\item On note $\ddp S_n=\sum_{k=1}^n \frac{1}{k}$, déduire des deux inégalités précédentes  que  pour tout $n\in \N^*$ :
$$\ln(n+1) <S_n  <\ln(n)+1$$
\item En déduire un équivalent de $\suiteun{S}$ quand $n$ tend vers $+\infty$. 



\end{enumerate}
\end{exercice}

\vspace{0.5cm}
\begin{exercice}


On définit l'application : 
$$g \left| \begin{array}{ccl}
\R^3 &\tv& \R^3 \\
(x,y,z) &\mapsto & (2y-2z, -2x+4y-2z, -2x+2y)
\end{array}\right.$$
\begin{enumerate}
\item Montrer que pour tout  $(\lambda_1,\lambda_2) \in \R^2$,  $u_1\in \R^3$ et $u_2\in \R^3$,  montrer que 
$$g(\lambda_1 u_1+\lambda_2 u_2) =\lambda_1 g(u_1)+\lambda g(u_2)$$
(On dit alors que $g$ est linéaire) 
\item Soit $u=(1,1,1)$, $v=(2,3,1)$ et $w=(0,1,1)$. Calculer $g(u), g(v)$ et $g(w)$ et les exprimer en fonction de $u, v$ et $w$.

\item Soit $E_0 = \{ (x,y,z)\in \R^3\, |\, g(x,y,z)=(0,0,0)\}$. 
Montrer que $E_0$ est un sous-espace vectoriel de $\R^3$ et en donner une base. 
\item  On note $\Id_3$ la fonction identité de $\R^3$, à savoir, 
$$\Id_3 : (x,y,z)\mapsto (x,y,z)$$

Soit  $E_2 = \{ (x,y,z)\in \R^3\, |\, (g-2\Id_3)(x,y,z)=(0,0,0)\}$. 
On admet que  $E_2$ est un sous-espace vectoriel de $\R^3$ de dimension 2. En donner une base. 
\item Montrer que $E_0\cap E_2= \{ (0,0,0)\}$.\footnote{Cette question peut se répondre avec ou sans sytème...}
\item On note toujours $u=(1,1,1)$, $v=(2,3,1)$ et $w=(0,1,1)$. Montrer que $(u,v,w)$ est une base de $\R^3$. 
\item Soit $A=(1,-1,-3)$. Donner les coordonées de  $A$ dans la base $(u,v,w)$. \footnote{Autrement dit,  exprimer $A$ en fonction de $(u,v,w)$}
\item A l'aide de la question précédente et de la question 1, montrer que 
$$g(A)= 2(v-3w)$$
%\item On note $g^2=g\circ g$. Exprimer $g(x,y,z)$ en fonction de $x,y,z$.
\item On note $g^2=g\circ g$. Montrer que  $g^2(A) = 4(v-3w)$.  \footnote{On pourra utiliser les questions 1 et 8}
\item On note $g^n = g\circ g\cdots \circ g $ où l'on a composé $n$ fois. Pour tout $n\in \N^*$ déterminer $g^n(A) $ en fonction de $n, v$ et $ w$.
\end{enumerate}
\end{exercice}

\begin{correction}
\begin{enumerate}

\item 
Soit $u_1= (x_1,y_1,z_1) $ et  $u_2= (x_2,y_2,z_2) $ et $\lambda \in \R$. On a donc 
$$u_1 +\lambda u_2 = (x_1+\lambda x_2,y_1+\lambda y_2 ,z_1+\lambda z_2)$$
et donc 
\begin{align*}
g(u_1+\lambda u_2) & = g  (x_1+\lambda x_2,y_1+\lambda y_2 ,z_1+\lambda z_2)\\
								&= (2(y_1 +\lambda y_2) - 2 (z_1+ \lambda z_2),  -2 (x_1 +\lambda x_2) + 4 (y_1 +\lambda y_2) - 2 (z_1 +\lambda z_2), -2 (x_1 +\lambda x_2) +2 (y_1 +\lambda y_2))\\
								&= (2y_1-2z_1, -2x_1 +4y_1-2z_1, -2x_1+2y_1) +\lambda  (2y_2-2z_2, -2x_2 +4y_2-2z_2, -2x_2+2y_2) \\
								&= g(u_1)+\lambda g(u_2)
\end{align*}
\conclusion{ $g(u_1+\lambda u_2)= g(u_1)+\lambda g(u_2)$} 

\item 

\begin{align*}
g(u)&= (2-2,-2+4-2, -2+2) =(0,0,0)\\
g(v)&=(6-2, -4+12-2, -4+6) = (4,6,2)=2v\\
g(w)&=(2-2, 4-2, 2) = (0,2,2) =2w
\end{align*}

\conclusion{$g(u)=0$ , $g(v)=2v$ , $g(w)=2w$}
\item $E_0$ n'est pas vide, en effet $g(0,0,0) = (0,0,0)$, donc $(0,0,0) \in E_0$.
Montrons que $E_0$ est stable par combinaisons linéaires. Soit $u_1, u_2\in E_0$ et $\lambda \in \R$, on a  alors $g(u_1+\lambda u_2) = g(u_1) + \lambda g(u_2) $ d'après la question $1$. Or $g(u_1) =g(u_2) = (0,0,0)$ par définition de $E_0$ donc 
$$g(u_1+\lambda u_2) =(0,0,0)$$
Ainsi $u_1+\lambda u_2 \in E_0$ 

\conclusion{ $E_0$ est un sous espace vectoriel de $\R^3$}

Trouvons maintenant une base de $E_0$, pour cela écrivons $E_0$ sous forme vectorielle. 
On a $(x,y,z) \in E_0 \equivaut g(x,y,z)=(0,0,0)$ ce qui équivaut au système suivant :
$$\left\{ 
\begin{array}{rrrc}
&2y&-2z&=0\\
-2x&+4y&-2z&=0\\
-2x&+2y& &=0
\end{array}
\right. \equivaut
\left\{ 
\begin{array}{rrrc}
-x&+y& &=0\\
-x&+2y&-z&=0\\
&y&-z&=0
\end{array}
\right. 
 $$
 
 
 $$\equivaut 
\left\{ 
\begin{array}{rrrc}
-x&+y& &=0\\
&y&-z&=0\\
&y&-z&=0
\end{array}
\right. 
\equivaut 
\left\{ 
\begin{array}{rrrc}
x& & &=z\\
&y& &=z
\end{array}
\right. 
 $$

Ainsi $E_0 = \{ (z,z,z) | z\in \R\} =Vect( (1,1,1))$

\conclusion{ $((1,1,1))$ est une base de $E_0$, $\dim(E_0)=1$}


\item D'après la question 2 :\\
$g(v)=2v$ donc $(g-2\Id)(v)=(0,0,0)$ donc $v\in E_2$. \\
$g(w) =2w$ donc $(g-2\Id)(w)=(0,0,0)$ donc $w\in E_2$.

On a donc $Vect(v,w) \subset E_2$
Or $(v,w)$ est une famille libre car les deux vecteurs ne sont pas proportionnels, donc c'est une base de $Vect(v,w)$. Donc $Vect(v,w)$ est de dimension $2$. Comme on d'après l'énoncé $Dim(E_2)=2$ on a 
l'égalité : 
$$\Vect(v,w) =E_2$$

\conclusion{ Finalement $(v,w)$ est donc aussi une base de $E_2$ }


\item Soit $X\in E_0 \cap E_2$, comme $X\in E_0$ on a $g(X)=(0,0,0)$ et comme $X\in E_2$ on a $(g-2\Id)(X)=(0,0,0) $ c'est-à-dire $g(X) =2X$. 
Ainsi $2X= (0,0,0)$ on a bien 
\conclusion{ $E_0\cap E_2 =\{ (0,0,0)\}$}
 
\item Soit $(a,b,c)\in \R^3$ tel que $au+bv+cw=(0,0,0)$. On  a donc $au =-bv-cw$.
Or $au\in E_0$ et $-bv-cw\in E_2$ et comme $au=-bv-cw$ 
$$ au \in E_0 \cap E_2 \quadet -bv-cw\in E_0\cap E_2$$
D'après la question précédente on a donc $au =(0,0,0)$ et comme $u\neq 0$, $a=0$. De même on a 
$-bv-cw=(0,0,0)$ et comme on a bvu que $v,w$ était libre, cecie impique que $b=c=0$

Ainsi $a=b=c=0$ donc la famille $(u,v,w)$ est libre, comme elle est de cardinal 3  on a finalment 
\conclusion{ $(u,v,w)$ est une base de $\R^3$}

\item On cherche $a,b,c\in \R^3$ tel que  $au+bv+cw= (1,-1,3)$ c'est-à-dire $(a,b,c)$ qui vérifie le système
$$\left\{ 
\begin{array}{ccc}
a+2b& &=1\\
a+3b&+c&=-1\\
a+b&+c&=-3
\end{array} \right.$$
Après calcul on obtient : $a=-1, b=1, c=-3$
\conclusion{  Dans la base $(u,v,w) $ les coordonnées de $A $ sont  $(-1,1,-3)$}

\item D'après la question précédente $g(A) = g(-u+v-3w) $. Ce qui donne d'après la question $1$, 
$$g(A) = -g(u) +g(v) - 3g(w)$$
Or $g(u)=0, g(v)=2v $ et $g(w)=2w$ donc 

$g(A)= 2v -3 \times 2w$
\conclusion{ $g(A) = 2 (v-3w)$}

\item 
\begin{align*}
g^2 (A) &= g\circ g(A)\\
			&= g(g(A))\\
			&= g( 2 (v-3w) & \text{ D'après la question 8}\\
			&= g(2v-2\times 3 w )\\
			&=2g(v) -2\times 3 g(w) & \text{D'après la question 1}		\\
			&=2 (2 v- 2 \times 3 \times 2w )& 	\text{D'après la question 2}		\\
			&= 4 (v-3w)
\end{align*}
\conclusion{ On a bien $g^2 (A) =4(v-3w)$}

\item On montre par récurrence la propriété  suivante 
$P(n) : " g^n (A) = 2^n (v-3w)"$

L'initialisation a été faite pour $n=1$ et $n=2$ dans les questions précédentes. 

Montrons que $P$ est héréditaire. On suppose donc qu'il existe $n$ tel que $P(n)$ soit vrai. On a alors 
$g^n(A) =2^n(v-3w)$ En composant par $g$ on obtient 
\begin{align*}
g^{n+1} (A) &= g\circ g^n(A)\\
			&= g(g^n(A))\\
			&= g( 2^n (v-3w) & \text{ Par hypothése de récurrence}\\
			&= g(2^nv-2^n\times 3 w )\\
			&=2^ng(v) -2^n\times 3 g(w) & \text{D'après la question 1}		\\
			&=2^n (2 v- 2 \times 3 \times 2w) & 	\text{D'après la question 2}		\\
			&= 2^{n+1} (v-3w)
\end{align*}

\conclusion{ Pour tout $n\in \N^*$, $ g^n (A) = 2^n (v-3w)$}

\end{enumerate}
\end{correction}

\vspace{
0.5cm}
\newpage
\begin{exercice}
Une puce se déplace le long d'un axe. Au temps $n=0$ la puce est en 0. Puis à chaque saut elle monte de 1 avec probabilité 1/2 et descend de 1 avec probabilité $1/2$.  

%On note $X_n$ la position de la puce au temps $n$.

 On s'intérresse à la probabilité que la puce revienne à l'origine. On note $A_n$ l'événement 
 $$A_n=\text{ 'La puce est en 0 au saut $n$'}$$

 \begin{enumerate}
  \item Quelle est la probabilité de l'événement  $A_1$ ? 
 \item Quelle est la probabilité de l'événement  $A_2$ ? 
 \item Soit $E_n$ l'événement 'la puce est sur un nombre pair au rang $n$'. Justifier que pour tout $n\in \N$, $P(E_{2n+1}) =0$ et $P(E_{2n}) =1$. En déduire, pour tout $n\in \N$ la valeur de $P(A_{2n+1})$
 \item On fixe un nombre entier pair que l'on note $2n$.   Soit $M_{k}$ l'événement 'la puce est montée $k$ fois durant les $2n$ sauts' et  $D_k$ l'événement 'la puce est descendue $k$ fois durant les $2n$ sauts'.
 \begin{enumerate}
 \item Calculer $P(M_k)$ en fonction de $k$ er $n$. 
% \item Calculer $P_{D_n}(M_n)$.
 \item Exprimer l'événement $A_{2n}$ à l'aide des événements $M_n$ et $D_n$.
 \item En déduire la valeur de  $P(A_{2n})$ en fonction de $n$.
\end{enumerate}
 
\item On considère le programme suivant censé modéliser la position de la puce après $n$ sauts:
\begin{lstlisting}
def sauts(n):
    puce=0
    for i in range(n):
        p=random()
        if   :
            puce=puce+1
        else:
            puce=
    return(puce)
\end{lstlisting}

Recopier et compléter sur votre copie le programme précedent pour qu'il fonctionne. 
\item Ecrire une fonction python \texttt{A} qui prend en argument le nombre de sauts $n$ et retourne \texttt{True} si la puce est en $0$ au temps $n$ et \texttt{False} sinon. 

\item Ecrire une fonction Python qui permet de donner une valeur approchée de $P(A_{2n})$ en itérant un grand nombre de fois l'expérience. (A l'aide de la fonction \texttt{A} et sans utiliser la formule obtenue en 5c) 

\item Ecrire une fonction Python qui permet de modéliser les sauts de puce jusqu'à la première fois où la puce revient en $0$ et retourne le nombre de sauts effectués. 
 \end{enumerate}
 
 
\end{exercice}

\begin{correction}
\begin{enumerate}


\item Au saut $1$ la puce est soit en $1$ soit en $-1$ donc $P(A_1)=0$
\item Soit $T_1$ l'événement la puce est en $1$ au saut $1$ et $T_{-1}$ la puce est en $-1$ au saut $1$. $(T_1,T_{-1})$  st un SCE et on peut appliquer la fomrule des probabilités totales, on obtient :
$$P(A_2)= P(A_2 |T_1) P(T_1)  +P(A_2|T_{-1})P(T_{-1}) $$
On a $P(A_2|T_1) = P(A_2| T_{-1}) =1/2$  et $P(T_1) =P(T_{-1}) =1/2$ 
donc 
\conclusion{ $P(A_2) = \frac{1}{2}$}
\item Soit $Q(n) $ la proposition " $P(E_{2n}) =1$ et $P(E_{2n+1} ) = 0 $" 
Initialisation  : 
En $0$ la puce est en $0$ donc $P(E_0) =1$

Au saut $1$ la puce est soit en $1$ soit en $-1$, en particulier elle n'est pas sur un nombre pair. Donc  $P(E_1)=0$ et la propriété $Q(0) $ est vérifiée. 

Hérédité : On suppose que la proposition $Q$ est vraie pour un entier $n\in \N$ on a donc 
$P(E_{2n}) =1$ et $P(E_{2n+1}) =0$. Calculons maintenant 
$P(E_{2(n+1)} )= P(E_{2n+2})$. 

On utilise le SCE $E_{2n+1}$, $\overline{E_{2n+1}}$
\begin{align*}
P(E_{2(n+1)} ) &= P(E_{2n+2}| E_{2n+1}) P(E_{2n+1}) + P(E_{2n+2}| \overline{E_{2n+1}}) P(\overline{E_{2n+1}})\\
&= 0 + 1 \\
&=1 
\end{align*}


\item $A_{2n+1}\subset E_{2n+1}$ donc $P(A_{2n+1} ) \leq P(E_{2n+1})=0$. Ainsi 
\conclusion{ $P(A_{2n+1}) =0$}
 
\item \begin{enumerate}
\item Pour monter $k$ fois il faut choisir les $k$ fois où  parmi les 2n sauts où la puce monte. On obtient donc  $P(M_k)  =\binom{2n}{k} (\frac{1}{2})^k \frac{1}{2}^{2n-k}$

\conclusion{ $P(M_k )= \binom{2n}{k}\frac{1}{2^{2n}}$}
\item $A_{2n} = M_n \cap D_n$ 
\item Remarquons que si $M_n$ est vérifiée alors nécessaire $D_n$ est vérifié. Ainsi 
$P_{M_n} (D_n) = 1$, donc 
$$P(A_{2n} )  = P(M_n) P_{M_n}(D_n)= P(M_n)$$
Finalement 
\conclusion{$P(A_{2n}) = \binom{2n}{n} \frac{1}{2^{2n}} = \binom{2n}{n} \frac{1}{4^{n}} $}
\end{enumerate}

\item 

\begin{lstlisting}
def sauts(n):
    puce=0
    for i in range(n):
        p=random()
        if  p>1/2 :
            puce=puce+1
        else:
            puce=puce-1
    return(puce)
\end{lstlisting}
\item 

\begin{lstlisting}
def A(n):
    x=sauts(n)
    if x==0:
    	return(True)
    else:
    	return(False)
\end{lstlisting}

\item 
\begin{lstlisting}
def approx(n):
	c=0
	for i in range(10000):
		if A(n):
			c=c+1
	return(c/10000)
\end{lstlisting}

\item 
\begin{lstlisting}
def tempsdarret():
	puce=0
	n=0
	while puce!=0 and n!=0:
		p=random()
		if  p>1/2 :
			puce=puce+1
		else:
			puce=puce-1
		n=n+1
	return(n)
\end{lstlisting}
\end{enumerate}




\end{correction}

\vspace{0.5cm}
\begin{exercice}
Soit $a\in ]-1,1[. $ On suppose l'existence d'une application $f$, continue sur $\R$, telle que :
$$\forall x\in \R, \quad f(x) =\int_0^{ax} f(t)dt.$$
\begin{enumerate}
\item Calcul des dérivées successives de $f$. 
\begin{enumerate}
\item Justifier l'existence d'une primitive $F$ de $f$ sur $\R$ et écrire alors, pour tout nombre réel $x,$
$f(x)$ en fonction de $x, a$ et $F$. 
%En déduire une expression de $f(x)$  en fonction de $x, a$ et $F$. 
\item Justifier la dérivabilité de $f$ sur $\R$ et exprimer, pour tout nombre réel $x$, $f'(x)$  en fonction de $x, a$ et $f$. 
\item Démontrer que $f$ est de classe $\cC^\infty $ sur $\R$ et que pour tout nombre entier naturel $n,$ on a 
$$\forall x\in \R \quad  f^{(n)} (x) =a^{n(n+1)/2} f(a^nx).$$
\item En déduire, pour tout nombre entier naturel $n$ la valeur de $f^{(n)} (0)$. 
\end{enumerate}
\item Démontrer que, pour tout nombre réel $x$ et tout nombre entier $n$, on a :
$$f(x) = \int_0^x \frac{(x-t)^n}{n!} f^{(n+1)} (t) dt.$$
\footnotesize{ \textit{On pourra faire une récurrence et utiliser une intégration par parties}}
\normalsize{}
\item Soit $A$ un nombre réel strictement positif. 
\begin{enumerate}
\item Justifier l'existence d'un nombre réel positif ou nul $M$ tel que : 
$$\forall x\in [-A,A], \quad |f(x) | \leq M$$
et en déduire que pour tout nombre entier naturel $n$, on a :
$$\forall x\in [-A,A], \quad |f^{(n)}(x) | \leq M$$.
\item Soit $x$ un nombre réel apartenant à $[-A,A].$ Démontrer que, pour tout nombre entier naturel $n$, on  a 
$$|f(x)| \leq M\frac{A^{n+1}}{(n+1)!}.$$
\item En déduire que $f(x) = 0$ pour tout $x\in [-A,A]$
\item Que peut-on en déduire sur la fonction $f$ ? 
\end{enumerate} 
\end{enumerate}
\end{exercice}



\begin{correction}
\begin{enumerate}
\item 
\begin{enumerate}
\item $f$ est continue sur $\R$ donc admet une primitive, notée $F$. On a 
par définition de l'intégrale $f(x) = F(ax) - F(0)$. 
\item Une primitive est par définiton une fonction de classe $\cC^1$ donc $F$ est de classe $\cC^1$ et finalemtn $f$ est de classe $\cC^1$. On a 
$$f'(x) = a F'(ax) = af(ax).$$ 
\item On pose $P(n)$ : " $f$ est de classe $C^n$ et $\forall x\in \R \quad  f^{(n)} (x) =a^{n(n+1)/2} f(a^nx)$ ".
\begin{itemize}
\item $P(0)$ est vraie par hypothèse. 
\item Supposons qu'il existe $n\in \N$ tel que $P(n)$ soit vraie. On a alors $f$ de classe $\cC^n$, et $\forall x\in \R \quad  f^{(n)} (x) =a^{n(n+1)/2} f(a^nx)$. 
Or comme $f$ est de classe $\cC^1$ d'après la question précédente, on a alors que $ f^{(n)}$ est de classe $\cC^1$ c'est à dire $f$ de classe $\cC^{n+1}$. Enfin 
$\forall x\in \R$,   \begin{align*}
 f^{(n+1)} (x) &= a^{n(n+1)/2}  f'(a^nx)\\ 
 						&= a^{n(n+1)/2+n} a f(a a^n x) \quad \text{d'après la question précédente}\\
 						&= a^{n(n+1)/2+n+1} f(a^{n+1} x) \\
 						&= a^{(n+1)(n+2)/2} f(a^{n+1} x) \\
\end{align*} 
\item On a montré par récurrence que pour tout $n\in \N$, $f$ est de classe $\cC^n$. Elle est donc de classe $\cC^\infty$ et $\forall x\in \R \quad  f^{(n)} (x) =a^{n(n+1)/2} f(a^nx)$.
\end{itemize}
\item  On a donc $f^{(n)} (0) = a^{n(n+1)/2 } f(0) $. Or $f(0) = \int_0^{0} f(t)df =0$
Donc pour tout $n\in \N$ $$f^{(n)} (0) =0$$
\end{enumerate}
\item On montre le résultat par récurrence. On pose pour tout nombre réel $x$ et tout nombre entier $n$, la proposition 
$P(n) : "f(x) = \int_0^x \frac{(x-t)^n}{n!} f^{(n+1)} (t) dt."$
\begin{itemize}
\item Réécrivons $P(0)$. On a $P(0) : "f(x) = \int_0^x \frac{(x-t)^0}{0!} f^{(0+1)} (t) dt. " $, c'est à dire : $f(x) = \int_0^x  f'(t) dt.$ Ce qui est vrai par définition de l'intégrale. 
\item Supposons qu'il existe $n\in \N$  tel que $P(n)$ soit vraie. On  a alors pour tout nombre réel $x$, $f(x) = \int_0^x \frac{(x-t)^n}{n!} f^{(n+1)} (t) dt$. 
Comme suggérer par l'énoncé on fait une IPP. On pose 
\begin{minipage}{0.4 \textwidth}
$u(t) = f^{(n+1)}(t)$\\
$v(t) = -\frac{(x-t)^{n+1}}{(n+1)!}$
\end{minipage}
\begin{minipage}{0.4 \textwidth}
$u'(t) = f^{(n+2)}(t)$\\
$v'(t) = \frac{(x-t)^{n}}{n!}$
\end{minipage}
On a donc 
\begin{align*}
f(x)&=\left[ \frac{(x-t)^{n+1}}{(n+1)!}  f^{(n+1)}(t)\right]_0^x - \int_0^x - \frac{(x-t)^{n+1}}{(n+1)!}f^{(n+2)}(t)dt\\
\end{align*}
Le crochet vaut $\frac{(x-x)^{n+1}}{(n+1)!}  f^{(n+1)}(x)- \frac{(x-0)^{n+1}}{(n+1)!}  f^{(n+1)}(0)$ les deux termes valent 0 (le second à l'aide de la question précédente). On obtient bien 
 \begin{align*}
f(x)&=  \int_0^x  \frac{(x-t)^{n+1}}{(n+1)!}f^{(n+2)}(t)dt
\end{align*}
\item Par récurrence la propriété est vraie pour tout $n\in \N$. 
\end{itemize}
\item \begin{enumerate}
\item Soit $A>0$. Comme $f$ est continue et $[-A,A]$ est un segment, le théorème de continuité sur un segment assure que $f$ est bornée et atteint ses bornes. Donc il existe $M>0$ tel que pour tout $x\in [-A,A]$, $|f(x)|\leq M$.

D'après 1c) on sait que pour tout $x\in \R$, $f^{(n)} (x) = a^{n(n+1)/2} f(a^n x)$ En particulier $|f^{(n)} (x)| =|a^{n(n+1)/2}| | f(a^n x)|$ 
Or comme $|a|<1$ , $|a^{n(n+1)/2}|  \leq 1$ et pour tout $x\in [-A,A]$, on a $a^n x \in  [-A,A]$ et ainsi $ | f(a^n x)| \leq M$. Au final pour tout  $x\in [-A,A]$ : 
$$|f^{(n)} (x)|\leq M.$$


\item  D'après la question 2 on a : 
$\ddp f(x) = \int_0^x \frac{(x-t)^n}{n!} f^{(n+1)} (t) dt$, donc $|f(x)| \leq\ddp   \int_0^x \left|  \frac{(x-t)^n}{n!} f^{(n+1)} (t) \right| dt$ c'est l'inégatilité triangulaire sur les intégrales. On majore maintenant $\left|  f^{(n+1)} (t) \right| $ à l'aide de la question précédente, on obtient pour tout $x\in [-A,A]$ :
$$f(x) \leq  \ddp M  \int_0^x \left|  \frac{(x-t)^n}{n!}  \right| dt.$$
Donc $f(x) \leq \ddp M \left[ \frac{|(x-t)|^{n+1}}{(n+1)!}\right]_0^x \leq M  \frac{|x|^{n+1}}{(n+1)!}$ Or comme $x\in [-A,A]$ on a bien : 
$$|f(x)|\leq M\frac{A^{n+1}}{(n+1)!}$$
\item Par croissance comparée, en passant à la limite on a $$\lim_{n\tv \infty} \frac{A^{n+1}}{(n+1)!} = 0$$
Ainsi le théorème des gendarmes assure que pour tout $x\in [-A,A]$ on a 
$$\lim_{n\tv \infty} f(x) = 0.$$ Evidemment $f(x) $ ne dépend pas de $n$ donc par unicité de la limite $f(x) = 0$

Ceci étant vrai pour tout $x \in [-A,A]$ et comme $A$ est arbitraire, ceci est vrai pour tout $x \in \R$. 

$$f\equiv 0$$

\end{enumerate}

\end{enumerate}

\end{correction}

\end{document}