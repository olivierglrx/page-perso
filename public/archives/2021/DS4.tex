\documentclass[a4paper, 11pt,reqno]{article}
\input{/Users/olivierglorieux/Desktop/BCPST/2020:2021/preambule.tex}
\newif\ifshow
\showfalse
\input{/Users/olivierglorieux/Desktop/BCPST/2021:2022/ifshow.tex}

\geometry{hmargin=3.0cm, vmargin=3.5cm}
\newenvironment{amatrix}[1]{%
  \left(\begin{array}{@{}*{#1}{c}|c@{}}
}{%
  \end{array}\right)
}

\author{Olivier Glorieux}


\begin{document}

\title{DS 4\\
\Large{Durée 3h00}
}

\vspace{1cm}
\begin{center}

\begin{description}
\item$\bullet$ Les calculatrices sont \underline{interdites} durant les cours, TD et \emph{a fortiori} durant les DS de mathématiques. \\

\item $\bullet $ Si vous pensez avoir découvert une erreur, indiquez-le clairement sur la copie et justifiez les initiatives que vous êtes amenés à prendre. \\

\item $\bullet$ Une grande attention sera apportée à la clarté de la rédaction et à la présentations des solutions. (Inscrivez clairement en titre le numéro de l'exercice, vous pouvez aussi encadrer les réponses finales.)  \\

\item $\bullet$ Vérifiez vos résultats. \\

\item $\bullet$ Le résultat d'une question peut être admis et utilisé pour traiter les questions suivantes en le signalant explicitement sur la copie. 
\end{description}

\end{center} 
\vspace{1cm}


%\begin{exercice}
%Résoudre l'inéquation :
%$$\frac{4x-3}{x-2}\leq \frac{1}{x+1}$$
%
%En déduire les solutions  sur $\R$ puis sur $[0,2\pi[$ de :
%$$\frac{4\cos(2t)-3}{\cos(2t)-2}\leq \frac{1}{\cos(2t)+1}$$
%\end{exercice}


\newpage


\begin{exercice}
\begin{enumerate}
\item Donner la définition d'une fonction injective. En donner un exemple et un contre-exemple.
\item  Exprimer à l'aide de quantificateurs le fait qu'une fonction $f$ soit majorée sur $\R$.
\item Donner la négation de la proposition suivante : \og Si il pleut alors je prends mon parapluie.\fg
\item Donner la contraposée de l'implication suivante : \og Il pleut et il y a du soleil \fg  $\implique $ \og il y a un arc-en-ciel . \fg  
\item Que vaut la matrice identité de taille 3 ?
\item Donner la définition d'une matrice symétrique. En donner un exemple de taille 3 (qui n'est pas la matrice identité ni la matrice nulle).
\item Donner un exemple d'une matrice de taille 2, telle que $A\neq 0$ mais $A^2=0$.
\end{enumerate}
\end{exercice}
\begin{correction}
\begin{enumerate}
\item Cf cours
\item $\exists M\in \R\, \forall x\in \R, f(x)\leq M$
\item 'Il pleut  et je ne prends pas mon parapluie'
\item 'Il n'y a pas d'arc-en-ciel' $\implique$ ' Il ne pleut pas ou il n'y a pas de soleil'
\item $\Id=\left(\begin{array}{ccc}
1&0&0\\
0&1&0\\
0&0&1
\end{array}\right)$
\item Une matrice $A$ est symétrique si  $A=A^T$ où $A^T$ désigne la transposée de $A$. 
ex $A=\left(\begin{array}{ccc}
1&2&0\\
2&1&0\\
0&0&1
\end{array}\right)$
\item $A=\left(\begin{array}{cc}
0&1\\
0&0
\end{array}\right)$ fonctionne. 
\end{enumerate}
\end{correction}
\vspace{1cm}

\begin{exercice}
On regarde une horloge comme un cercle trigonométrique... Ainsi quand il est midi pile, l'aiguille des heures est à $\frac{\pi}{2}$ et l'aiguille des minutes est aussi à $\frac{\pi}{2}$ (à $2\pi$ près évidemment.) Quand il est $15:00$, l'aiguille des heures est à $0$ tandis que celle des minutes est à $\frac{\pi}{2}$. 

\begin{center}
\includegraphics[scale=0.5]{12.png}
\includegraphics[scale=0.5]{15.png}
\includegraphics[scale=0.5]{1640.png}
\end{center}


A quelle place se trouve l'aiguille des heures à 16H00 et à 16h40 ? (on justifiera la réponse proprement pour 16h40) 
\end{exercice}
\begin{correction}
A 16h00, il est 1h de plus que 15H00. 1H correspond à 1/12 de tour du cercle, donc $\frac{1}{12}2\pi = \frac{\pi}{6}$. Comme le sens trigonométrique est opposé à celui du sens horaire,
\conclusion{l'aiguille des heures est  à $\frac{-\pi}{6}$ à 16H00.}

Pour 16h40, on a ajouté $40/60=2/3$ d'heures à 16h00. L'aiguille des heures a donc avancé de $\frac{2}{3} \frac{\pi}{6} = \frac{\pi}{9}$, elle est donc à  
\conclusion{$-(\frac{\pi}{6} \frac{\pi}{9})= - \frac{5\pi}{18}$}


 

\end{correction}
\vspace{1cm}
\begin{exercice}
\begin{enumerate}
\item Résoudre l'inéquation d'inconnue $y$ suivante : 
$$\frac{y-3}{2y-3}\leq 2y \quad (E_1)$$

\item En déduire les solutions sur $\R$ de l'inéquation d'inconnue $X$  : 
$$\frac{\sin^2(X)-3}{2\sin^2(X) -3} \leq 2 \sin^2(X)\quad (E_2)$$

\item Finalement donner les solutions sur $[0,2\pi[ $ de l'inéquation d'inconnue $x$ : 
$$\frac{\sin^2(2x+\frac{\pi}{6})-3}{2\sin^2(2x+\frac{\pi}{6}) -3} \leq 2 \sin^2(2x+\frac{\pi}{6}) \quad (E_3)$$
\end{enumerate}

\end{exercice}

\begin{correction}
\begin{enumerate}
\item 
$$\begin{array}{lrl}
&\frac{y-3}{2y-3}&\leq 2y\\
\equivaut &0 &\leq 2y - \frac{y-3}{2y-3}\\
\equivaut &0 &\leq \frac{4y^2-7y+3}{2y-3}
\end{array}$$
$4y^2-7y+3$ admet pour racines : $y_0 = 1$ et $y_1 =\frac{3}{4}$, donc 
$$\begin{array}{lrl}
&\frac{y-3}{2y-3}&\leq 2y\\
\equivaut &0&\leq \frac{4(y-1)(y-\frac{3}{4})}{2(y-\frac{3}{2})}
\end{array}$$
Donc les solutions de $(E_1)$ sont 
\conclusion{ $\cS_1 = \left[ \frac{3}{4}, 1\right] \cup \left] \frac{3}{2}, +\infty\right[ $} 


\item $X$ est solutions de $(E_2)$ si et seulement si : 
$$\sin^2(X) \in \left[ \frac{3}{4}, 1\right] \cup \left] \frac{3}{2}, +\infty\right[ $$
Comme pour tout $X\in \R$,  $\sin(X) \in [-1,1]$, ceci équivaut à 
$$sin^2(X) \in \left[ \frac{3}{4}, 1\right] $$
c'est-à-dire : $\sin^2(X) \geq \frac{3}{4}$, soit 
$\left(\sin(X) -\frac{\sqrt{3}}{2}\right)\left(\sin(X) +\frac{\sqrt{3}}{2}\right)\geq 0$ 
On obtient donc 
$$\sin(X) \in  \left[ -1, \frac{-\sqrt{3}}{2},\right] \cup  \left[ \frac{\sqrt3}{4}, 1\right] $$
On a  d'une part $\sin(X) \leq  \frac{-\sqrt{3}}{2} \equivaut X \ddp \in \bigcup_{k\in \Z} \left[ \frac{5\pi}{6} +2k\pi,\frac{7\pi}{6} +2k\pi \right] $
et d'autre part 
$\sin(X) \geq  \frac{\sqrt{3}}{2} \equivaut X \in \ddp \bigcup_{k\in \Z} \left[ \frac{-\pi}{6} +2k\pi,\frac{\pi}{6} +2k\pi \right] $

Ainsi les solutions de $(E_2)$ sont
\conclusion{  $\cS_2 =\ddp   \bigcup_{k\in \Z} \left[ \frac{5\pi}{6} +2k\pi,\frac{7\pi}{6} +2k\pi \right]  \cup \left[ \frac{-\pi}{6} +2k\pi,\frac{\pi}{6} +2k\pi \right]$  } 


\item $x$ est solution de $(E_3)$ si et seulement si 
$$2x+\frac{\pi}{6}\in \ddp   \bigcup_{k\in \Z} \left[ \frac{5\pi}{6} +2k\pi,\frac{7\pi}{6} +2k\pi \right]  \cup \left[ \frac{-\pi}{6} +2k\pi,\frac{\pi}{6} +2k\pi \right]$$ 
C'est-à-dire 

$$2x\in \ddp   \bigcup_{k\in \Z} \left[ \frac{4\pi}{6} +2k\pi,\frac{6\pi}{6} +2k\pi \right]  \cup \left[ \frac{-2\pi}{6} +2k\pi, +2k\pi \right]$$ 
Soit encore : 
$$x\in \ddp   \bigcup_{k\in \Z} \left[ \frac{2\pi}{9} +k\pi,\frac{\pi}{2} +k\pi \right]  \cup \left[ \frac{-\pi}{6} +k\pi, k\pi \right]$$
Maintenant on se restreint à l'intervalle $[0,2\pi[$, les solutions sont donc :
\conclusion{ $\cS_3= \left[ \frac{2\pi}{9} ,\frac{\pi}{2} \right]\cup \left[ \frac{11\pi}{9} ,\frac{3\pi}{2}  \right]   \cup\left\{0\right\} \cup \left[ \frac{5\pi}{6}, \pi \right]\cup \left[ \frac{11\pi}{6} , 2\pi \right[      $}


\end{enumerate}



\end{correction}

\begin{exercice}
Soit $M$ la matrice : 
$$M=\left( \begin{array}{ccc}
2 &1& 0\\
0 &1 & 0  \\
 -1&0&1
\end{array}\right) $$

\begin{enumerate}
\item  Résoudre le système $MX=\lambda X$ d'inconnue $X =\left(
\begin{array}{c}
x\\
y\\
z
\end{array}
 \right)$ où $\lambda$ est un paramètre réel. 
\item Calculer $(M- \Id)^2$. Donner son rang.

 \item Soit $e_1= \left(
\begin{array}{c}
1\\
0\\
-1
\end{array}
 \right)$,  $e_2= \left(
\begin{array}{c}
0\\
0\\
1
\end{array}
 \right)$, et  $e_3= \left(
\begin{array}{c}
1\\
-1\\
0
\end{array}
 \right)$.
Exprimer  $Me_1, Me_2$ en fonction de $e_1, e_2$.
\item Montrer qu'il existe $(\alpha, \beta)\in \R^2$ tel que $M e_3 = \alpha e_2 +\beta e_3$.
 
\item Soit $P= \left(
\begin{array}{ccc}
1&0&1\\
0&0&-1\\
-1&1&0
\end{array}
 \right)$ 
 
 Montrer que $P$ est inversible et calculer son inverse. 
 \item Soit $T=P^{-1}MP$. Calculer $T$. 
 \item Montrer par récurrence que pour tout $n\in \N$: 
 $$T^n = P^{-1}M^n P$$
 \item Montrer qu'il existe une matrice diagonale $D$ et une matrice $N$ telles que 
 $$T =D+N \quadet ND=DN$$
\item Montrer que $N^2=0$
\item En déduire  que $T^n = D^n +nND^{n-1}$.
\item Finalement déterminer la valeur de $M^n$ en fonction de $n$.
\end{enumerate}
\end{exercice}
\begin{correction}
\begin{enumerate}
\item
$$MX=\lambda X \equivaut   \left( \begin{array}{c}
2x +y  \\
 y \\
 -x +z
\end{array}\right) = \left(
\begin{array}{c}
\lambda x\\
\lambda y\\
\lambda z
\end{array} \right)$$ 

$$\left\{ \begin{array}{ccccc}
2x &+y& & =&\lambda x \\
 &y & & =& \lambda y \\
 -x& &+z&=&\lambda z
\end{array}\right. 
\equivaut \left\{ \begin{array}{ccccc}
(2-\lambda)x &+y& & =&0 \\
 &(1-\lambda)y & & =& 0 \\
 -x& &+(1-\lambda)z&=&0
\end{array}\right. 
$$ 
 En échangeant les lignes et les colonnes on peut voir que le système est déjà échelonné.
$L_3\leftarrow L_1, L_2 \leftarrow _3, L_1\leftarrow L_2$
$$MX=\lambda X 
\equivaut  \left\{ \begin{array}{ccccc}
 -x& &+(1-\lambda)z&=&0\\
(2-\lambda)x &+y& & =&0 \\
 &(1-\lambda)y & & =& 0 
\end{array}\right.$$
$ C_3\leftarrow C_1, C_2 \leftarrow C_3, C_1\leftarrow C_2$
$$
\equivaut \left\{ \begin{array}{ccccc}
 (1-\lambda)z&-x& &=&0\\
 &(2-\lambda)x&+y & =&0 \\
 &  & (1-\lambda)y& =& 0 
\end{array}\right.$$

Si $\lambda \notin \{ 1,2\} $ alors le système est de rang 3, il est donc de Cramer et l'unique solution est 
\conclusion{ $\cS= \{ (0,0,0)\}$}

Si $\lambda =1$, le système est équivalent à 
$$\left\{ \begin{array}{cccc}
 -x& &=&0\\
 (2-1)x&+y & =&0 \\
   & 0& =& 0 
\end{array}\right. \equivaut \left\{ \begin{array}{cc}
 x& =0\\
 y&  =0
\end{array}\right.$$
Le système est de rang 2. L'ensemble des solutions est 
\conclusion{ $\cS= \{ (0,0,z) \, |\, z\in \R\}$}

Si $\lambda =2$, le système est équivalent à 
$$\left\{ \begin{array}{ccccc}
(1-2)z&-x& &=&0\\
 & 0 &+y & =&0 \\
 &  & (1-2)y& =& 0 
\end{array}\right. \equivaut \left\{ \begin{array}{cccc}
-z&-x& &=0\\
 &  &y  &=0 \\
 &  & y &= 0 
\end{array}\right.\equivaut \left\{ \begin{array}{cl}
x &=-z\\
  y  &=0 
\end{array}\right.$$
Le système est de rang 2. L'ensemble des solutions est 
\conclusion{ $\cS= \{ (-z,0,z) \, |\, z\in \R\}$}

\item $M-\Id= \left( \begin{array}{ccc}
1 &1& 0\\
0 &0& 0  \\
 -1&0&0
\end{array}\right) $

Donc \conclusion{$(M-\Id)^2= \left( \begin{array}{ccc}
1 &1& 0\\
0 &0& 0  \\
 -1&-1&0
\end{array}\right) $}

Le système associé est 

$\left\{  \begin{array}{ccr}
x &+y&  =0\\
 & & 0  =0\\
 -x&-y&=0
\end{array}\right. \equivaut \left\{  \begin{array}{cr}
x +y&  =0\\
\end{array}\right. $
Il est de rang 1. Donc
\conclusion{$(M-\Id)^2$ est de rang $1$}
\item 
Le calcul montre que $Me_1 =2e_1$ et  $Me_2=e_2$
\item Le calcul montre que 
$Me_3 =  \left( \begin{array}{c}
1\\
-1\\
-1\\
\end{array}\right) = \left( \begin{array}{c}
1\\
-1\\
0\\
\end{array}\right) - \left( \begin{array}{c}
0\\
\\
1\\
\end{array}\right)  e_3-e_2 $


Ainsi on peut prendre 
\conclusion{$\alpha =-1$ et $\beta =1$}

\item  On considère la matrice augmentée : 
$\left(\begin{array}{ccc|ccc}  
1&0&1 & 1&0&0 \\
0&0&-1& 0&1&0 \\
-1&1&0& 0&0&1 
\end{array}\right)$

$L_3\leftarrow L_3+L_1$  donnent
$$\left(\begin{array}{ccc|ccc}  
1&0&1 & 1&0&0 \\
0&0&-1& 0&1&0 \\
0&1&1& 1&0&1 
\end{array}\right)$$
$L_1\leftarrow L_1+L_2$ et $L_3\leftarrow L_3+L_2$
donne 
$$\left(\begin{array}{ccc|ccc}  
1&0&0 & 1&1&0 \\
0&0&-1& 0&1&0 \\
0&1&0& 1&1&1 
\end{array}\right)$$

$L_2\leftarrow -L_2$
donne 
$$\left(\begin{array}{ccc|ccc}  
1&0&0 & 1&1&0 \\
0&0&1& 0&-1&0 \\
0&1&0& 1&0&1 
\end{array}\right)$$
Enfin 
$L_2\leftrightarrow L_3$
donne 
$$\left(\begin{array}{ccc|ccc}  
1&0&0 & 1&1&0 \\
0&1&0& 1&1&1 \\
0&0&1& 0&-1&0 
\end{array}\right)$$

\conclusion{ $P$ est inversible d'inverse $\left(\begin{array}{ccc}  
1&1&0 \\
 1&1&1 \\
0&-1&0 
\end{array}\right)$}

\item Le calcul donne  
\conclusion{ $T=\left(\begin{array}{ccc}  
2&0&0 \\
0 &1&-1 \\
0&0&1 
\end{array}\right)$}
(sur une copie, le produit intermédiaire $MP$ serait apprécié)

\item 


(CF ex 6-3 du DM de Noël) \\

On pose $P(n) : "T^n =P^{-1} M^n P"$
\begin{itemize}
\item[Initialisation] 
$T^1 =T$ et $P^{-1} M^1 P= P^{-1} M P=T$ d'après la définition de $T$.
Donc $P(1) $ est vrai. 

\item[Hérédité] On suppose qu'il existe $n\in \N$ tel que $P(n)$ soit vraie. 
On a alors 
\begin{align*}
 (T)^{n+1}&=  T^n  T
\end{align*}
et donc par Hypothése de récurrence : 
\begin{align*}
 T^{n+1}&=  (P^{-1}M^n P )  (P^{-1}M P )\\
 							&=  (P^{-1}M^n P  P^{-1}M P )\\
 							&=  (P^{-1}M^n \Id M P )\\
 							&=  (P^{-1}M^n M P )\\
 							&=  (P^{-1}M^{n+1} P )
\end{align*}
\item[Conclusion] $P(n)$ est vraie pour tout $n$. 




\end{itemize}
\item On a 
$T= \left(\begin{array}{ccc}  
2&0&0 \\
0 &1&-1 \\
0&0&1 
\end{array}\right)  = \left(\begin{array}{ccc}  
2&0&0 \\
 0&1&0 \\
0&0&1 
\end{array}\right) + \left(\begin{array}{ccc}  
0&0&0 \\
0 &0&-1 \\
0&0&0 
\end{array}\right)  $  

On pose $D= \left(\begin{array}{ccc}  
2&0&0 \\
 0&1&0 \\
0&0&1 
\end{array}\right) $ et $N= \left(\begin{array}{ccc}  
0&0&0 \\
 0&0&-1 \\
0&0&0 
\end{array}\right)  $ 
On a bien $T =D+N$ et  le calcul donne 
$DN = \left(\begin{array}{ccc}  
0&0&0 \\
0 &0&-1 \\
0&0&0 
\end{array}\right) =DN $ 


\item C'est un calcul. La question \og normale\fg\,  devrait être \og Calculer $N^2$ \fg \,  , mais ne permet pas de faire la question suivante si on n'a pas trouvé la forme de $N$. 

\item Solution 1 : On peut appliquer le binome de Newton à $T= D+N$   car $D$ et $N$ commutent. On a alors 

$$T^n =\sum_{k=0}^n \binom{n}{k} N^k D^{n-k}$$
Comme pour tout $k\geq 2$, $N^2=0$ il reste dans cette somme seulement les termes $k=0$ et $k=1$. On obtient donc 
\begin{align*}
T^n  &= \binom{n}{0} N^0 D^{n-0}+ \binom{n}{1} N^1 D^{n-1}\\
		&=D^n + nND^{n-1}
\end{align*}






Solution 2: 

On pose $P(n) : \og  T^n =D^n +n D^{n-1} N \fg$

\begin{itemize}
\item \underline{Initialisation }
$T^1 =T$ et $D^1+1D^0 N = D^1 +\Id N=D+N =T$ d'après la définition de $D,N$.
Donc $P(1) $ est vrai. 

\item \underline{Hérédité} On suppose qu'il existe $n\in \N$ tel que $P(n)$ soit vraie. 
On a alors 
\begin{align*}
 (T)^{n+1}&=  T^n  T
\end{align*}
et donc par Hypothése de récurrence : 
\begin{align*}
 T^{n+1}&= (D^n +nD^{n-1} N)(D+N)\\
 							&=  D^n D +n D^{n-1} N D + D^n N + nD^{n-1}N^2\\
\end{align*}
Comme $ND=DN$ on a $D^{n-1} N D= D^{n-1} DN  =D^{n} N$.  on a par ailleurs $N^2=0$ donc 

\begin{align*}
 T^{n+1}&=D^{n+1} +D^n N +nD^n N\\
 			&=D^{n+1} + (n+1) D^{(n+1)-1} N 
\end{align*}
Ainsi la propriété est héréditaire. 

\item \underline{Conclusion} $P(n)$ est vraie pour tout $n$. 
\end{itemize}

\item On a d'après la question 7 
$$M^n = P T^n P^{-1}$$
et d'après la question précédente : 
$$T^n = D^{n} + n D^{n-1} N  =\left(\begin{array}{ccc}  
2^n&0&0 \\
0 &1&-n \\
0&0&1 
\end{array}\right)$$
Le calcul donne 

$T^n P^{-1} = \left(\begin{array}{ccc}  
2^n&2^n&0 \\
1 &1+n&1 \\
0&-1&0 
\end{array}\right)$
et 
\conclusion{
 $M^n=\left(\begin{array}{ccc}  
2^n&2^n-1&0 \\
0&1&0 \\
-2^n+1&-2^n+1+n&1 
\end{array}\right)$

}




\end{enumerate}
\end{correction}


\begin{exercice}[Ensemble de Mandelbrot]
Soit $\suite{z}$ la suite définie par $z_0=0$ et 
$$z_{n+1} =z_n^2 +c$$
où $c\in \bC$ est un complexe. 

Selon la valeur de $c$, il y a deux possibilités : soit $\suite{z}$ reste bornée, soit son module tends vers l'infini. Le but de ce problème est d'écrire un algorithme qui permet de tracer l'ensemble des $c$ pour lesquels la suite $\suite{z}$ reste bornée. Cette ensemble s'appelle l'ensemble de Mandelbrot, que l'on note $\bM$:
$$\bM =  \{ c\in \bC\, |\,  \text{ La suite $\suite{z}$ définie par $z_0=0$ et$z_{n+1} =z_n^2 +c$ est bornée } \} $$
\begin{enumerate}
\item Que vaut la suite $\suite{z}$ pour $c=0$. Est ce que $c=0$ appartient à l'ensemble de Mandelbrot ?

\item Que valent les premières valeurs ($n=0,1,2,3,4$) de la suite $\suite{z}$ pour $c=i$.  A votre avis est-ce-que $c=i$ appartient à l'ensemble de Mandelbrot $\bM$ ? 
\item Même question pour $c=1+i$ (pour $n=0,1,2,3$).
\item Ecrire une fonction Python \texttt{suite\_z} qui prend en argument un entier $n\in \N$ et un complexe $c\in \bC$ et qui retourne la valeur de $z_n$.
\item On peut montrer que $c$ appartient à $\bM$ si et seulement pour tout $n\in \N$,  $|z_n|<2$.
On suppose pour simplifier qu'un nombre $c$ appartient  à $\bM$ si et seulement si  pour tout $n\in \intent{0,100},   \, |z_n|<2$.
Ecrire une fonction \texttt{verif} qui prend en argument un nombre complexe $c$ et retourne \texttt{True} si $c$ appartient à l'ensemble de Mandelbrot et \texttt{False} sinon. 
\item Ecrire une fonction \texttt{tracer} qui prend en argument deux réels $(x,y)$ et qui trace le point $(x,y)$ sur un graphique si le point d'affixe $x+iy$ appartient à l'ensemble de Mandelbrot. 
\item Ecrire un script python qui teste  si les points de coordonnées $\left( \frac{i}{100}, \frac{j}{100}\right)$ pour $i, j\in \intent{-100,100}$ appartiennent  à $\bM$ et les trace le cas échéant. 
\end{enumerate}
\begin{center}
\includegraphics[scale=0.4]{mandelbrot.png}
\end{center}

\end{exercice}

On pourra utiliser les bibliothéques \texttt{matplolib.pyplot} et  \texttt{numpy}.
On rappelle la définition des fonctions suivantes : 
\begin{itemize}
\item La fonction \texttt{abs(z)} donne la valeur absolue ou le module de $z$ 
\item On peut obtenir le conjugué d'un nombre complexe $z$ en écrivant \texttt{z.conjugate()}
%\item La fonction \texttt{sqrt(x)} donne la racine d'un nombre réel positif $x$
\item La fonction \texttt{plot} de la bibliothéque \texttt{matplolib.pyplot} permet de  marquer sur un graphique un point dont on donne les  coordonnées $(a,b)$ sous forme de croix grace à  \texttt{plot(a,b,'x')} où $a$ et $b$ sont des nombres réels, 
\item  La fonction \texttt{show} de la bibliothéque \texttt{matplolib.pyplot}   permet d'afficher le graphique tracé. 
\item La fonction \texttt{linspace(a,b,n)} de la bibliothéqe \texttt{numpy} retourne un tableau (ligne) de $n$ nombres espacés linérairement entre $a$ et $b$.
\item La fonction \texttt{size(M)} de  la bibliothéqe \texttt{numpy} retourne la taille du tableau \texttt{M}
\item La fonction \texttt{dot(M,N)} de  la bibliothéqe \texttt{numpy} retourne le produit matricielle entre $M$ et $N$.
\item  La fonction \texttt{transpose(M)} de  la bibliothéqe \texttt{numpy} retourne le  tableau transposé de  \texttt{M}
\end{itemize}


\begin{correction}
\begin{enumerate}
\item Pour $c=0$ la suite $\suite{z}$ est constante égale à $0$. $0$ appartient donc à l'ensemble de Mandelbrot. 
\item Pour $c=i$, $z_0=0$, $z_1=i$, $z_2= i^2+i=-1+i$, $z_3= (-1+i)^2 +i = -i$, $z_4=(-i)^2+i = -1+i$. La suite semble périodique et donc le module est borné. Ainsi $c=i$  appartient donc à l'ensemble de Mandelbrot. 

\item Pour $c=1+i$ : $z_0=0$, $z_1=1+i$, $z_2=(1+i)^2+1+i=1+3i$, $z_3= (1+3i)^2 +1+i = -7+7i$, $z_4=(-7+7i)^2+1+i= 49(-1+i)^2+1+i = 49(-2i) +1+i = 1-97i$. Le module semble tendre vers l'infini. 
$c=1+i$ n'appartient donc pas à l'ensemble de Mandelbrot. 


\end{enumerate}
\end{correction}


\end{document}