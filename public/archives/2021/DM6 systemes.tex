\documentclass[a4paper, 11pt,reqno]{article}
\input{/Users/olivierglorieux/Desktop/BCPST/2020:2021/preambule.tex}
\newif\ifshow
\showfalse
\input{/Users/olivierglorieux/Desktop/BCPST/2021:2022/ifshow.tex}
\geometry{hmargin=1.0cm,vmargin=2.5cm }

\author{Olivier Glorieux}


\begin{document}

\title{DM6 \\
\small{Copie acceptée jusqu'au mercredi 17 Novembre} \\
A faire avant le vendredi 19 novembre
}




%%%%%%%%%%%%%%%%%%%%%%%%%


\begin{exercice}
Soit $\lambda \in \R$. On considère le système suivant 
$$(S_\lambda)\quad  \left\{ \begin{array}{ccc}
2x +2y & =& \lambda x\\
x +3y  & =& \lambda y 
\end{array}\right. $$

\begin{enumerate}
\item Déterminer $\Sigma$ l'ensemble des réels $\lambda$ pour lequel ce système \underline{n'est pas} de Cramer. 
\item Pour $\lambda \in \Sigma$, résoudre $S_\lambda$
\item Quelle est la solution si $\lambda \notin \Sigma$. 
\end{enumerate}
\end{exercice}


\begin{correction}
\begin{enumerate}
\item  On met le système sous forme échelonné
$$(S_\lambda)\equivaut  \left\{ \begin{array}{ccc}
(2-\lambda)x +2y & =& 0\\
x +(3-\lambda)y  & =& 0
\end{array}\right.
\equivaut 
\left\{ \begin{array}{ccc}
x +(3-\lambda)y  & =& 0\\
(2-\lambda)x +2y & =& 0
\end{array}\right.
\equivaut 
\left\{ \begin{array}{rcc}
x +(3-\lambda)y  & =& 0\\
+2y -  (3-\lambda) (2-\lambda) y& =& 0
\end{array}\right.
 $$
 D'où 
 
 $$(S_\lambda)\equivaut  \left\{ \begin{array}{rcc}
x +(3-\lambda)y  & =& 0\\
  (-\lambda^2 +5\lambda -4) y& =& 0
\end{array}\right.$$
 
Le système n'est pas de Cramer si  $ (\lambda^2 +5\lambda -4) =0$, soit 
$$\Sigma =\{ 1, 4\}$$
\item \begin{itemize}
\item $\underline{\lambda=1}$
On obtient $S_1 \equivaut x+ 2y =0$
$$\cS_1 =\{ (-2y, y) | y \in \R\}$$

\item $\underline{\lambda=4}$
On obtient $S_4 \equivaut x-y =0$
$$\cS_4 =\{ (x, x) | x \in \R\}$$
\end{itemize}

\item Si $\lambda$ n'est pas dans $ \Sigma$, le système est de Cramer, il admet donc une unique solution. Comme $(0,0)$ est solution, c'est la seule. 
$$\cS_\lambda =\{ (0,0)\}$$
\end{enumerate}
\end{correction}




\begin{exercice}
Résoudre le système suivant où $x,y,z$ sont des réels positifs (on pourra utiliser une fonction qui transforme les $\times$ en $+$.... ): 
$$\left\{ \begin{array}{ccc}
x^2y^2z^6 & =& 1\\
x^4y^5z^{13}& =& 2 \\
x^2yz^7 & =& 3
\end{array}\right. $$
\end{exercice}




\begin{correction}
Comme tous les éléments sont positifs on peut prendre le logartihme. 
On note $X= \ln(x), Y=\ln(y)$ et $Z=\ln(z)$ on obtient : 
$$\left\{ \begin{array}{ccc}
2X +2Y+6Z& =& 0\\
4X+5Y+13Z& =& \ln(2) \\
2X+Y+7Z & =& \ln(3)
\end{array}\right. $$
On résout ensuite le système en $(X,Y,Z)$. Tout d'abord on échelonne le système : 
$$\left\{ \begin{array}{rcl}
2X +2Y+6Z& =& 0\\
4X+5Y+13Z& =& \ln(2) \\
2X+Y+7Z & =& \ln(3)
\end{array}\right. \equivaut 
\left\{ \begin{array}{rcl}
2X +2Y+6Z& =& 0\\
0+Y+Z& =& \ln(2) \\
-Y+Z & =& \ln(3)
\end{array}\right. \equivaut 
\left\{ \begin{array}{rcl}
2X +2Y+6Z& =& 0\\
Y+Z& =& \ln(2) \\
2 Z & =& \ln(3) +\ln(2)
\end{array}\right.$$
Une fois que le système est échelonné, on résout en remontant les lignes. 
On obtient : 
$$\left\{ \begin{array}{rcl}
2X +2Y+6Z& =& 0\\
Y+Z& =& \ln(2) \\
Z & =& \ln(\sqrt{6})
\end{array}\right. \equivaut 
\left\{ \begin{array}{rcl}
2X +2Y+6Z& =& 0\\
Y& =& \ln(\frac{2}{\sqrt{6}}) \\
Z & =& \ln(\sqrt{6})
\end{array}\right. \equivaut 
\left\{ \begin{array}{rcl}
2X & =& -\ln(\frac{4}{6})  -\ln(6^3)\\
Y& =& \ln(\frac{2}{\sqrt{6}}) \\
Z & =& \ln(\sqrt{6})
\end{array}\right.
$$


Soit encore  $X= \ln( \sqrt{\frac{1}{6^2 4}}) $ , $Y = \ln(\frac{2}{\sqrt{6}}) $ et $Z= \ln(\sqrt{6})$. 
D'où $$x= \frac{1}{12}, \quad  y = \frac{2}{\sqrt{6}}  \quadet z= \sqrt{6}$$







\end{correction}

\begin{exercice}
Soient $\suite{u}$ et $\suite{v}$ deux suites réelles définies par 
$$\left\{ 
\begin{array}{rl}
u_0&=0\\
\forall n\geq 0,\, u_{n+1} &= 2u_n +v_n
\end{array}\right. \quad \left\{ 
\begin{array}{rl}
v_0&=1\\
\forall n\geq 0,\,v_{n+1} &= u_n +v_n
\end{array}\right.$$
\end{exercice}

\begin{enumerate}
\item Ecrire une fonction Python qui prend en argument $n$  et retourne la valeur de $u_n$ et $v_n$. 
\item Montrer que pour tout $n\in \N$ :
$$u_{n+2} = 3u_{n+1} -u_n$$
\item En déduire la valeur de $u_n$ en fonction de $n\in \N$.  (Il va y avoir des $\frac{3\pm \sqrt{5}}{2}$ qui trainent mais c'est faisable)
\end{enumerate}

\begin{correction}
\begin{itemize}
\item Fonction prenant $n$ et $u_0, v_0 $ en paramètre. Pour calculer $u_10$ on peut faire par $\texttt{suite(10,1,0)}$ 
\begin{lstlisting}
def suite(n,u0,v0):
    u,v=u0,v0 \# affection simultanee
    for i in range(n):
        u,v=2*u+v, u+v \# affection simultanee
    return(u,v)
\end{lstlisting}

\item D'après la définition de $\suite{u}$ on a :
$$u_{n+2} = 2u_{n+1}+v_{n+1}$$
D'après la définition de $\suite{v}$ on a :
$$v_{n+1} = u_n +v_n$$
Donc 
$$u_{n+2} = 2u_{n+1}+ u_n +v_n$$
et en reprenant la définition de $\suite{u}$ on a :
$$v_n = u_{n+1}-2u_n.$$
Donc 
\begin{align*}
u_{n+2}& = 2u_{n+1}+ u_n +u_{n+1}-2u_n\\
			&= 3u_{n+1}- u_n
\end{align*}
\item On applique la méthode du cours. La suite $\suite{u}$ est une suite linéaire récurrente d'ordre deux à coefficients constant. L'équation caractéristique de $\suite{u}$ est 
$X^2-3X+1=0$ dont les solutions sont 
$$r_1 = \frac{3+\sqrt{5}}{2} \quadet r_2 =\frac{3-\sqrt{5}}{2}$$
Ainsi il existe $(A,B)\in \R^2$ tel que pour tout $n\in\N$:
$$u_n = A \left(\frac{3+\sqrt{5}}{2} \right)^n+B\left(\frac{3-\sqrt{5}}{2} \right)^n$$
Les conditions initiales donnent : 
$$u_ 0  = 0=A+B$$
et 
$$u_1 = 2u_0+v_0= 1=  A \left(\frac{3+\sqrt{5}}{2} \right)+B\left(\frac{3-\sqrt{5}}{2} \right)$$

Le système à résoudre est donc 
$$\left\{\begin{array}{cccc}
A&+&B&=0\\
 \left(\frac{3+\sqrt{5}}{2} \right)A&+&\left(\frac{3-\sqrt{5}}{2} \right)B&=1
\end{array}\right.$$
on fait $L_2\leftarrow L_2- \left(\frac{3+\sqrt{5}}{2} \right)L_1$ On obtient : 
$$\left\{\begin{array}{cccc}
A&+&B&=0\\
 & -&\sqrt{5}B&=1
\end{array}\right.$$

Donc $$A=\frac{1}{\sqrt{5}} \quadet B=-\frac{1}{\sqrt{5}}$$
Finalement 
\conclusion{ $\forall n\in \N,\, u_n= \frac{1}{\sqrt{5}} \left(\frac{3+\sqrt{5}}{2} \right)^n-\frac{1}{\sqrt{5}}\left(\frac{3-\sqrt{5}}{2} \right)^n$}

\end{itemize}

\end{correction}



\end{document}