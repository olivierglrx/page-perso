\documentclass[a4paper, 11pt,reqno]{article}
\input{/Users/olivierglorieux/Desktop/BCPST/2020:2021/preambule.tex}
\newif\ifshow
\showtrue
\input{/Users/olivierglorieux/Desktop/BCPST/2021:2022/ifshow.tex}




\author{Olivier Glorieux}


\begin{document}

\title{DM9  \\
\small{à refaire avant le prochain DS } 
}

\begin{exercice}
Soit $A$ la matrice 
$$A=\left(
\begin{array}{ccc}
0&-1&1\\
4&1&-2\\
2&-2&1
\end{array}
 \right)$$
 
 
\begin{enumerate}
\item Résoudre le système $AX=\lambda X$ d'inconnue $X =\left(
\begin{array}{c}
x\\
y\\
z
\end{array}
 \right)$ où $\lambda$ est un paramètre réel. 
 
 \item Soit $e_1= \left(
\begin{array}{c}
1\\
1\\
2
\end{array}
 \right)$,  $e_1= \left(
\begin{array}{c}
1\\
0\\
2
\end{array}
 \right)$, et  $e_1= \left(
\begin{array}{c}
0\\
1\\
1
\end{array}
 \right)$.
 Calculer $Ae_1, Ae_2$ et $Ae_3$. 
 
\item Montrer par récurrence que $A^ne_1= $. 
\item Par analogie avec la question précédente, donner la valeur de $A^n e_2 $ et $A^n e_3$.
\item Soit $P= \left(
\begin{array}{ccc}
1&1&0\\
1&0&1\\
2&2&1
\end{array}
 \right)$ 
 
 Montrer que $P$ est inversible et calculer son inverse. 
 \item Soit $D=P^{-1}AP$. Calculerr $D$. 
 \item Montrer par récurrence que $D^n = P^{-1}A^n P$
 \item En déduire la valeur de $A^n$. 
\item Soit $\suite{x}, \suite{y} $ et $\suite{z}$ les suites définies par : 
$x_0=1, y_0=1 $ et $z_0=1$ et pour tout $n\in \N$ :
$$\left\{
\begin{array}{cll}
x_{n+1} &= &-y_n+z_n\\
y_{n+1}&=4x_n&+y_n-2z_n\\
z_{n+1}&=2x_n&-2x_n+z_n
\end{array}
 \right.$$ 
Soit $X_n = \left(
\begin{array}{c}
x_{n}\\
y_{n}\\
z_{n}
\end{array}
 \right)$
 
Montrer que $X_{n+1} = A X_n$. 
\item Montrer par récurrence que pour tout $n\in \N$, $$X_n = A^n X_0$$
\item En déduire le terme général de $\suite{x}$ en fonction de $n$. 
\end{enumerate} 
\end{exercice}

\begin{correction}
\begin{enumerate}
\item $AX= \lambda X \equivaut \left\{ \begin{array}{cccc}
&-y&+z=&\lambda x\\
4x &+y&-2z=&\lambda y\\
2x &-2y&+z=&\lambda z
\end{array}
\right.\equivaut \left\{ \begin{array}{cccc}
-\lambda x&-y&+z=&0\\
4x &+(1-\lambda)y&-2z=&0\\
2x &-2y&+(1-\lambda) z=&0
\end{array}
\right.$  
Ensuite on échelonne le système (Après beaucoup de fautes de calculs) on obtient :
\begin{align*}
&\equivaut \left\{ \begin{array}{cccc}
2x &-2y&+(1-\lambda) z=&0\\
0 &+(5-\lambda)y&(-4+2\lambda) z=&0\\
0 &(-\lambda -1) y&+\frac{1}{2}(2-\lambda-\lambda^2) z=&0
\end{array}
\right.\\
&\equivaut \left\{ \begin{array}{cccc}
2x &-2y&+(1-\lambda) z=&0\\
0 &+(5-\lambda)y&2(\lambda-2) z=&0\\
0 &-(\lambda +1) y&-\frac{1}{2}(\lambda+1)(\lambda-2) z=&0
\end{array}
\right.\\
&\equivaut \left\{ \begin{array}{cccc}
2x &+(1-\lambda) z&-2y=&0\\
0 &+2(\lambda-2) z &+(5-\lambda)y=&0\\
0 &-\frac{1}{2}(\lambda+1)(\lambda-2) z&-(\lambda +1) y=&0
\end{array}
\right.
\end{align*}
et enfin $L_3\leftarrow L_3+\frac{1}{4}(\lambda+1) L_2$ donne : 

\begin{align*}
&\equivaut \left\{ \begin{array}{cccc}
2x &+(1-\lambda) z&-2y=&0\\
0 &+2(\lambda-2) z &+(5-\lambda)y=&0\\
0 &0&\frac{1}{4}(-\lambda^2 +1) y=&0
\end{array}
\right.\\
&\equivaut \left\{ \begin{array}{cccc}
2x &+(1-\lambda) z&-2y=&0\\
0 &+2(\lambda-2) z &+(5-\lambda)y=&0\\
0 &0&(-\lambda+1)(\lambda+1) y=&0
\end{array}
\right.
\end{align*}
Donc si $\lambda-2 \neq 0 $ et $(-\lambda+1)(\lambda+1) \neq 0$, le système est de rang 3. Il admet une unique solution à savoir $S=\{(0,0,0)\}$

Si $\lambda=1$
Le système équivaut à 
$$\left\{ \begin{array}{cccc}
2x &  &-2y=&0\\
0 &-2 z &4y=&0\\
0 &0&0=&0
\end{array}
\right.$$
Il est échelonné de rang 2. Les solutions sont de la forme : 
$$\cS=\{ (y,y,2y) \, y\in \R\} $$

Si $\lambda=2$
Le système équivaut à 
$$\left\{ \begin{array}{cccc}
2x & -z &-2y=&0\\
0 & &3y=&0\\
0 &0&-3y=&0
\end{array}
\right. \equivaut\left\{ \begin{array}{cccc}
2x & -z&-2y=&0\\
0 & &3y=&0\\
0 &0&0=&0
\end{array}
\right.  $$
Il est échelonné de rang 2. Les solutions sont de la forme : 
$$\cS=\{ (2x,0,x) \, x\in \R\} $$


Si $\lambda=-1$
Le système équivaut à 
$$\left\{ \begin{array}{cccc}
2x & +2z &-2y=&0\\
0 & -6z&6y=&0\\
0 &0&0=&0
\end{array}
\right. \equivaut\left\{ \begin{array}{cccc}
2x &+ 2z&-2y=&0\\
0 & z&=&y\\
\end{array}
\right.  $$
Il est échelonné de rang 2. Les solutions sont de la forme : 
$$\cS=\{ (0,y,y) \, y\in \R\} $$



\item $Ae_1 = e_1$, $Ae_2 =2e_2$ et $Ae_3 =-e_3$

\item C'est vrai pour $n=1$. On suppose que le résultat est vrai pour un certain entier $n\in \N$, on a alors $A^{n+1} e_1 =A A^ne_1=Ae_1$ par HR. Puis $Ae_1=e_1$ d'après la question précédente. On a alors $A^{n+1}e_1 =e_1$. Par récurrence le résultat est vrai pour tout $n\in \N$

\item $A^n=2^ne_2$ et $A^n e_3 =(-1)^n e_3$

\item cf ex 8 $$P^{-1}= \left(
\begin{array}{ccc}
2&1&-1\\
-1&-1&1\\
-2&0&1
\end{array}
 \right)$$ 

\item cf ex 8 $D= \left(
\begin{array}{ccc}
1&0&0\\
0&2&0\\
0&0&-1
\end{array}
 \right)$ 
 \item cf ex 6
 \item $A^n = PD^n P^{-1}$ 
Or $D^n  =  \left(
\begin{array}{ccc}
1&0&0\\
0&2^n&0\\
0&0&(-1)^n
\end{array}
 \right)$  (ca ne marche QUE pour les matrices diagonales) 
 
 Donc 
 $$A^n = \left(
\begin{array}{ccc}
2-2^n&1-2^n&-1+2^n\\
2-2(-1)^n&1&-1+(-1)^n\\
4-2^{n+1} -2(-1)^n& 2-2^{n+1}&-2+2^{n+1}+(-1)^n
\end{array}
 \right)$$ 
 
 \item $X_{n+1} = \left(
\begin{array}{c}
x_{n+1}\\
y_{n+1}\\
z_{n+1}
\end{array}
 \right)$
 
 et $AX_n = A=\left(
\begin{array}{ccc}
0&-1&1\\
4&1&-2\\
2&-2&1
\end{array}
 \right)  \left(
\begin{array}{c}
x_{n}\\
y_{n}\\
z_{n}
\end{array}
 \right)=\left(
\begin{array}{c}
-y_n+z_{n}\\
4x_n +y_n -2z_{n}\\
2x_n-2y_n+z_{n}
\end{array}
 \right) $
 
 
Ce qui est bien le système vérifiée par les suites $\suite{x}, \suite{y}, \suite{z}$.

\item C'est vrai pour $n=0$ ($A^0=\Id$) C'est aussi vrai pour $n=1$ (calcul) 
On suppose le résultat vrai pour UN $n\in \N$ On a alors :
$X_n = A^nX_0$ et donc $AX_n =A^{n+1}X_0$. Or d'après la question précédente  
$AX_n =X_{n+1}$. La propriété est donc héréditaire et donc vraie pour tout $n\in \N$. 


\item On fait le calcul de $A^n X_0$ grace au résultat trouvé à la question 8. 
On obtient 
$$x_n = 2-2^n +1-2^n -1+2^n = 2-2^n$$

\end{enumerate}
\end{correction}





\end{document}