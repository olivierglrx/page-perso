\documentclass[a4paper, 11pt,reqno]{article}
\input{/Users/olivierglorieux/Desktop/BCPST/2020:2021/preambule.tex}
\newif\ifshow
\showtrue
\input{/Users/olivierglorieux/Desktop/BCPST/2021:2022/ifshow.tex}

\geometry{hmargin=3.0cm, vmargin=3.5cm}
\newenvironment{amatrix}[1]{%
  \left(\begin{array}{@{}*{#1}{c}|c@{}}
}{%
  \end{array}\right)
}

\author{Olivier Glorieux}


\begin{document}

\title{Correction : DS5
}

\begin{exercice}
On condière un ensemble de personnes composé de $n_1$ hommes et de $n_2 $ femmes. On désire élire un bureau de $p$ représentants choisis parmi ces $(n_1+n_2)$ personnes. 
\begin{enumerate}
\item Combien y-a-t-il  de bureaux possibles ? 
\item Combien y-a-t-il  de bureaux possibles contenant exactement $k$ hommes. 
\item En déduire la relation suivante :
Pour tout $(n_1, n_2 , p )\in \N^3$, tel que $p\leq n_1$ et $p\leq n_2$ :
$$\binom{n_1+n_2}{p} =\sum_{k=0}^p \binom{n_1}{k}\binom{n_2}{p-k}$$

\end{enumerate}
\end{exercice}
\begin{correction}
\begin{enumerate}
\item Il y a $\binom{n_1+n_2}{p}$ bureaux possibles (choix de $p$ personnes sans ordre et sans répétition parmi les  $n_1+n_2$ personnes)
\item On choisit les $k$ hommes parmi les $n_1$ possibles et ensuite on choisit $(p-k)$ femmes parmi les $n_2$ possibles. Il y a donc 
$$\binom{n_1}{k}\binom{n_2}{p-k}$$ 
possibilités. 
\item Soit $A$ l'ensemble des bureaux possibles. 
Soit $A_k$ l'ensemble des bureaux  possibles composés de $k$ hommes. 
$(A_k)_{k\in [0,p]}  $ est une partition de $A$ et donc : 
$$Card(A)  =\sum_{k=0}^p Card(A_k)$$
Ainsi 
$$\binom{n_1+n_2}{p}= \sum_{k=0}^p \binom{n_1}{k}\binom{n_2}{p-k}$$ 

\end{enumerate}
\end{correction}

\begin{exercice}
On considère l'équation suivante , d'inconnue $z\in \bC$ : 
\begin{equation}\tag{$E$}
z^3 +z+1=0
\end{equation} 
\begin{enumerate}
\item On note $f : \R\mapsto \R, $ la fonciton définie par $f(t) = t^3+t+1.$
A l'aide de l'étude de $f$, justifier que l'équation $(E)$ possède une unique solution réelle, que l'on notera $r$. Montrer que $r \in ]-1, \frac{-1}{2}[$.
\item On note $z_1$ et $z_2$ les deux autres solutions complexes de $(E)$ qu'on ne cherche pas à calculer. On sait alors que le polynôme $P(X) = X^3+X+1$ se factorise de la manière suivante : 
$$P(X)  = (X-r)(X-z_1) (X-z_2).$$
En déduire que $z_1+z_2=-r$ et $z_1z_2=\frac{-1}{r}$.
\item Justifier l'encadrement  : $\frac{1}{2}<|z_1+z_2 |<1.$\\
De même montrer que  $1< |z_1z_2|< 2.$
\item Rappeler l'inégalité triangulaire. En déduire que pour tout $x,y\in \bC$ 
$$|x-y|\geq |x|-|y|$$
\item Montrer alors que  $$|z_1+z_2| >|z_1| -\frac{2}{|z_1|}$$
\item Grâce à un raisonnement par l'absurde montrer que $|z_1|<2$.
\item Conclure que toutes les solutions de $(E)$ sont de modules strictement inférieures à $2$. 
\end{enumerate}
\end{exercice}


\begin{correction}
\begin{enumerate}
\item  Comme $f(-1)=-1<0$ et $f(\frac{-1}{2}) = \frac{3}{8}>0$, le théoréme des valeurs intermédiaires assure qu'il existe une solution a 
$f(t)=0$ dans l'intervalle $]-1, \frac{-1}{2}[$. De plus 
$f'(t)=2t^2+1$ donc $f'>0$ pour tout $t\in \R$, donc $f$ est strictement croissante et cette racine est unique. 

\item En développant on obtient 
$$P(X) =X^3  +(-r-z_1-z_2) X^2+\alpha X -z_1z_2r$$
On n'est pas obligé de calculer $\alpha$. 
Par identification on obtient : 
$$-r-z_1-z_2=0\quad \text{et} \quad z_1z_2r=-1$$
$$z_1+z_2=-r\quad \text{et} \quad z_1z_2=\frac{-1}{r}$$
($r\neq 0$)

\item On a $\frac{1}{2} < -r < 1$ et $|z_1+z_2 | =|-r|=-r$. D'où 
$$\frac{1}{2} < |z_1+z_2 | < 1.$$

On a $1 < \frac{-1}{r} < 2$ et $|z_1z_2 | =\left|\frac{-1}{r}\right|=  \frac{-1}{r} $. D'où 
$$1 < |z_1z_2 | < 2.$$

\item L'inégalité triangulaire 'inversée' donne 
$$|x-y| \geq |x|-|y|.$$

\item On a donc 
$$|z_1+z_2| \geq |z_1|-|z_2|$$
Or $|z_1z_2| <2$, donc $|z_2| <\frac{2}{|z_1|}$
D'où $-|z_2| >-\frac{2}{|z_1|}$. On obtient donc l'inégalité voulue. 

\item Supposons par l'absurde que $|z_1|\geq 2$. On a alors d'après la questions précédente 
$$|z_1-z_2| > 2 -1 =1$$
Ceci est en contradiction avec le résultat de la question $3$. Donc 
$$|z_1|\leq 2.$$
\item Le raisonement de la question 5 et 6 s'applique de façon similaire à $z_2$. Comme $|r|\leq 1$, toutes les racines de $P$ sont bien de module strictement inférieur à $2$. 

\end{enumerate}
\end{correction}





\begin{exercice}
Dans l'espace $\cE$ muni d'un repère orthonormé, on considère la droite $\cD$ de représenation paramétrique : 
$$\cD : \left\{ 
\begin{array}{ccc}
x&=& -3+t\\
y&=& 1+2t\\
z&=& 2-t
\end{array}
\right.\,  ,\, t\in \R$$

Pour tout ce problème on fixe un point $A(\alpha, \beta, \gamma) $ et on note $H(\lambda, \mu, \nu) $ son projeté orthogonal sur $\cD$. 

\begin{enumerate}
\item  Soit $\cP_1$ le plan contenant les points $B(-1,0,0), C(2,7,-3)$ et $D(-2,4,1)$.
\begin{enumerate}
\item Montrer que $\cP_1$ a pour équation $x+z+1=0$
\item Montrer que $\cP_1$ contient $\cD$. 
\end{enumerate}
\item Soit $\cP_2$ le plan contenant $\cD$ et le point $E(-2,3,-1)$. 
\begin{enumerate}
\item  Donner deux vecteurs parallèles à $\cP_2$ qui ne sont pas colinéaires entre eux. 
\item En déduire un vecteur orthogonal à $\cP_2$.
\item Montrer alors  que $-2x+y-7=0$ est une équation cartésienne de $\cP_2$
\end{enumerate}
\item Soit  $\cP_3$ le plan perpendiculaire à $\cD$ et passant par   $A$. 
\begin{enumerate} 
\item Donner un vecteur directeur de $\cD$. 
\item Déterminer une équation cartésienne de $\cP_3$.  (En fonction évidemment de $\alpha, \beta, \gamma)$
\end{enumerate}

\item En déduire que les coordonnées de $H$ vérifient un système linéaire qu'on peut écrire sous la forme : 
$$M_1 \begin{pmatrix}
\lambda \\
\mu \\
\nu 
\end{pmatrix} = M_2\quad \text{où }  M_1 =\begin{pmatrix}
1 & 0 & 1\\
-2 & 1 & 0\\
1 & 2 &-1 
\end{pmatrix}  $$
et  $M_2\in \cM_{3,1} (\R)$  \text{ est une matrice colonne à déterminer} (qui dépendra de $(\alpha, \beta , \gamma)$ 
\item Montrer que $M_1$ est inversible et calculer son inverse. 
\item En déduire que les coordonnées de $H$ sont données par :
$$\begin{pmatrix}
\lambda \\
\mu \\
\nu 
\end{pmatrix} = P_1 \begin{pmatrix}
\alpha \\
\beta \\
\gamma 
\end{pmatrix} +P_2\quad \text{où }  P_1 =\frac{1}{6}\begin{pmatrix}
1 & 2 & -1\\
2 & 4 & -2\\
-1 & -2 &1 
\end{pmatrix} \text{ et } P_2 =\frac{1}{2}\begin{pmatrix}
-5 \\
4\\
3
\end{pmatrix} $$
%(On pourra calculer séparément $M_1^{-1}M_2$ et $ P_1 \begin{pmatrix}
%\alpha \\
%\beta \\
%\gamma 
%\end{pmatrix} +P_2$)
\item \begin{enumerate}
\item Déterminer en fonction de $\alpha, \beta$ et $\gamma$ la valeur du paramètre $t\in \R$ du point $M(x,y,z)\in \cD$ telle que le veteur $\vec{AM}$ soit orthogonal au vecteur $\vec{u} = (1,2,-1)$
\item Retrouver le résultat de la question 6 à l'aide de la valeur du paramétre $t$ obtenue à la question précédente. 
\end{enumerate}
%\item \begin{enumerate}
%\item A l'aide du résultat de la question 4, montrer que :
%$$P_1 \begin{pmatrix}
%\lambda \\
%\mu \\
%\nu 
%\end{pmatrix}+P_2=\begin{pmatrix}
%\lambda \\
%\mu \\
%\nu 
%\end{pmatrix}. $$
%\item Quelle est l'interprétation géométrique du réqsultat précédent. 
%
%\end{enumerate}
%\item On note $\cS$ l'ensemble des  points $M(x,y,z)\in \cE$ qui verifient la propriété suivante   
%\begin{enumerate}
%\item  $$P_1 \begin{pmatrix}
%x \\
%y\\
%z
%\end{pmatrix} +P_2=\begin{pmatrix}
%x \\
%y\\
%z
%\end{pmatrix}. \quad (\Sigma)$$
%Résoudre le système $(\Sigma)$ et en déduire que $\cS = \cD$
%\end{enumerate}
\end{enumerate}
\end{exercice}

\begin{correction}
\begin{enumerate}
\item \begin{enumerate}
\item Soit $ax+by+cz+d=0$ l'équation du plan $\cP_1$. On a 
\begin{itemize}
\item $B\in \cP_1$ donc $-a+d=0$
\item $C\in \cP_1$ donc $2a+7b-3c+d=0$
\item $D\in \cP_1$ donc $-2a+4b+c+d=0$
\end{itemize}
On obtient 
$$
\left\{ \begin{array}{ccc}
-a+d&=&0\\
7b-3c+3d&=&0\\
4b+c-d&=&0
\end{array}\right.
\equivaut 
\left\{ \begin{array}{ccc}
-a+d&=&0\\
c+4b-d&=&0\\
-3c+7b+3d&=&0
\end{array}\right.
\equivaut 
\left\{ \begin{array}{ccc}
-a+d&=&0\\
c+4b-d&=&0\\
19b&=&0
\end{array}\right.
$$

$$\equivaut 
\left\{ \begin{array}{ccc}
a&=&d\\
c&=&d\\
b&=&0
\end{array}\right.
$$

On obtient donc comme équation du plan $\cP_1$
\conclusion{$x+z+1=0$}

\item Soit $M(x,y,z)\in D$, il existe donc $t\in \R$ tel que 
$$x=-3+t, \quad y=1+2t \quadet z=2-t$$
et donc 
$$x+z+1 = -3+t +2-t+1= 0$$
Ainsi $M\in \cP_1$
\conclusion{ $D\subset \cP_1$} 

\end{enumerate}

\item 
\begin{enumerate}
\item  Le vecteur  $\vec{u}$ de $\cD$ de coordonnées $\vec{u} =\begin{pmatrix}
1\\
2\\
-1
\end{pmatrix}$  est un vecteur directeur de $\cD$ donc est parallèle à $\cP_2$. Le point $F=\begin{pmatrix}
-3\\
1\\
2
\end{pmatrix}$ appartient à $\cD$ donc à $\cP_2$ et ainsi le vecteur $\vec{FE}$ de coordonnées $\begin{pmatrix}
1\\
2\\
-3
\end{pmatrix}$ est aussi parallèle à $\cP_2$ et n'est pas colinéaire à $\vec{u}$. 


\item  On cherche donc un vecteur $\vec{n}$ de coordonnées $\begin{pmatrix}
a\\
b\\
c
\end{pmatrix}$ tel que $\vec{n}\cdot \vec{u}=0$ et $\vec{n} \cdot \vec{FE}=0$
On obtient les deux équations suivantes : 
$$\left\{ \begin{array}{cc}
a+2b-c&=0\\
a+2b-3c&=0
\end{array} \right. \equivaut \left\{ \begin{array}{cc}
a+2b&=0\\
c&=0
\end{array} \right. $$

On peut ainsi prendre \conclusion{$\vec{n}=\begin{pmatrix}
-2\\
1\\
0
\end{pmatrix}$ }
\item L'équation du plan $\cP_2$ est donc de la forme 
$$-2x+y+d=0$$
Comme $E\in cP_2$ on a $-2\times -2 + 3+d =0$ d'où $d=-7$

\conclusion{ Le plan $\cP_2$ a pour équation $-2x+y-7=0$}

\end{enumerate}
\item 
\begin{enumerate}
\item \conclusion{Le vecteur  $\vec{u}$ de $\cD$ de coordonnées $\vec{u} =\begin{pmatrix}
1\\
2\\
-1
\end{pmatrix}$  est un vecteur directeur de $\cD$}
\item L'équation du plan $\cP_3$ est donc de la forme 
$$x+2y-z+d=0$$
Comme $A\in cP_3$ on a $\alpha + 2\beta - \gamma +d =0$ d'où $d=-\alpha -2\beta + \gamma$

\conclusion{ Le plan $\cP_3$ a pour équation $c+2y-z =\alpha + 2\beta - \gamma $}

\end{enumerate}

\item Comme $H$ est le projeté orthogonal sur $\cD$ de $A$, on a $H\in \cD$ donc en particulier à $H\in \cP_1$ (d'après la question 1b) et  à $H\in \cP_2$ par définition de $\cP_2$ 

$H\in \cP_3$ car $\cP_3$ est orthogonal à $\cD$ et passe par $A$ et $H$ est le projeté de $A$ sur $\cD$. 

Ainsi les coordonnées de $H$ vérifient les équations cartésiennes des trois plans : 

$\begin{array}{cc}
\lambda +\nu &=-1\\
-2\lambda +mu &=7\\
\lambda +2\mu -\nu &= \alpha + 2\beta - \gamma 
\end{array}$

En mettant ce système sous forme d'équation matricielle on obtient : 
$$\begin{pmatrix}
1 & 0 & 1\\
-2 & 1 & 0\\
1 & 2 &-1 
\end{pmatrix} \begin{pmatrix}
\lambda\\
\mu\\
\nu
\end{pmatrix} =\begin{pmatrix}
-1\\
7\\
 \alpha + 2\beta - \gamma 
\end{pmatrix} $$

Les coordonnées de $H$ vérifient 
\conclusion{$ M_1 =\begin{pmatrix}
\lambda\\
\mu\\
\nu
\end{pmatrix} =\begin{pmatrix}
-1\\
7\\
 \alpha + 2\beta - \gamma 
\end{pmatrix} $}
où $M_1$ est donné par l'énoncé. 

\item Un pivot de Gauss long et laborieux montre que $M_1$ est inversible et 
\conclusion{$M_1^{-1} = \begin{pmatrix}
1/6 & -1/3 & 1/6\\
1/3 & 1/3 & 1/3\\
5/6 & 1/3 &-1/6 
\end{pmatrix} = \frac{1}{6} \begin{pmatrix}
1 & -2 & 1\\
2 & 2 & 2\\
5 & 2 &-1
\end{pmatrix} $}

\item D'après la question 4 on a 
$$\begin{pmatrix}
\lambda\\
\mu\\
\nu
\end{pmatrix} =M_1^{-1} M_2  $$


Et d'après la question 5 on  a 
$$M_1^{-1}M_2 =\frac{1}{6} \begin{pmatrix}
1 & -2 & 1\\
2 & 2 & 2\\
5 & 2 &-1
\end{pmatrix}   \begin{pmatrix}
-1\\
7\\
 \alpha + 2\beta - \gamma 
\end{pmatrix}   = \frac{1}{6}   \begin{pmatrix}
-1 -14+ \alpha + 2\beta - \gamma \\
-2 +14 +2( \alpha + 2\beta - \gamma )\\
-5+14 -( \alpha + 2\beta - \gamma) 
\end{pmatrix} $$

D'où $$M_1^{-1}M_2 = \frac{1}{6}   \begin{pmatrix}
-15+ \alpha + 2\beta - \gamma \\
12 +2( \alpha + 2\beta - \gamma )\\
9 -( \alpha + 2\beta - \gamma) 
\end{pmatrix} $$

Par ailleurs : 
$$  P_1 \begin{pmatrix}
\alpha \\
\beta \\
\gamma 
\end{pmatrix} +P_2 = \frac{1}{6}\begin{pmatrix}
\alpha+2\beta -\gamma \\
2\alpha+4\beta -2\gamma \\
-\alpha-2\beta +\gamma
\end{pmatrix}
 + \frac{1}{2}\begin{pmatrix}
-5 \\
4 \\
3
\end{pmatrix}  =\frac{1}{6}\begin{pmatrix}
\alpha+2\beta -\gamma \\
2(\alpha+2\beta -\gamma) \\
-(\alpha-2\beta +\gamma) 
\end{pmatrix}
 + \frac{1}{6}\begin{pmatrix}
-15 \\
12 \\
9
\end{pmatrix} $$
Donc 
$$P_1 \begin{pmatrix}
\alpha \\
\beta \\
\gamma 
\end{pmatrix} +P_2 = \frac{1}{6}\begin{pmatrix}
-15+\alpha+2\beta -\gamma \\
12+2(\alpha+2\beta -\gamma) \\
9-(\alpha-2\beta +\gamma) 
\end{pmatrix} = M_1^{-1}M_2
$$

\conclusion{ $ \begin{pmatrix}
\lambda \\
\mu \\
\nu 
\end{pmatrix} = P_1 \begin{pmatrix}
\alpha \\
\beta \\
\gamma 
\end{pmatrix} +P_2 =\frac{1}{6}\begin{pmatrix}
-15+\alpha+2\beta -\gamma \\
12+2(\alpha+2\beta -\gamma) \\
9-(\alpha-2\beta +\gamma) 
\end{pmatrix} $}

\item \begin{enumerate}
\item  Les coordonnées de $M\in \cD$ vérifient $ \left\{ 
\begin{array}{ccc}
x&=& -3+t\\
y&=& 1+2t\\
z&=& 2-t
\end{array}\right.$

Donc $\vec{AM} $ a pour coordonées : $\begin{pmatrix}
-3+t-\alpha\\
1+2t-\beta\\
2-t-\gamma
\end{pmatrix}$

Ce vecteur est orthogona à $\vec{u}$ si et seulement si $u\cdot \vec{AM} =0$ c'est-à-dire :
$$-3+t-\alpha + 2(1+2t-\beta) -1 (2-t-\gamma) =0$$

On obtient  $6t = 3+\alpha +2\beta-\gamma $ donc 

\conclusion{ $t=\frac{ 3+\alpha +2\beta-\gamma}{6}$}

\item $H$ a donc pour coordonnées 
$\begin{pmatrix}
-3+t\\
1+2t\\
2-t
\end{pmatrix}$
avec le $t$ prenant la valeur trouvé dans la question précédente, c'est-à-dire : 
$$\begin{pmatrix}
-3+\frac{ 3+\alpha +2\beta-\gamma}{6}\\
1+2\frac{ 3+\alpha +2\beta-\gamma}{6}\\
2-\frac{ 3+\alpha +2\beta-\gamma}{6}
\end{pmatrix} 
=
\frac{1}{6}\begin{pmatrix}
-18+3+\alpha +2\beta-\gamma\\
6+2(3+\alpha +2\beta-\gamma)\\
12-(3+\alpha +2\beta-\gamma)
\end{pmatrix} 
=
\frac{1}{6}\begin{pmatrix}
-15+\alpha +2\beta-\gamma\\
12+2(\alpha +2\beta-\gamma)\\
9-(\alpha +2\beta-\gamma)
\end{pmatrix} 
$$

On retrouve bien les coordonnées obtenues dans la question $6$. 

\end{enumerate}


\end{enumerate}
\end{correction}

\begin{exercice}
\begin{enumerate}
\item Combien y-a-t-il de codes possibles au MasterMind ? 
\item Combien y-a-t-il de codes possibles au MasterMind  avec exactement 1 rouge ? 
\item Combien y-a-t-il de codes possibles au MasterMind avec au plus 1 rouge ? 
\item Combien y-a-t-il de codes possibles au MasterMind  où toutes les couleurs sont différentes ?
\end{enumerate}

\end{exercice}

\begin{correction}
\begin{enumerate}
\item Choix de 4 couleurs avec ordre avec répétition parmi 5 couleurs 
\conclusion{ Il y a $5^4$ codes possibles. }
\item On choisit la place du pion rouge cela donne $\binom{4}{1} =4$ choix. Puis on choisit 3 pions qui ne sont pas rouges ($4^3$ choix ) et 1 pion rouge $(1^1$ choix)
\conclusion{ Il y  a $4\times 4^3=4^4$ codes avec exactement 1 rouge}
\item Les codes avec au plus 1 rouge sont les codes avec 0 ou 1 rouge. 
Il y a $4^4$ choix avec 0 rouge et $4^4$ choix avec 1 rouge. 

\conclusion{ Il y  a $2\times4^4$ code avec au moins 1 rouge}
\item C'est un choix avec ordre et cette fois sans répétition 
\conclusion{ Il y a $\frac{5!}{1!}=5!$ codes possibles sans répétitions.  }
\end{enumerate}

\begin{lstlisting}[language=Python]
from random import randint

couleur= ['J','R','M','B','V']
def code():
    L=[]
    for i in range(4):
        x=randint(0,4)
        L=L+[couleur[x]]
    return(L)

def place(code, couleur):
    L=[]
    n=len(code)
    for i in range(n):
        if code[i]==couleur:
            L=L+[i]
    return(L)

def compare_deux_couleurs(L1,L2):
    N=0
    B=0
    for couleur in L2:
        if couleur in L1:
            N=N+1
    B=min(len(L1),len(L2)) -N
    return(B,N)

def couleur_distincte(code):
    L=[]
    for c in code:
        if c in L:
            L=L
        else:
            L=L+[c]
    return(L)

def decode(code, proposition):
    N=0
    B=0
    c_distincte=couleur_distincte(code)
    for c in c_distincte:
        L1= place(code, c)
        L2= place(proposition, c)
        Bc,Nc=compare_deux_couleur(L1,L2)
        B,N=B+Bc,N+Nc
    return(B,N)

def transform(S):
    L=[]
    for s in S:
        L=L+[s]
    return(L)

def master_mind():
    code_cherche=code()
    print(code_cherche)
    c=0 
    proposition =[]
    while code_cherche !=proposition and c<12:
        prop= input('quel est le code ?')
        proposition = transform(prop)
        B,N= decode(code_cherche, proposition)
        print('Blanche : ', B, 'Noire :', N)
        c+=1

    if code_cherche==proposition:
        print('gagne')
    else:
        print('perdu')

master_mind()


\end{lstlisting}



\end{correction}


\end{document}